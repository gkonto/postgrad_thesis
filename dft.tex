\setstretch{1.5}
\clearpage
\subsection{Παράδειγμα μετασχηματισμού \en{Fourier}}
Ο μετασχηματισμός \en{Fourier} έχει πολλές εφαρμογές στη φυσική και στη μηχανική. Μετατρέπει μια μαθηματική συνάρτηση χρόνου σε μια συνάρτηση που η μονάδα μέτρησης της είναι συνήθως η συχνότητα. Στη μελέτη των σειρών \en{Fourier}, πολύπλοκες αλλά περιοδικές συναρτήσεις μπορεί να περιγραφούν ως ένα άθροισμα απλών κυμάτων.\cite{wiki_fourier}

Στο παράδειγμα της παραγράφου υλοποιείται ο αλγόριθμος \emph{\en{DFT}} που αποτελεί παραλλαγή του αλγορίθμου μετασχηματισμού \en{Fourier} έχοντας όμως ως κύριο χαρακτηριστικό την έλλειψη συνεχούς μέσου και τον μετασχηματισμό ακολουθίας πεπερασμένων τμημάτων σταθερού μεγέθους.
\clearpage
\subsubsection{Περιγραφή κοινού τμήματος αλγορίθμου μετασχηματισμού \en{Fourier}}
\selectlanguage{english}
\begin{spacing}{1.0}
\begin{lstlisting}[showstringspaces=false, language=C++, caption={\en{DFT: main()}}, frame=tb]{Name}
int main( int argc, char **argv ) {
    Opts o;
    parseArgs(argc, argv, o);

    double *real = new double[o.size];
    double *imag = new double[o.size];
    fill_arr(real, o.size);
    //fill_arr(imag, o.size);
    double *real_out = new double[o.size];
    double *imag_out = new double[o.size];
    double *ireal_out = nullptr;
    double *iimag_out = nullptr;

    double start = omp_get_wtime();
    dft(real, imag, real_out, imag_out, o.size, 0);
    double end = omp_get_wtime();

    if (o.verify) {
        ireal_out = new double[o.size];
        iimag_out = new double[o.size];
        compute_dft_real_pair(real_out, imag_out, 
        	ireal_out, iimag_out,
        	o.size, 1);
        verify(real, imag, ireal_out, iimag_out, o.size);
    }
    std::cout << "Execution Time: " << std::fixed <<
                 std::setprecision(3) << end - start << 
                 " sec" << std::endl;

    delete []real;
    delete []imag;
    delete []real_out;
    delete []imag_out;
    if (ireal_out) delete []ireal_out;
    if (iimag_out) delete []iimag_out;

    return 0;
}

\end{lstlisting}
\end{spacing}
\selectlanguage{greek}
\clearpage

\subsubsection{Σειριακή εκτέλεση}
Το πρόβλημα στη σειριακή του μορφή αποτελείται από έναν διπλό βρόγχο επανάληψης, 
μεγέθους ίσου με το μέγεθος της συστοιχίας που δίνεται ως δεδομένο για τον μετασχηματισμό. Στο πρόβλημα εισάγονται δύο συστοιχίες ως δεδομένα, μια που αντιπροσωπεύει το πραγματικό μέρος των μιγαδικών αριθμών και μια που αντιπροσωπεύει το φανταστικό μέρος. Τα αποτελέσματα των υπολογισμών αποθηκεύονται σε δυο νέες συστοιχίες ίδιου μεγέθους.
\selectlanguage{english}
\begin{spacing}{1.0}
\begin{lstlisting}[language=C++, caption={\en{DFT: }\el{Σειριακή εκτέλεση}} , frame=tb]{Name}
void dft(const double *inreal, const double *inimag,
        double *outreal, double *outimag, size_t n, int inverse) {
         
    for (size_t k = 0; k < n; k++) {  // For each output element
        double sumreal = 0;
        double sumimag = 0;
        for (size_t t = 0; t < n; t++) {  // For each input element
           double angle = 2 * M_PI * t * k / n;
           if (inverse) angle = -angle;
           sumreal +=  inreal[t] * cos(angle) +
           			 inimag[t] * sin(angle);
           sumimag += -inreal[t] * sin(angle) +
          			 inimag[t] * cos(angle);
        }
        if (inverse) {
            outreal[k] = sumreal/n;
            outimag[k] = sumimag/n;
        } else {
            outreal[k] = sumreal;
            outimag[k] = sumimag;
        }
    }
}
\end{lstlisting}
\end{spacing}
\selectlanguage{greek}

\begin{table}[h]
    \centering
    \caption{\en{DFT}: Επιλογές μεταγλώττισης \en{Alt1, Alt2}}
    \label{my-label}
    \begin{tabular}{
    |p{0.1\textwidth}
    | >{\centering\arraybackslash}p{0.8\textwidth}
    |}
    \hline
 {\textbf{\en{Label}}} & \textbf{\en{Options}} \\ \hline
     \textbf{\en{Alt1}} & \en{-fopt-info-vec=builds/alt1.log -O2 -fno-inline -fno-tree-vectorize -fopenmp -o ./builds/Alt1} \\ \hline
      \textbf{\en{Alt2}} & \en{-fopt-info-vec=builds/alt2.log -O2 -fno-inline -ftree-vectorize -fopenmp -o ./builds/Alt2} \\ \hline
    \end{tabular}
\end{table}
\clearpage

\begin{table}[h]
    \centering
	\captionof{table}{\en{DFT:} Αποτελέσματα \en{Alt1}, \en{Alt2}}
    \label{my-label}
    \resizebox{0.7\textwidth}{!} {
    \begin{tabular}{|p{0.20\textwidth}
    | >{\centering\arraybackslash}p{0.12\textwidth}
    | >{\centering\arraybackslash}p{0.12\textwidth}
    |}
    \hline
    \multirow{2}{*}{\textbf{\shortstack{Μέγεθος\\προβλήματος}}} & \multicolumn{2}{|c|}{\textbf{Χρόνοι εκτέλεσης \en{(sec)}}} \\ \cline{2-3} 
               & \textbf{\en{Alt1}} & \textbf{\en{Alt2}}\\ \hline
     1000 & 0.142 & 0.142 \\ \cline{1-3} 
     2000 & 0.537 & 0.535 \\ \cline{1-3} 
     3000 & 1.185 & 1.184 \\ \cline{1-3} 
     4000 & 2.072 & 2.070 \\ \cline{1-3} 
     5000 & 3.164 & 3.230 \\ \cline{1-3} 
     6000 & 4.635 & 4.631 \\ \cline{1-3} 
     7000 & 6.234 & 6.220 \\ \cline{1-3} 

    \end{tabular}}
\end{table}

\clearpage


\subsubsection{Παραλλαγή με \en{parallel for} (1)}
Στην υλοποίηση της παραγράφου, ο πρώτος βρόγχος επανάληψης εκτελείται παράλληλα με τη χρήση της οδηγίας \en{pragma omp parallel for}.
\selectlanguage{english}
\begin{spacing}{1.2}
\begin{lstlisting}[language=C++, caption={\en{DFT: omp parallel for}} , frame=tb]{Name}
void dft(const double *inreal, const double *inimag,
        double *outreal, double *outimag, size_t n, int inverse) {

#pragma omp parallel for
    for (size_t k = 0; k < n; k++) {  // For each output element
        double sumreal = 0;
        double sumimag = 0;
        for (size_t t = 0; t < n; t++) {  // For each input element
            double angle = 2 * M_PI * t * k / n;
            if (inverse) angle = -angle;
            sumreal +=  inreal[t] * cos(angle) + 
            		 inimag[t] * sin(angle);
            sumimag += -inreal[t] * sin(angle) +
            		 inimag[t] * cos(angle);
        }
        if (inverse) {
            outreal[k] = sumreal/n;
            outimag[k] = sumimag/n;
        } else {
            outreal[k] = sumreal;
            outimag[k] = sumimag;
        }
    }
}
\end{lstlisting}
\end{spacing}
\selectlanguage{greek}

\begin{table}[h]
    \centering
    \caption{\en{DFT}: Επιλογές μεταγλώττισης \en{Alt3, Alt4}}
    \label{my-label}
    \begin{tabular}{
    |p{0.1\textwidth}
    | >{\centering\arraybackslash}p{0.8\textwidth}
    |}
    \hline
 {\textbf{\en{Label}}} & \textbf{\en{Options}} \\ \hline
     \textbf{\en{Alt3}} & \en{-fopt-info-vec=builds/alt3.log -O2 -fno-inline -fno-tree-vectorize -fopenmp -o ./builds/Alt3} \\ \hline
      \textbf{\en{Alt4}} & \en{-fopt-info-vec=builds/alt4.log -O2 -fno-inline -ftree-vectorize -fopenmp -o ./builds/Alt4} \\ \hline
    \end{tabular}
\end{table}
\clearpage

\begin{table}[h]
    \centering
	\captionof{table}{\en{DFT:} Αποτελέσματα \en{Alt3}, \en{Alt4}}
    \label{my-label}
    \resizebox{0.7\textwidth}{!} {
    \begin{tabular}{|p{0.20\textwidth}
    | >{\centering\arraybackslash}p{0.12\textwidth}
    | >{\centering\arraybackslash}p{0.12\textwidth}
    |}
    \hline
    \multirow{2}{*}{\textbf{\shortstack{Μέγεθος\\προβλήματος}}} & \multicolumn{2}{|c|}{\textbf{Χρόνοι εκτέλεσης \en{(sec)}}} \\ \cline{2-3} 
               & \textbf{\en{Alt3}} & \textbf{\en{Alt4}}\\ \hline
     1000 & 0.029 & 0.026 \\ \cline{1-3} 
     2000 & 0.052 & 0.051 \\ \cline{1-3} 
     3000 & 0.100 & 0.091 \\ \cline{1-3} 
     4000 & 0.149 & 0.152 \\ \cline{1-3} 
     5000 & 0.216 & 0.225 \\ \cline{1-3} 
     6000 & 0.313 & 0.307 \\ \cline{1-3} 
     7000 & 0.402 & 0.401 \\ \cline{1-3} 

    \end{tabular}}
\end{table}

\paragraph{Παρατηρήσεις}
\ \\

\begin{figure}[h]
\begin{tabular}{*{2}{>{\centering\arraybackslash}b{\dimexpr0.5\linewidth-2\tabcolsep\relax}}}
\resizebox{0.45\textwidth}{!} {
\begin{tikzpicture}[state/.append style={minimum size=7mm}]
     \begin{axis}[
         xlabel={Μέγεθος διανυσμάτων},
         ylabel={Χρόνος εκτέλεσης},
         xmin=10, xmax=7000,
         ymin=0, ymax=7,
         xtick={ 1000, 2000, 3000, 4000, 5000, 6000, 7000},
         ytick={ 0, 1, 2, 3, 4,5, 6, 7},
         legend pos=north west,
        % ymajorgrids=true,
        % grid style=dashed,
     ] 
 	

 	\addplot[ color=red, mark=square,]
      coordinates {
          (1000, 0.142)(2000, 0.537)(3000, 1.185)
          (4000, 2.072)(5000, 3.164)(6000, 4.635)
          (7000, 6.234)
 	};
 	\addlegendentry{\en{Alt1}}
 	
 	\addplot[ color=blue, mark=triangle,]
      coordinates {
          (1000, 0.029)(2000, 0.052)(3000, 0.1)
          (4000, 0.149)(5000, 0.216)(6000, 0.313)
          (7000, 0.402)
 	};
 	\addlegendentry{\en{Alt3}}
 	 	
     \end{axis}
 \end{tikzpicture}}

	\captionof{table}{\en{DFT:} Σύγκριση \en{Alt1} \en{Alt3}}
    &
\renewcommand{\arraystretch}{1.1}
\resizebox{0.45\textwidth}{!} {
\begin{tabular}{c|c}
Μέγεθος & Ποσοστό μείωσης χρόνου (\%)  \\
\hline
\en{1000} & 80 \\
\en{2000} & 90 \\
\en{3000} & 92 \\
\en{4000} & 92 \\
\en{5000} & 92 \\
\en{6000} & 93\\
\en{7000} & 93 \\

\end{tabular}}
\captionof{table}{\en{DFT}: Ποσοστιαία σύγκριση \en{Alt1} και \en{Alt3}}
\end{tabular}
\end{figure}

\clearpage


\subsubsection{Παραλλαγή με \en{parallel for} (2)}
\selectlanguage{english}
\begin{spacing}{0.5}
\begin{lstlisting}[language=C++, caption={\en{DFT: omp parallel for collapse critical}} , frame=tb]{Name}
void dft(const double *inreal, const double *inimag,
        double *outreal, double *outimag, size_t n, int inverse) {

        double sumreal = 0.0;
        double sumimag = 0.0;
#pragma omp parallel for collapse(2)
    for (size_t k = 0; k < n; k++) {  // For each output element
        for (size_t t = 0; t < n; t++) {  // For each input element
            double angle = 2 * M_PI * t * k / n;
            if (inverse) {
                angle = -angle;
#pragma omp critical
                {
                    outreal[k] += (inreal[t] * cos(angle) +
                     	inimag[t] * sin(angle))/n;
                    outimag[k] += (-inreal[t] * sin(angle) +
                     	inimag[t] * cos(angle))/n;
                }
            } else {
#pragma omp critical
                {
                    outreal[k] +=  inreal[t] * cos(angle) +
                   		inimag[t] * sin(angle);
                    outimag[k] += -inreal[t] * sin(angle) +
                    	 	inimag[t] * cos(angle);
                }
            }
        }
    }
}
\end{lstlisting}
\end{spacing}
\selectlanguage{greek}

\begin{table}[h]
    \centering
    \caption{\en{DFT}: Επιλογές μεταγλώττισης \en{Alt5, Alt6}}
    \label{my-label}
        \resizebox{0.9\textwidth}{!} {
    \begin{tabular}{
    |p{0.1\textwidth}
    | >{\centering\arraybackslash}p{0.8\textwidth}
    |}
    \hline
 {\textbf{\en{Label}}} & \textbf{\en{Options}} \\ \hline
     \textbf{\en{Alt5}} & \en{-fopt-info-vec=builds/alt5.log -O2 -fno-inline -fno-tree-vectorize -fopenmp -o ./builds/Alt5} \\ \hline
      \textbf{\en{Alt6}} & \en{-fopt-info-vec=builds/alt6.log -O2 -fno-inline -ftree-vectorize -fopenmp -o ./builds/Alt6} \\ \hline
    \end{tabular}}
\end{table}

\begin{table}[h]
    \centering
    \caption{\en{DFT}: Αποτελέσματα \en{Alt5, Alt6}}
    \label{my-label}
    \resizebox{0.6\textwidth}{!} {
    \begin{tabular}{|p{0.20\textwidth}
    | >{\centering\arraybackslash}p{0.12\textwidth}
    | >{\centering\arraybackslash}p{0.12\textwidth}
    |}
    \hline
    \multirow{2}{*}{\textbf{\shortstack{Μέγεθος\\προβλήματος}}} & \multicolumn{2}{|c|}{\textbf{Χρόνοι εκτέλεσης \en{(sec)}}} \\ \cline{2-3} 
               & \textbf{\en{Alt5}} & \textbf{\en{Alt6}}\\ \hline
     1000 & 2.170 & 2.086 \\ \cline{1-3} 
     2000 & 9.141 & 9.711 \\ \cline{1-3} 

    \end{tabular}}
\end{table}
\clearpage

\paragraph{Παρατηρήσεις}
\ \\
Η χρήση της οδηγίας \en{critical} όχι μόνο έδωσε στο πρόβλημα σειριακή μορφή, αλλά αύξησε κατα πολύ τους χρόνους εκτέλεσης λόγω των υποκείμενων κλειδωμάτων των μεταβλητών από τα εκάστοτε νήματα που τις μορφοποιούν.

\subsubsection{Παραλλαγή με \en{simd}}
\selectlanguage{english}
\begin{spacing}{0.6}
\begin{lstlisting}[language=C++, caption={\en{DFT: omp simd}}, frame=tb]{Name}
void dft(const double *inreal, const double *inimag,
        double *outreal, double *outimag, size_t n, int inverse) {

#pragma omp simd
    for (size_t k = 0; k < n; k++) {  // For each output element
        double sumreal = 0;
        double sumimag = 0;
        for (size_t t = 0; t < n; t++) {  // For each input element
            double angle = 2 * M_PI * t * k / n;
            if (inverse) angle = -angle;
            sumreal +=  inreal[t] * cos(angle) +
            		 inimag[t] * sin(angle);
            sumimag += -inreal[t] * sin(angle) +
            		 inimag[t] * cos(angle);
        }
        if (inverse) {
            outreal[k] = sumreal/n;
            outimag[k] = sumimag/n;
        } else {
            outreal[k] = sumreal;
            outimag[k] = sumimag;
        }
    }
}
\end{lstlisting}
\end{spacing}
\selectlanguage{greek}

\begin{table}[h]
    \centering
    \caption{\en{DFT}: Επιλογές μεταγλώττισης \en{Alt7, Alt8, Alt9}}
    \label{my-label}
    \resizebox{0.9\textwidth}{!} {
    \begin{tabular}{
    |p{0.1\textwidth}
    | >{\centering\arraybackslash}p{0.8\textwidth}
    |}
    \hline
 {\textbf{\en{Label}}} & \textbf{\en{Options}} \\ \hline
     \textbf{\en{Alt7}} & \en{-fopt-info-vec=builds/alt7.log -O2 -fno-inline -fno-tree-vectorize -fopenmp -o ./builds/Alt7} \\ \hline
      \textbf{\en{Alt8}} & \en{-fopt-info-vec=builds/alt8.log -O2 -fno-inline -ftree-vectorize -fopenmp -o ./builds/Alt8} \\ \hline
      \textbf{\en{Alt9}} & \en{-fopt-info-vec=builds/alt9.log -O2 -fno-inline -ftree-vectorize -fopenmp -o ./builds/Alt9} \\ \hline
    \end{tabular}}
\end{table}
\clearpage

\begin{table}[h]
    \centering
	\captionof{table}{\en{DFT:} Αποτελέσματα \en{Alt7}, \en{Alt8, Alt9}}
    \label{my-label}
    \resizebox{0.6\textwidth}{!} {
    \begin{tabular}{|p{0.20\textwidth}
    | >{\centering\arraybackslash}p{0.12\textwidth}
    | >{\centering\arraybackslash}p{0.12\textwidth}
    | >{\centering\arraybackslash}p{0.12\textwidth}
    |}
    \hline
    \multirow{2}{*}{\textbf{\shortstack{Μέγεθος\\προβλήματος}}} & \multicolumn{3}{|c|}{\textbf{Χρόνοι εκτέλεσης \en{(sec)}}} \\ \cline{2-4} 
               & \textbf{\en{Alt7}} & \textbf{\en{Alt8}} & \textbf{\en{Alt9}}\\ \hline
     1000 & 0.142 & 0.140 & 0.144\\ \cline{1-4} 
     2000 & 0.532 & 0.531 & 0.535\\ \cline{1-4} 
     3000 & 1.185 & 1.175 & 1.183\\ \cline{1-4} 
     4000 & 2.065 & 2.054 & 2.064\\ \cline{1-4} 
     5000 & 3.162 & 3.138 & 3.162\\ \cline{1-4} 
     6000 & 4.617 & 4.593 & 4.618\\ \cline{1-4} 
     7000 & 6.207 & 6.198 & 6.202\\ \cline{1-4} 

    \end{tabular}}
\end{table}


\subsubsection{Παραλλαγή με \en{parallel for simd}}
\selectlanguage{english}
\begin{spacing}{1.2}
\begin{lstlisting}[language=C++, caption={\en{DFT: omp parallel for simd}} , frame=tb]{Name}
void dft(const double *inreal, const double *inimag,
        double *outreal, double *outimag, size_t n, int inverse) {

#pragma omp parallel for simd
    for (size_t k = 0; k < n; k++) {  // For each output element
        double sumreal = 0;
        double sumimag = 0;
        for (size_t t = 0; t < n; t++) {  // For each input element
            double angle = 2 * M_PI * t * k / n;
            if (inverse) angle = -angle;
            sumreal +=  inreal[t] * cos(angle) + 
            		inimag[t] * sin(angle);
            sumimag += -inreal[t] * sin(angle) +
            		inimag[t] * cos(angle);
        }
        if (inverse) {
            outreal[k] = sumreal/n;
            outimag[k] = sumimag/n;
        } else {
            outreal[k] = sumreal;
            outimag[k] = sumimag;
        }
    }
}
\end{lstlisting}
\end{spacing}
\selectlanguage{greek}
\clearpage
\begin{table}
    \centering
    \caption{\en{DFT}: Επιλογές μεταγλώττισης \en{Alt10, Alt11, Alt12}}
    \label{my-label}
    \resizebox{0.8\textwidth}{!} {
    \begin{tabular}{
    |p{0.1\textwidth}
    | >{\centering\arraybackslash}p{0.8\textwidth}
    |}
    \hline
 {\textbf{\en{Label}}} & \textbf{\en{Options}} \\ \hline
     \textbf{\en{Alt10}} & \en{-fopt-info-vec=builds/alt10.log -O2 -fno-inline -fno-tree-vectorize -fopenmp -o ./builds/Alt10} \\ \hline
      \textbf{\en{Alt11}} & \en{-fopt-info-vec=builds/alt11.log -O2 -fno-inline -ftree-vectorize -fopenmp -o ./builds/Alt11} \\ \hline
      \textbf{\en{Alt12}} & \en{-fopt-info-vec=builds/alt12.log -O2 -fno-inline -ftree-vectorize -fopenmp -o ./builds/Alt12} \\ \hline
    \end{tabular}}
\end{table}

\begin{table}[h]
    \centering
	\captionof{table}{\en{DFT:} Αποτελέσματα \en{Alt10}, \en{Alt11, Alt12}}
    \label{my-label}
    \resizebox{0.7\textwidth}{!} {
    \begin{tabular}{|p{0.20\textwidth}
    | >{\centering\arraybackslash}p{0.12\textwidth}
    | >{\centering\arraybackslash}p{0.12\textwidth}
    | >{\centering\arraybackslash}p{0.12\textwidth}
    |}
    \hline
    \multirow{2}{*}{\textbf{\shortstack{Μέγεθος\\προβλήματος}}} & \multicolumn{3}{|c|}{\textbf{Χρόνοι εκτέλεσης \en{(sec)}}} \\ \cline{2-4} 
               & \textbf{\en{Alt10}} & \textbf{\en{Alt11}} & \textbf{\en{Alt12}}\\ \hline
	 1000 & 0.024 & 0.023 & 0.024\\ \cline{1-4}  
     2000 & 0.060 & 0.063 & 0.044\\ \cline{1-4} 
     3000 & 0.102 & 0.094 & 0.102\\ \cline{1-4} 
     4000 & 0.138 & 0.143 & 0.149\\ \cline{1-4} 
     5000 & 0.224 & 0.223 & 0.207\\ \cline{1-4} 
     6000 & 0.348 & 0.324 & 0.301\\ \cline{1-4} 
     7000 & 0.416 & 0.413 & 0.404\\ \cline{1-4} 
     
    \end{tabular}}
\end{table}

\paragraph{Παρατηρήσεις}
\ \\
\begin{figure}[h]
\centering 
\resizebox{0.4\textwidth}{!} {
\begin{tikzpicture}[state/.append style={minimum size=7mm}]
    \begin{axis}[
         title={Χρόνοι εκτέλεσης \en{Alt12 - Alt19}},
         xlabel={Μέγεθος διανυσμάτων},
         ylabel={Χρόνος εκτέλεσης},
         xmin=0, xmax=7000,
         ymin=0, ymax=0.5,
         xtick={ 1000, 3000, 5000, 7000},
         ytick={ 0, 0.1, 0.2, 0.3, 0.4, 0.5},
         legend pos=north west,
        % ymajorgrids=true,
        % grid style=dashed,
     ] 
 	
 	\addplot[ color=red, mark=square,]
      coordinates {
          (1000, 0.024)(2000, 0.06)(3000, 0.102)
          (4000, 0.138)(5000, 0.224)(6000, 0.348)
          (7000, 0.416)
 	};
 	\addlegendentry{\en{Alt1}}
 	
 	\addplot[ color=blue, mark=triangle,]
      coordinates {
          (1000, 0.029)(2000, 0.052)(3000, 0.1)
          (4000, 0.149)(5000, 0.216)(6000, 0.313)
          (7000, 0.402)
 	};
 	\addlegendentry{\en{Alt10}}
 	
     \end{axis}
 \end{tikzpicture}}
 \captionof{table}{\en{DFT:} Σύγκριση \en{Alt1} \en{Alt10}}

\end{figure}
