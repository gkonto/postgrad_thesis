\documentclass[12pt]{article}
\usepackage{fullpage}
\usepackage[titletoc,toc,page]{appendix}
\usepackage{setspace}
\usepackage{titlesec}
\usepackage{blindtext}
\usepackage{graphicx}
\usepackage{amssymb}
\usepackage[main=greek, english]{babel}
\usepackage[utf8]{inputenc}
\usepackage[labelfont=bf]{caption}
\usepackage{kerkis}
\usepackage{url}
\usepackage[titles]{tocloft}
\usepackage[top=2.5cm, bottom=2.5cm, left=2.8cm, right=3cm]{geometry}

\renewcommand\cftsecfont{\normalfont}
\renewcommand\cftsecpagefont{\normalfont}

\newcommand{\en}[1]{\foreignlanguage{english}{#1}}

\titlespacing*{\subsection}
{0pt}{2.5ex plus 1ex minus .2ex}{0.3ex plus .2ex}
\titlespacing*{\subsubsection}
{0pt}{2.5ex plus 1ex minus .2ex}{0.3ex plus .2ex}
\pagenumbering{gobble}

\begin{document}
 \begin{center}
\selectlanguage{greek}

\textsc{ ΠΑΝΕΠΙΣΤΗΜΙΟ ΜΑΚΕΔΟΝΙΑΣ\\[0.3 cm]
ΠΡΟΓΡΑΜΜΑ ΜΕΤΑΠΤΥΧΙΑΚΩΝ ΣΠΟΥΔΩΝ\\[0.3 cm]
ΤΜΗΜΑΤΟΣ ΕΦΑΡΜΟΣΜΕΝΗΣ ΠΛΗΡΟΦΟΡΙΚΗΣ}\\[2.5 cm]
{ \large 
ΠΑΡΑΛΛΗΛΟΣ ΠΡΟΓΡΑΜΜΑΤΙΣΜΟΣ ΜΕ ΧΡΗΣΗ \en{OpenMP}\\[0.4 cm] }
Διπλωματική Εργασία\\[1 cm]
του\\[0.5 cm]
\large
Κοντογιάννη Γεώργιου
\begin{minipage}{0.4\textwidth}
\end{minipage}
\vfill
{\large Θεσσαλονίκη, Οκτώβριος 2020}

 \end{center}
 
\pagenumbering{gobble}
\newpage
\mbox{}


\newpage
\pagenumbering{roman}
\setcounter{page}{3} 

 \begin{center}
{\large {ΠΑΡΑΛΛΗΛΟΣ ΠΡΟΓΡΑΜΜΑΤΙΣΜΟΣ ΜΕ ΧΡΗΣΗ \en{OpenMP}}}\\[2 cm]
Κοντογιάννης Γεώργιος\\[0.5 cm]
Δίπλωμα Πολιτικού Μηχανικού, ΑΠΘ, 2016\\[2 cm]
Διπλωματική Εργασία\\[0.5 cm]
υποβαλλόμενη για τη μερική εκπλήρωση των απαιτήσεων του\\[0.5 cm]
ΜΕΤΑΠΤΥΧΙΑΚΟΥ ΤΙΤΛΟΥ ΣΠΟΥΔΩΝ ΣΤΗΝ ΕΦΑΡΜΟΣΜΕΝΗ ΠΛΗΡΟΦΟΡΙΚΗ\\[2 cm]
\begin{flushleft}
Επιβλέπων Καθηγητής\\
Μαργαρίτης Κωνσταντίνος
\vfill
Εγκρίθηκε από την τριμελή εξεταστική επιτροπή την ηη/μμ/εεεε\\[0.5 cm]
\begin{tabular}{  p{\dimexpr 0.3333\linewidth-2\tabcolsep} 
                   p{\dimexpr 0.3333\linewidth-2\tabcolsep} 
                   p{\dimexpr 0.3333\linewidth-2\tabcolsep}  }
Ονοματεπώνυμο 1 & Ονοματεπώνυμο 2  & Ονοματεπώνυμο 3 \\[1 cm]
\dotfill & \dotfill  & \dotfill \\
\end{tabular}\\[2 cm]
Κοντογιάννης Γεώργιος \\[0.5 cm]
\begin{tabular}{  p{\dimexpr 0.3333\linewidth-2\tabcolsep}   }
\dotfill
\end{tabular}\\[1 cm]
\end{flushleft}
\end{center}
  
\setstretch{1.5}

\clearpage
\begin{small}
\begin{flushleft}
\
\vfil
\emph{Η σύνταξη της παρούσας εργασίας έγινε στο   \begin{LARGE}\en{\LaTeX}\end{LARGE}}
\end{flushleft}
\vfil
\end{small}



\clearpage
\begin{flushleft}
{\large \textbf{Περίληψη}}\\[0.5 cm]
\end{flushleft}

\subparagraph{}
Αντικείμενο της παρούσας διπλωματικής εργασίας είναι η μελέτη του \en{OpenMP}, ενός πρότυπου παράλληλου προγραμματισμού, που δίνει στο χρήστη τη δυνατότητα αναπτύξης παράλληλων προγραμμάτων για συστήματα μοιραζόμενης μνήμης, τα οποία  είναι ανεξάρτητα από τη συγκεκριμένη αρχιτεκτονική και έχουν μεγάλη ικανότητα κλιμάκωσης\cite{pdplab}.

Σκοπός της εργασίας είναι η μελέτη και συνοπτική περιγραφή των κύριων χαρακτηριστικών του \en{OpenMP 2.5} αλλά και των νεότερων εκδόσεων 3.0 και 4.5 και η υλοποίηση αλγορίθμων σειριακά και παράλληλα εκτελέσιμων, με σκοπό τη συγκριτική μελέτη της απόδοσής τους. Για την παράλληλη υλοποίηση θα γίνει χρήση της Διεπαφής Προγραμματισμού Εφαρμογών \en{(Application Programming Interface {-} API) OpenMP}, με χαρακτηριστικά που εισήχθησαν στις εκδόσεις \en{OpenMP} 3.0 που δημοσιεύθηκε το 2008 και \en{OpenMP} 4.5 που δημοσιεύθηκε 2015. Χρησιμοποιήθηκαν επίσης χαρακτηριστικά παλαιότερων εκδόσεων\cite{thenextstep}.

Τον Μαιο του 2008 κυκλοφόρησαν οι προδιαγραφές του \en{OpenMP} 3.0 με την εισαγωγή των διεργασιών \en{(Tasking)} αλλά και βελτιώσεις στη \en{C++}. Αυτή ήταν η πρώτη ενημέρωση από την έκδοση 2.5 με σημαντικές βελτιώσεις. Το 2011 κυκλοφόρησε το \en{OpenMP} 3.1 χωρίς καινούργιο χαρακτηριστικά. Νέα λειτουργικότητα υλοποιήθηκε στο \en{OpenMP} 4.0 που κυκλοφόρησε τον Ιούλιο του 2013, όπου έγινε υποστήριξη της αρχιτεκτονικής \en{cc-NUMA}, του ετερογενούς προγραμματισμού, της διαχείρισης σφαλμάτων στο μπλοκ παράλληλου κώδικα και της διανυσματικοποίησης μέσω \en{SIMD}. Τον Ιούλιο του 2015 σημαντική βελτίωση έγινε στα παραπάνω χαρακτηριστικά με την έκδοση \en{OpenMP} 4.5[2].

Τα προαναφερθέντα χαρακτηριστικά χρησιμοποιήθηκαν για την υλοποίηση των αλγορίθμων 
με διαφορετικές εναλλακτικές μεθόδους, με στόχο τη συγκριτική μελέτη τους για την εξαγωγή συμπερασμάτων αναφορικά με τη βελτίωση της απόδοσης σε σχέση με τη σειριακή υλοποίηση αλλά και τη μεταξύ τους σύγκριση καθώς επίσης, και αξιολόγηση της ευχρηστίας της υλοποίησής τους. Στόχος της έρευνας είναι να βρεθούν οι καλύτερες υλοποιήσεις των αλγορίθμων με την επίτευξη της μέγιστης αξιοποίησης της χρήσης \en{CPU} και/ή \en{GPU}. Ακόμη, γίνεται καταγραφή και αναφορά των προβλημάτων που μπορεί να προκύψουν για κάποια υλοποίηση.

Για την παραλληλοποίηση κώδικα, απαιτείται η σχεδίαση με τέτοιο τρόπο ώστε να παράγεται ένας μεγάλος αριθμός παράλληλων λειτουργιών που εκτελούνται από διαφορετικούς επεξεργαστές. Οι  αλγόριθμοι που χρησιμοποιήθηκαν στην παρούσα εργασία περιέχουν ένα μεγάλο αριθμών λειτουργιών, ικανών να εκτελεστούν παράλληλα. 

Τα βασικότερα παραδείγματα που χρησιμοποιήθηκαν είναι:
\begin{itemize}
    \item μετασχηματισμός \en{Fourier}, 
    \item \en{mergesort},
    \item υπολογισμός $\pi$, 
    \item πολλαπλασιασμός πινάκων,
    \item απλή εξίσωση διάδοσης θερμότητας,
    \item παραγοντοποίηση \en{cholensky}.
\end{itemize}

      
Για να υπάρχει άμεση σύγκριση των αποτελεσμάτων ο βασικός κορμός υλοποίησης είναι ο ίδιος για κάθε εφαρμογή, και οι χρονικές καταγραφές έγιναν σε συγκεκριμένα τμήματα του κώδικα. Οι παραλλαγές του κάθε αλγόριθμοι χρησιμοποιούν την \en{CPU} με απλή εκτέλεση χωρίς παραλληλοποίηση και με παράλληλη εκτέλεση. Οπου είναι εφικτό ο αλγόριθμους υλοποιείται  για εκτέλεση στην \en{GPU} για την επίλυση του. Οι χρονικές καταγραφές συγκρίνονται μεταξύ τους για την εξαγωγή συμπερασμάτων. Ακόμη, γίνεται αξιολόγηση της ευχρηστίας για την υλοποίηση της κάθε παραλλαγής αλλά και προβλημάτων που προέκυψαν.\\[1 cm]

\indent \textbf{Λέξεις Κλειδιά:}
Παράλληλος Προγραμματισμός, Παραλληλοποίηση, \en{OpenMP, accelerators, offloading, vectorization, SIMD, OpenMP4.5, UDRs}

\clearpage
\selectlanguage{english}
\begin{flushleft}

{\large \textbf{Abstract}}\\[0.5 cm]
\end{flushleft}
[Enter abstract here.]\\[1 cm]
\indent \textbf{Keywords:}

\clearpage
\selectlanguage{greek}
\begin{flushleft}
{\large \textbf{Ευχαριστίες}}\\[0.5 cm]
\end{flushleft}
\subparagraph{}
Εκφράζω τις θερμές μου ευχαριστίες στον επιβλέποντα καθηγητή κ. Κωνσταντίνο Μαργαρίτη, για την ουσιαστική του συνεισφορά στην εκπόνηση της παρούσας εργασίας.

\clearpage
\singlespacing
\tableofcontents
\setstretch{1.5}

\selectlanguage{greek}
\renewcommand{\listfigurename}{Κατάλογος Εικόνων (αν υπάρχουν)}
\clearpage
\listoffigures

\selectlanguage{greek}
\renewcommand{\listtablename}{Κατάλογος Πινάκων (αν υπάρχουν)}
\clearpage
\listoftables

\clearpage
\begin{flushleft}
{\large \textbf{Συμβολισμοί (αν υπάρχουν)}}\\[0.5 cm]
\end{flushleft}

\clearpage
\setcounter{page}{1}
\pagenumbering{arabic}

\section{Εισαγωγή}
\subsection{Πρόβλημα – Σημαντικότητα του θέματος}
\subparagraph{}
Εδώ κάνουμε μια γενική περιγραφή του θέματος που διαπραγματεύεται η διπλωματική. Αναφέρουμε τα κύρια χαρακτηριστικά, προσδιορίζουμε σύντομα τις επιμέρους περιοχές και εξηγούμε γιατί έχουν ενδιαφέρον. Η συζήτηση των περιοχών θα πρέπει να προϊδεάζει τον αναγνώστη για το που θα επικεντρωθεί η βιβλιογραφική μελέτη της διπλωματικής, χωρίς ακόμα να αναφερόμαστε συγκεκριμένα στα θέματα της μελέτης.

 \newpage
 \subsection{Σκοπός – Στόχοι}
\subparagraph{}
Ο σκοπός και οι στόχοι της έρευνας θα πρέπει να μας εξηγούν συνοπτικά το σκοπό της μελέτης. Ο σπουδαστής με λίγες προτάσεις πρέπει να καταγράψει σκοπούς και παρατηρήσεις, να παραθέσει το υπόβαθρο της έρευνας του και να οριοθετήσει το πλαίσιό της.

\subsection{Ερωτήματα – Υποθέσεις ( εάν υπάρχουν )}

\subsection{Συνεισφορά}
\subparagraph{}
Εδώ παραθέτουμε αριθμητικά συγκεκριμένες ενέργειες που κάναμε κατά τη βιβλιογραφική μελέτη και που στην ουσία συνιστούν τη συνεισφορά της μελέτης αυτής για τους αναγνώστες.

\subsection{Βασική Ορολογία ( εάν υπάρχει)}

\subsection{Διάρθρωση της μελέτης}
\subparagraph{}
Εδώ περιγράφουμε τα κεφάλαια της διπλωματικής. Συνήθως η ενότητα αυτή έχει την παρακάτω μορφή (δεν θα σας πάρει πάνω από 1 μεγάλη παράγραφο): Εργασίες σχετικές με το αντικείμενο της διπλωματικής παρουσιάζονται στο Κεφάλαιο 2. Το Κεφάλαιο 3 συζητά…. Στο Κεφάλαιο 4 αναπτύσσουμε …κλπ.

\clearpage
\section{Βιβλιογραφική Επισκόπηση – Θεωρητικό Υπόβαθρο}
\subparagraph{}
Απαιτείται μια κριτική αξιολόγηση της βιβλιογραφίας. Αν χρησιμοποιείτε υποθέσεις ή εκ των προτέρων προϋποθέσεις, αυτές θα πρέπει να υποστηριχθούν και να δικαιολογηθούν με βάση τις υπάρχουσες θεωρίες, προηγούμενες μελέτες ή ερευνητικά πορίσματα. Επίσης, περιγράφουμε έννοιες και θέματα χρήσιμα για τη διπλωματική που ενδεχομένως ο αναγνώστης να πρέπει να εξοικειωθεί προτού ξεκινήσουμε τη βιβλιογραφική μελέτη.

\clearpage
\section{Μεθοδολογία}
\subparagraph{}
Αυτό το μέρος της εργασίας ασχολείται με τον πραγματικό ερευνητικό σχεδιασμό που θα ακολουθηθεί. Μία σύντομη ανασκόπηση των εναλλακτικών λύσεων και των αντίστοιχων πλεονεκτημάτων και μειονεκτημάτων τους θα πρέπει να παρουσιαστεί και η επιλογή μιας συγκεκριμένης προσέγγισης θα πρέπει να εξεταστεί καλά και να συνδεθεί με το(α) πρόβλημα(τα) που εξετάζεται(ονται). Οι δευτερεύουσες και οι βασικές πηγές των στοιχείων θα πρέπει να καθοριστούν και η χρήση τους θα πρέπει να εξεταστεί και να δικαιολογηθεί. Σε έρευνες πεδίου θα πρέπει να διατυπωθούν καθαρά  ο τρόπος συλλογής των στοιχείων, η επιλογή του πληθυσμού που θα ερευνηθεί και η δειγματοληπτική διαδικασία που θα ακολουθηθεί. Οι μετρήσεις που χρησιμοποιήθηκαν θα πρέπει επίσης να δικαιολογηθούν και θα πρέπει να γίνει μια περιγραφή του εργαλείου συλλογής στοιχείων.

Οι εικόνες (και τα σχήματα) εμφανίζονται στο κείμενο όπως παρακάτω. Ο Κατάλογος των Εικόνων ενημερώνεται αυτόματα. Η περιγραφή τους πρέπει να εμφανίζεται κάτω από την εικόνα.


\begin{figure}[htbp]
\includegraphics[width=0.15\textwidth]{figure1.png}
\captionsetup{justification=raggedright,
singlelinecheck=false
}
\caption{Η Υδρόγειος σφαίρα}
\label{fig:figure1}
\end{figure}

Οι πίνακες εμφανίζονται όπως παρακάτω. Ο Κατάλογος των Πινάκων ενημερώνεται αυτόματα. Η περιγραφή τους πρέπει να είναι πάνω από τον πίνακα.
\begin{table}[htbp]
\captionsetup{justification=raggedright,
singlelinecheck=false
}
\caption{Ένας πίνακας χωρίς νόημα!}
\begin{otherlanguage}{english}
\def\arraystretch{1.5}
\begin{tabular}{| p{2.0cm} | p{2.0cm}| p{2.0cm} | p{2.0cm} | p{2.0cm} |}
\cline{1-5}
A-D & A & B & C & D \\
\cline{1-5}
1 & A1 & B1 & C1 & D1 \\
\cline{1-5}
2 & A2 & B2 & C2 & D2 \\
\cline{1-5}
3 & A3 & B3 & C3 & D3 \\
\cline{1-5}
\end{tabular}
\end{otherlanguage}
\end{table}


\selectlanguage{greek}

\clearpage
\section{Επίλογος}
\subparagraph{}
Εδώ συνοψίζουμε την παρουσίαση της διπλωματικής.

\subsection{Σύνοψη και συμπεράσματα}
\subparagraph{}

Εδώ συνοψίζουμε τα αποτελέσματα της διπλωματικής και περιγράφουμε τα συμπεράσματα που προέκυψαν, αρνητικά και θετικά. Επιβεβαιώνουμε τη συνεισφορά της διπλωματικής στα προβλήματα που αναφέραμε στην εισαγωγή. Τα συμπεράσματα θα πρέπει να παρουσιάζονται συστηματικά για κάθε αντικειμενικό στόχο ή υπόθεση που έχουμε κάνει.

\subsection{Όρια και περιορισμοί της έρευνας}
\subsection{Μελλοντικές Επεκτάσεις}
\subparagraph{}
Εδώ δίνουμε ιδέες για επέκταση της διπλωματικής. Αναφέρουμε ότι θα ακολουθήσει παρουσίαση μελλοντικών κατευθύνσεων έρευνας ανά θεματική περιοχή. Περιγράφουμε προβλήματα που δεν έχουν λυθεί από τις τεχνικές/μεθοδολογίες που παρουσιάσαμε στο προηγούμενο κεφάλαιο. Τα άλυτα αυτά προβλήματα, αποτελούν στην ουσία προκλήσεις για περαιτέρω έρευνα. Ακόμα καλύτερα, θα ήταν ωραία να προτείνουμε τρόπους επίλυσης των προβλημάτων αυτών έστω και ως γενική ιδέα.

\clearpage
\selectlanguage{greek}

 \selectlanguage{english}
 \bibliography{bibliogr.bib}
 \bibliographystyle{abbrv}
\end{document}