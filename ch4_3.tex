\subsection{Παράδειγμα πολλαπλασιασμού πινάκων}
\subparagraph{}
Ο πολλαπλασιασμός πινάκων αποτελεί έναν από τους πιο κλασικούς αλγόριθμους στον προγραμματισμό και σε αυτό το παράδειγμα μελετάται η παράλληλη υλοποίησή του. Μελετώνται οι χρονικές επιδόσεις πολλαπλασιασμού πικάνων διαφόρων διαστάσεων και συγκρίνονται μεταξύ τους. Η αρχικοποίηση των πινάκων γίνεται με τυχαία νούμερα που οι τιμές τους βρίσκονται στο εύρος [-10, 10].

\selectlanguage{english}
\begin{lstlisting}[language=C++, caption={\el{Αρχικοποίηση τιμών διανύσματος}} , frame=tlrb]{Name}
void fill_random_1d(int *mat, int width, int height) {
        for (int row = 0; row < height; row++) {
                for (int col = 0; col < width; ++col) {
                        mat[col + row*width] = rand() % 20 - 10;
                }
        }
}
\end{lstlisting}
\selectlanguage{greek}

Η μεταγλώττιση του κώδικα έγινε με μεταγλωττιστή \emph{\en{g++-7}} και τις επιλογές:$-Wall -o exec -O0$.

\clearpage
\subsubsection{Σειριακή εκτέλεση}
\subparagraph{}

Στο παράδειγμα, υλοποιείται ένας από τους απλούστερους αλγορίθμους πολλαπλασιασμού πινάκων, που αποτελείται από ένα τριπλό βρόγχο επανάληψης όπως φαίνεται στο παρακάτω τμήμα κώδικα:
\ \\
\selectlanguage{english}
\begin{lstlisting}[language=C++, caption={\el{Υλοποίηση σειριακού αλγορίθμου}} , frame=tlrb]{Name}
int matmul_1d(int *a, int r1, int c1,
                int *b, int r2, int c2,
                int *c, int r3, int c3)
{
	if (c1 != r2) return 1;

    for (int i = 0; i < r1; ++i) {
		for (int j = 0; j < c2; ++j) {
        	int temp = 0;
			for (int k = 0; k < c1; ++k) {
            		temp += a[k + i*c1] * b[j + k*c2];
			}
			c[j + i * c3] = temp;
		}
	}
	return 0;
}

\end{lstlisting}
\selectlanguage{greek}

\begin{table}[htbp]
\centering
\captionsetup{justification=raggedright,
singlelinecheck=false
}
\caption{ \emph{Καταγραφή χρόνων εκτέλεσης παραδειγμάτων}}
\def\arraystretch{1.5}
\begin{tabular}{ | p{0.35\textwidth} | p{0.35\textwidth}| }
 \textbf{Διαστάσεις Πινάκων\cellcolor[HTML]{D0D0D0}} & \textbf{Χρόνος Υπολογισμού (\emph{\en{sec}}) }\cellcolor[HTML]{D0D0D0} \\
\hline
[900 ,   900] * [900  ,  900] & 12.611\\
\hline
[100 , 60000] * [60000,  100] & 12.284\\
\hline
[3000,   100] * [100  , 3000] & 13.173\\
\hline
\end{tabular}
\end{table}

\paragraph{Σχόλιο:}
\subparagraph{}
Υπάρχει πλειθώρα σειριακών αλγορίθμων με καλύτερες επιδόσεις.
Ωστόσο, ο συγκεκριμένος βοηθάει στην εύκολη κατανοήση και τη δημιουργία παράλληλων εκδόσεων, λόγω του διπλού βρόγχου επανάληψης. Οι χρονικές καταγραφές των τριών πινάκων θα χρησιμοποιηθούν ως σημείο σύγκρισης με τις υπόλοιπες υλοποιήσεις.

\clearpage
\subsubsection{Παραλλαγή 1\textsuperscript{η}}
\subparagraph{}
Στη συγκεκριμένη παραλλαγή εφαρμόζεται παραλληλοποίηση με τη χρήση της οδηγίας \textbf{\emph{\en{pragma omp parallel for}}}. 
\selectlanguage{english}
\begin{lstlisting}[language=C++, caption={\el{Υλοποίηση παραλλαγής 1}} , frame=tlrb]{Name}
void matmul_1d(int *a, int r1, int c1,
                int *b, int r2, int c2,
                int *c, int r3, int c3)
{
        if (c1 != r2) return 1;
        
#pragma omp parallel for
        for (int i = 0; i < r1; ++i) {
                for (int j = 0; j < c2; ++j) {
                        int temp = 0;
                        for (int k = 0; k < c1; ++k) {
                                temp += a[k + i*c1] * b[j + k*c2];
                        }

                        c[j + i * c3] = temp;
                }
        }
        return 0;
}
\end{lstlisting}
\selectlanguage{greek}

\begin{table}[htbp]
\centering
\captionsetup{justification=raggedright,
singlelinecheck=false
}
\caption{ \emph{Καταγραφή χρόνων εκτέλεσης παραδειγμάτων}}
\def\arraystretch{1.5}
\begin{tabular}{| p{0.35\textwidth} | p{0.35\textwidth}|}
 \textbf{Διαστάσεις Πινάκων\cellcolor[HTML]{D0D0D0}} & \textbf{Χρόνος Υπολογισμού (\emph{\en{sec}}) }\cellcolor[HTML]{D0D0D0} \\
\hline
[900 900] * [900 900] & 0.821 \\
\hline
[1400 1400] * [1400 1400] & 4.113\\
\hline
[1500 1500] * [1500 1500] & 5.711\\
\hline
[100 60000] * [60000 100] & 1.452\\
\hline
[3000 100] * [100 3000] & 0.861 \\
\hline
\end{tabular}
\end{table}


\paragraph{Σχόλιο:}
\subparagraph{}
Η χρήση της οδηγίας παραλληλοποίησης
\textbf{\emph{\en{pragma omp parallel for}}}
οδηγεί σε έως και 15 φορές ταχύτερο υπολογισμό στο ίδιο μέγεθος προβλήματος.
 
\clearpage
\subsubsection{Παραλλαγή 2\textsuperscript{η}}
\subparagraph{}
Σε αυτή την παραλλαγή χρησιμοποιείται πάλι η οδηγία\textbf{\emph{\en{pragma omp parallel for}}}, όμως εισάγεται και η φράση \textbf{\emph{\en{collapse}}} που χρησιμοποιείται σε περιπτώσεις ένθετων βρόγχων, όπως στο συγκεκριμένο πρόβλημα.
\selectlanguage{english}
\begin{lstlisting}[language=C++, caption={\el{Αρχικοποίηση τιμών διανύσματος}} , frame=tlrb]{Name}
void matmul_1d(int *a, int r1, int c1,
                int *b, int r2, int c2,
                int *c, int r3, int c3)
{
        if (c1 != r2) return 1;

#pragma omp parallel for collapse(2)
        for (int i = 0; i < r1; ++i) {
                for (int j = 0; j < c2; ++j) {
                        int temp = 0;
                        for (int k = 0; k < c1; ++k) {
                                temp += a[k + i*c1] * b[j + k*c2];
                        }

                        c[j + i * c3] = temp;
                }
        }
        return 0;
}
\end{lstlisting}
\selectlanguage{greek}

\begin{table}[htbp]
\centering
\captionsetup{justification=raggedright,
singlelinecheck=false
}
\caption{\emph{Καταγραφή χρόνων εκτέλεσης παραδειγμάτων}}
\def\arraystretch{1.5}
\begin{tabular}{| p{0.35\textwidth} | p{0.35\textwidth}|}
 \textbf{Διαστάσεις Πινάκων\cellcolor[HTML]{D0D0D0}} & \textbf{Χρόνος Υπολογισμού (\emph{\en{sec}}) }\cellcolor[HTML]{D0D0D0} \\
\hline 
[900 900] * [900 900] & 0.795\\
\hline
[1400  1400] * [1400 1400] & 3.879\\
\hline
[1500 1500] * [1500 1500] & 5.301\\
\hline
[100 60000] * [60000 100] &  2.080\\
\hline
[3000 100] * [100 3000] & 0.829\\
\hline
\end{tabular}
\end{table}


\paragraph{Σχόλιο:}
\subparagraph{}
Η χρήση της φράσης \textbf{\emph{\en{collapse}}} οδηγεί σε μικρή βελτίωση της επίδοσης σε σχέση με την παραλλαγή 1.


\clearpage
\subsubsection{Παραλλαγή 3\textsuperscript{η}}
\subparagraph{}
Στη συγκρεκριμένη παραλλαγή εφαρμόζεται \emph{\en{offloading}}, δηλαδή ο αλγόριθμος μεταφέρεται για να εκτελεστεί στη συσκευή της κάρτας γραφικών μέσω της οδηγίας \textbf{\emph{\en{pragma omp target}}}.

\selectlanguage{english}
\begin{lstlisting}[language=C++, caption={\el{Υλοποίηση παραλλαγής 3}} , frame=tlrb]{Name}
void matmul_1d(int *a, int r1, int c1,
                int *b, int r2, int c2,
                int *c, int r3, int c3)
{
        if (c1 != r2) return 1;

        int a_s = r1 * c1, b_s = r2 * c2, c_s = r3 * c3;
        
#pragma omp target teams distribute parallel for simd\
	map(a[:a_s], b[:b_s], c[:c_s])
        for (int i = 0; i < r1; ++i) {
                for (int j = 0; j < c2; ++j) {
                        int temp = 0;
                        for (int k = 0; k < c1; ++k) {
                                temp += a[k + i*c1] * b[j + k*c2];
                        }
                        c[j + i * c3] = temp;
                }
        }
        return 0;
}
\end{lstlisting}
\selectlanguage{greek}

\begin{table}[htbp]
\centering
\captionsetup{justification=raggedright,
singlelinecheck=false
}
\caption{\emph{Καταγραφή χρόνων εκτέλεσης παραδειγμάτων}}
\def\arraystretch{1.5}
\begin{tabular}{| p{0.35\textwidth} | p{0.35\textwidth}|}
 \textbf{Διαστάσεις Πινάκων\cellcolor[HTML]{D0D0D0}} & \textbf{Χρόνος Υπολογισμού (\emph{\en{sec}}) }\cellcolor[HTML]{D0D0D0} \\
\hline 
[900 900] * [900 900] & 2.931\\
\hline
[1400 1400] * [1400 1400] & 8.120\\
\hline
[1500 1500] * [1500 1500] & 9.477\\
\hline
[100 60000] * [60000 100] &  5.975\\
\hline
[3000 100] * [100 3000] & 3.032\\
\hline
\end{tabular}
\end{table}

\clearpage
\subsubsection{Παραλλαγή 4\textsuperscript{η}}
\subparagraph{}
Μοναδική διαφορά με τη προηγούμενη παραλλαγή είναι οτι σε αυτή την περίπτωση χρησιμοποιούνται οι φράσεις \textbf{\emph{\en{from, to}}} για να συγκριθεί ο χρόνος που απαιτείται για τη μεταφορά των δεδομένων ανάμεσα στις δυο συσκευές. Στη συγκεκριμένη παραλλαγή, σε αντίθεση με τη προηγούμενη, γίνονται μόνο οι απαιτούμενες ενέργειες αντιγραφής των δεδομένων.
\selectlanguage{english}
\begin{lstlisting}[language=C++, caption={\el{Αρχικοποίηση τιμών διανύσματος}} , frame=tlrb]{Name}
void matmul_1d(int *a, int r1, int c1,
                int *b, int r2, int c2,
                int *c, int r3, int c3)
{
        if (c1 != r2)  return 1;

        int a_size = r1 * c1, b_s = r2 * c2, c_s = r3 * c3;

#pragma omp target teams distribute parallel for simd\
                        map(to:a[:a_size], b[:b_size])\
                        map(from:c[:c_size])
        for (int i = 0; i < r1; ++i) {
                for (int j = 0; j < c2; ++j) {
                        int temp = 0;
                        for (int k = 0; k < c1; ++k) {
                                temp += a[k + i*c1] * b[j + k*c2];
                        }

                        c[j + i * c3] = temp;
                }
        }
        return 0;
}
\end{lstlisting}
\selectlanguage{greek}

\begin{table}[htbp]
\centering
\captionsetup{justification=raggedright,
singlelinecheck=false
}
\caption{\emph{Καταγραφή χρόνων εκτέλεσης παραδειγμάτων}}
\def\arraystretch{1.5}
\begin{tabular}{| p{0.35\textwidth} | p{0.35\textwidth}|}
 \textbf{Διαστάσεις Πινάκων\cellcolor[HTML]{D0D0D0}} & \textbf{Χρόνος Υπολογισμού (\emph{\en{sec}}) }\cellcolor[HTML]{D0D0D0} \\
\hline 
[900 900] * [900 900] & 2.94  \\
\hline
[1400 1400] * [1400 1400] & 8.19 	\\
\hline
[1500 1500] * [1500 1500] & 9.43\\
\hline
[100 60000] * [60000 100] &  5.89\\
\hline
[3000 100] * [100 3000] & 3.035 \\
\hline
\end{tabular}
\end{table}

\clearpage
\subsubsection{Παραλλαγή 5\textsuperscript{η}}
\subparagraph{}
Η παραλλαγή αυτή μοιάζει με τις δύο προηγούμενες. Η διαφορά είναι οτι τα δεδομένα αποθηκεύονται απευθείας στη μνήμη της κάρτας γραφικών μέσω της οδηγίας \textbf{\emph{\en{is\_device\_ptr}}}. Ετσι, δεν απαιτείται αντιστοίχιση δεδομένων ανάμεσα στις δυο συσκευές με τη χρήση της οδηγίας \emph{\en{map}}.
\selectlanguage{english}
\begin{lstlisting}[language=C++, caption={\el{Αρχικοποίηση τιμών διανύσματος}} , frame=tlrb]{Name}
void matmul_1d(int *a, int r1, int c1,
                int *b, int r2, int c2,
                int *c, int r3, int c3)
{
        if (c1 != r2) return 1;

#pragma omp target teams distribute parallel for simd\
        is_device_ptr(a) is_device_ptr(b)\
        is_device_ptr(c)
        for (int i = 0; i < r1; ++i) {
                for (int j = 0; j < c2; ++j) {
                        int temp = 0;
                        for (int k = 0; k < c1; ++k) {
                                temp += a[k + i*c1] * b[j + k*c2];
                        }
                        c[j + i * c3] = temp;
                }
        }
}
\end{lstlisting}
\selectlanguage{greek}

\begin{table}[htbp]
\centering
\captionsetup{justification=raggedright,
singlelinecheck=false
}
\caption{\emph{Καταγραφή χρόνων εκτέλεσης παραδειγμάτων}}
\def\arraystretch{1.5}
\begin{tabular}{| p{0.35\textwidth} | p{0.35\textwidth}|}
 \textbf{Διαστάσεις Πινάκων\cellcolor[HTML]{D0D0D0}} & \textbf{Χρόνος Υπολογισμού (\emph{\en{sec}}) }\cellcolor[HTML]{D0D0D0} \\
 \hline
 [900 900] * [900 900] & 2.045\\
 \hline
 [1400 1400] * [1400 1400] & 7.33\\
\hline
[1500 1500] * [1500 1500] & 8.47\\
\hline
[100 60000] * [60000 100] & 4.99\\
\hline
[3000 100] * [100 3000] & \en{Could not allocate memory} \\
\hline
\end{tabular}
\end{table}

\paragraph{Σχόλιο:}
\subparagraph{}

Ο πολλαπλασιασμός πινάκων με μέγεθος Α[3000 100] και Β[100 3000] έχει οδηγεί σε αδυναμία εκτέλεσης. Αντίθετα, ο υπολογισμός Α[100 60000] και Β[60000 100] γίνεται κανονικά.

\newpage
\subsubsection{Συγκρίσεις με γραφική απεικόνιση}
\subparagraph{}


%%%<
%%%>
\begin{comment}
%:Title: Bar plot
%:Tags: 2D;Bar plots;Manual
%:Author: Christian Feuersänger
%:Slug: bar-plot

%Bar plots place horizontal or vertical bars at coordinates. %Multiple bar plots
%in one axis can be stacked on top of each other or aligned next to each other.

%The code is from the PGFPlots 1.10 manual: "4.5.4 Bar Plots".
\end{comment}
\begin{tikzpicture}
\begin{axis}[
    ybar,
    enlargelimits=0.15,
    legend style={at={(0.5,-0.25)},
      anchor=north,legend columns=-1,
      },
    ylabel={Χρόνος υπολογισμού \emph{\en{(sec)}}},
    x tick label style={rotate=30,anchor=east},
    symbolic x coords={
   					 [900 900 900],
					[1400 1400 1400],
					[1500 1500 1500],
                    [100 600000 100],
                    [3000 100 3000]
     },
    xtick=data,
    ]
    
   
\addplot coordinates { ([900 900 900], 0.821)
					   ([1400 1400 1400], 4.133)
					   ([1500 1500 1500], 5.711)
					   ([100 600000 100], 1.452)
					   ([3000 100 3000], 0.861)

					   };
\addplot coordinates {([900 900 900], 0.795) 
                      ([1400 1400 1400], 3.879)
                      ([1500 1500 1500], 5.301)
                      ([100 600000 100], 2.080)
				      ([3000 100 3000], 0.829)
                      };
\addplot coordinates {([900 900 900], 2.931) 
                      ([1400 1400 1400], 8.120)
                      ([1500 1500 1500], 9.477)
                      ([100 600000 100], 5.975)
					   ([3000 100 3000], 3.032)


                      };
\addplot coordinates {([900 900 900], 2.94)
                      ([1400 1400 1400], 8.19)
                      ([1500 1500 1500], 9.43)
                      ([100 600000 100], 5.89)
					   ([3000 100 3000], 3.035)

                      };
\addplot coordinates {([900 900 900], 2.045) 
                      ([1400 1400 1400], 7.33)
                      ([1500 1500 1500], 8.47)
                      ([100 600000 100], 4.99)

                      };


%\addplot coordinates {(tool8,7) (tool9,9) (tool10,4)};
%\addplot coordinates {(tool8,4) (tool9,4) (tool10,4)};
%\addplot coordinates {(tool8,1) (tool9,1) (tool10,1)};
%\legend{used,understood,not understood}
%\legend{used,understood,not understood}
\legend{Παραλλαγή 1, Παραλλαγή 2, Παραλλαγή 3, Παραλλαγή 4, Παραλλαγή 5}

\end{axis}
\end{tikzpicture}

Από το παραπάνω διάγραμμα εξάγονται τα παρακάτω συμπεράσματα:
\begin{itemize}
\item{H απόδοση του αλγορίθμου με χρήση της οδηγίας \emph{\en{pragma omp parallel}} οδηγεί σε ταχύτερη επίλυση των υπολογισμών, με μεγάλη διαφορά σε σύγκριση με οποιαδήποτε επίλυση στη συσκευή της κάρτας γραφικών.}
\item{Η φράση \emph{\en{collapse}} δε συμβάλλει ιδιαίτερα στην βελτίωση της απόδοσης του αλγοριθμου.}
\end{itemize}
