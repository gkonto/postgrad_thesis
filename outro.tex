\setstretch{1.5}
\section{Επίλογος}
\subsection{Σύνοψη και συμπεράσματα}
Σε αυτή την εργασία μελετήθηκε το \en{OpenMP}, με την περιγραφη των βασικότερων χαρακτηριστικών, τη συγκέντρωση, την υλοποίηση και την καταγραφή παραδειγμάτων εκτελεσμένων σειριακά αλλά και παράλληλα. Είναι σημαντικό να σημειωθεί ότι η πληθώρα των παρατηρήσεων και συμπερασμάτων καταγράφονται σε κάθε επιμέρους παραλλαγή του προβλήματος. Ωστόσο κάποιες από τις γενικές παρατηρήσεις που προκύπτουν είναι οι παρακάτω:
\begin{itemize}
\item{Η εισαγωγή των εργασιών (\en{Tasks}) αποτέλεσε ένα από τα σημαντικότερα νέα χαρακτηριστικά της διεπαφής του \en{OpenMP}, καθώς επιτρέπει την παραλληλοποίηση προβλημάτων για τα οποία δεν ήταν εφικτή ή ήταν δύσκολη ή ήταν εφικτή με μειονεκτήματα η υλοποίηση τους χωρίς αυτό το χαρακτηριστικό.}

\item{Η παραλληλοποίηση πολλές φορές αποδεικνύεται λύση εύκολα υλοποιήσιμη -κυρίως όταν το πρόβλημα εμπεριέχει βρόγχους επανάληψης- και ικανή να μειώσει δραματικά το χρόνο εκτέλεσης σε σύγκριση με τη σειριακή.}

\item{Η μεταφορά του κώδικα στο περιβάλλον της κάρτας γραφικών και η εκτέλεση του αλγορίθμου εκεί επιδέχεται βελτίωσης. Βρίσκεται σε πρώιμο στάδιο καθώς οι χρόνοι εκτέλεσης δεν είναι οι αναμενόμενοι σε σύγκριση με άλλες διεπαφές για εκτέλεση με \en{Offloading}.}

\item{Βασικό μειονέκτημα στο \en{offloading} αποτελεί η οδηγία \en{map} και ο χρόνος που απαιτείται για την αντιστοίχιση των χρόνων εκτέλεσης στο περιβάλλον της κάρτας γραφικών.}
\end{itemize}

\clearpage
\subsection{Όρια και περιορισμοί της έρευνας}
Ο παράλληλος προγραμματισμός αποτελεί έναν τομέα του ευρύτερου προγραμματισμού που αναπτύσσεται εδώ και χρόνια. Η μελέτη και η κατανόησή του είναι σχεδόν αδύνατο να περιοριστεί στα πλαίσια μιας διπλωματικής εργασίας μεταπτυχιακού επιπέδου, ακόμη και αν η αναφορά γίνεται αποκλειστικά στη διεπαφή του \en{OpenMP}. Για το λόγο αυτό, η εργασία περικλείεται από ορισμένα όρια, τα οποία αναφέρονται σε αυτή την παράγραφο.
\begin{itemize}

\item{Η εργασία αποτελείται από ένα σύνολο παραδειγμάτων, για τη μεταγλώττισή των οποίων χρησιμοποιείται αποκλειστικά ο μεταγλωττιστής \en{GCC/G++}. Παρόλα αυτά, υπάρχουν αρκετές έρευνες που αναφέρουν ότι μια από τις βασικές αιτίες χαμηλής επίδοσης των προβλημάτων που χρησιμοποιούν \en{offloading}, είναι ο μεταγλωττιστής.\cite{offloading_compilers}.}

\item{Σχετικά με την υλοποίηση των προβλημάτων, υπάρχουν πολλές υποψήφιες παραλλαγές. Η εργασία περιορίζεται στις σημαντικότερες από αυτές.}

\item{Στην εργασία δεν χρησιμοποιήθηκε ευρέως η οδηγία \en{is\_device\_ptr}, καθώς για την χρήση της, απαιτείται δέσμευση μνήμης έξω από την παράλληλη περιοχή, στο κοινό τμήμα όλων των παραλλαγών. Κατά συνέπεια, κάθε πρόβλημα θα έπρεπε αποτελείται απο δύο \en{main} ρουτίνες και στη μεταγλώττιση να χρησιμοποιούνται εναλλάξ, κάτι που θα οδηγούσε στην μεγάλη αύξηση του μεγέθους της εργασίας. Παρόλα αυτά, οι ιδιότητες της οδηγίας \en{is\_device\_ptr} αναφέρονται στο παράδειγμα \en{SAXPY}.}

\item{Στις περιπτώσεις που τα αποτελέσματα είναι εμφανώς ίδια με προηγούμενες παραλλαγές, δε δημιουργήθηκαν συγκριτικά διαγράμματα.}

\item{Οι καταγεγραμμένοι χρόνοι της εργασίας αποτελούν τους μέσους όρους πολλαπλών εκτελέσεων του ίδιου προβλήματος, για να αποφευχθούν καταγραφές με σφάλμα.}
\end{itemize}

%\begin{itemize}
%\setlength\itemsep{-0.8em}
%end{itemize}

\clearpage
\subsection{Μελλοντικές Επεκτάσεις}
\ \\
Η εργασία μπορεί να εξελιχθεί μελλοντικά στους παρακάτων τομείς:
\begin{itemize}
\item{Εμπλουτισμός οδηγιών που δεν έχουν συμπεριληφθεί στην εργασία.}
\item{Επίλυση προβλημάτων με διαφορετικούς μεταγλωττιστές και σύγκριση μεταξύ τους.}
\item{Εισαγωγή χαρακτηριστικών νεότερων εκδόσεων (OpenMP 5.0)}
\item{Εμπλουτισμός των ήδη υπαρχόντων παραδειγμάτων με επιπλέον παραλλαγές.}
\item{Εξαγωγή νέων επιπλέον παρατηρήσεων.}
\item{Δημιουργία και καταγραφή νέων παραδειγμάτων και αλγόριθμων.}
\item{Αύξηση του μεγέθους των προβλημάτων με βέλτιστες επιδόσεις σε σχέση με τη σειριακή τους υλοποίηση}
\end{itemize}