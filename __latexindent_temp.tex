\subsection{Πρόσθεση διανυσμάτων αριθμών μικρής ακρίβειας - \emph{\en{SAXPY}}}
\subparagraph{}
Μια από τις λειτουργίες που κατέχει θεμελιώδη θέση σε εφαρμογές της γραμμικής άλγεβρας, αποτελεί η πρόσθεση διανυσμάτων
δεκαδικών αριθμών μικρής ακρίβειας (\emph{\en{floats}}), γνωστή ως \textbf{\en{SAXPY}}.

Στο παράδειγμα \emph{\en{SAXPY}}, ο λόγος του μεγέθους υπολογισμών προς το μέγεθος των δεδομένων που τελούν υπό επεξεργασία
είναι μικρός. Ως εκτότου, αποτελεί πρόβλημα περιορισμένης επεκτασιμότητας. Παρόλα αυτά, πρόκειται για ένα χρήσιμο
παράδειγμα που ανήκει στην κατηγορία προβλημάτων παραλληλοποίησης τύπου \emph{\en{map}} και των εννοιών
\emph{\en{uniform}} και \emph{\en{varying parameters}}\cite{patters}.
\\
\subsubsection{Περιγραφή προβλήματος}
Η λειτουργία \en{SAXPY} δέχεται ως δεδομένα δυο διάνυσματα δεκαδικών αριθμών. Το πρώτο διάνυσμα πολλαπλασιάζεται με μια
σταθερά \emph{\en{a}} και το αποτέλεσμα προστίθεται στο δεύτερο διάνυσμα \emph{\en{y}}. Τα διανύσματα \emph{\en{x,y}}
πρέπει να έχουν το ίδιο μέγεθος.

Ο υπολογισμός αυτός εμφανίζεται συχνά στη γραμμική άλγεβρα, όπως για παράδειγμα στη διαγραφή σειρών για την απαλοιφή
\textbf{\en{Gauss}}. Το όνομα \emph{\en{SAXPY}} δόθηκε από την βιβλιοθήκη \en{\textbf{BLAS} (\emph{"Basic Linear Algrbra
Subprograms"})} για δεκαδικούς αρθμούς μικρής ακρίβειας (\emph{\en{floats}}). Ο αντίστοιχος αλγόριθμος διπλής ακρίβειας
ονομάζεται \emph{\en{DAXPY}}, ενώ για μιγαδικούς αριθμούς ονομάζεται \emph{\en{CAXPY}}.
Η μαθηματική διατύπωση του \emph{\en{SAXPY}} είναι:
                              $$\en{\textbf{y} = a*\textbf{x} + \textbf{y}}$$ όπου το διάνυσμα \emph{\en{x}}
χρησιμοποιείται ως είσοδος, το \en{\emph{y}} ως είσοδος και έξοδος. Δηλαδή το αρχικό διάνυσμα \emph{\en{y}}
τροποποιείται. Εναλλακτικά, η λειτουργία \emph{\en{SAXPY}} μπορεί να περιγραφεί ως συνάρτηση που δρα σε μεμονωμένα
στοιχεία, οπως φαίνεται παρακάτω: 
\begin{equation*}\label{eq:pareto mle2}
\begin{aligned}
f(t, p, q) = tp + q\\
\forall _i : y_i \leftarrow f(a, x_i, y_i)
\end{aligned}
\end{equation*}
\clearpage
Οι συναρτήσεις τύπου \emph{\en{f}} δεχονται ως ορίσματα, δύο είδη παραμέτρων. Τις παραμέτρους όπως την \en{\emph{a}} που
παραμένουν σταθερές και ονομάζονται \emph{\en{uniform}}, οι παράμετροι που είναι μεταβλητές σε κάθε κλίση της
\emph{\en{f}} ονομάζονται \emph{\en{varying}}. To μοτίβο \emph{\en{map}} καλεί τη συνάρτηση \emph{\en{f}} τόσες φορές
όσες και ο αριθμός των στοιχείων του διανύσματος.\cite{patters}.

\subsection{Περιγραγή κεντρικού τμήματος προβλήματος \en{SAXPY}}
\selectlanguage{english}
\begin{spacing}{1.0}
\begin{lstlisting}[language=C++, caption={\el{Κεντρικός κώδικας προβλήματος \en{SAXPY}}} , frame=tlrb]{Name}
        int main(int argc, char **argv) {
                Opts o;
                parseArgs(argc, argv, o);
         
                Containers cond(o.size);
         
                cond.setRandomValues()
                float c = float(rand()) / float(RAND_MAX);
         
                auto start = omp_get_wtime();
                saxpy(cond.m_size, cond.m_c, cond.m_a, cond.m_b);
                auto end = omp_get_wtime();
         
                verify(cond.m_size, c, cond.m_a, cond.m_b, cond.m_verification);
         
                std::cout << "Execution Time : " << std::fixed
                     << end - start << std::setprecision(5);
                std::cout << " sec " << std::endl;
                return 0;
            }
          
\end{lstlisting}
\end{spacing}
\selectlanguage{greek}

\subsection{Σειριακή εκτέλεση}
\subparagraph{}


\subsubsection{Περιγραφή προβλήματος}
\subparagraph{}



\selectlanguage{english}
\begin{lstlisting}[language=C++, caption={\el{Αρχικοποίηση τιμών διανύσματος}} , frame=tlrb]{Name}
void fill_array(int *arr, size_t size) {
    for (size_t k = 0; k < size; ++k) {
            arr[k] = static_cast<int>(k);
    }
}
\end{lstlisting}
\selectlanguage{greek}

\clearpage
\subsubsection{Σειριακή εκτέλεση}
\label{sec:ch42_serial}
\subparagraph{}
Στο σειριακό υπολογισμό, το πρόγραμμα εκτελείται από ένα μοναδικό νήμα, χωρίς βελτιστοποίηση παραλληλισμού. Στη συγκεκριμένη περίπτωση, καλείται μια ρουτίνα που δέχεται ως όρισμα ένα μοναδιαίο πίνακα με ακέραιους αριθμούς και έναν αριθμό που υποδηλώνει το μέγεθος αυτού του πίνακα, όπως φαίνεται παρακάτω:

Οι χρόνοι εκτέλεσης που καταγράφηκαν εμφανίζονται στον παρακάτω πίνακα:

\begin{table}[htbp]
\centering
\captionsetup{justification=raggedright,
singlelinecheck=false
}
\caption{ \emph{Καταγραφή χρόνων εκτέλεσης παραδειγμάτων}}
\def\arraystretch{1.5}
\begin{tabular}{| p{0.25\textwidth} | p{0.25\textwidth}|}
 \textbf{Αριθμός στοιχείων πίνακα\cellcolor[HTML]{D0D0D0}} & \textbf{Χρόνος εκτέλεσης (\emph{\en{sec}}) }\cellcolor[HTML]{D0D0D0} \\
\hline
 100000 & 0.002  \\
\hline
1000000 & 0.0097 \\
\hline
10000000 & 0.098  \\
\hline
100000000 &  0.980\\
\hline
200000000 & 1.978 \\
\hline
300000000 & 2.968 \\
\hline
400000000 & 4.935 \\
\hline
\end{tabular}
\end{table}

\subparagraph{Σχόλιο:}\ \\
\emph{Τα αποτελέσματα της σειριακής εκτέλεσης θα χρησιμοποιηθούν ως σημείο αναφοράς για τη σύγκριση με τις υπόλοιπες παραλλαγές παράλληλης εκτέλεσης του προβλήματος.}
