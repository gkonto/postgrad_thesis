\subsection{\en{Tasking}}
\subparagraph{}
Η έννοια των διεργασιών(\emph{\en{Tasking}}) εισήχθει στο \emph{\en{OpenMP}} το 2008, με την έκδοση
3.0\cite{parallel_dist}. Οι διεργασίες παρέχουν τη δυνατότητα παραλληλοποίησης αλγορίθμων με ακανόνιστη και εξαρτώμενη
από το χρόνο ροή εκτέλεσης μορφή. Ένας μηχανισμός ουράς διαχειρίζεται δυναμικά την εκχώρηση νημάτων στη διεργασία που
πρέπει να εκτελεστεί. Τα νήματα παραλαμβάνουν διεργασίες από την ουρά εως ότου αυτή αδειάσει. Η διεργασία είναι ένα
τμήμα κώδικα της παράλληλης περιοχής που εκτελειται ταυτόχρονα με μια άλλη διεργασία της ίδιας περιοχής. 

Οι διεργασίες αποτελούν τμήμα της παράλληλης περιοχής κώδικα. Χωρίς ειδική μέριμνα, η ίδια διεργασία μπορεί να
εκτελεστεί από διαφορετικά νήματα. Αυτό αποτελεί πρόβλημα, καθώς οι οδηγίες διαμοιρασμού μνήμης καθορίζουν με σαφήνεια
τον τρόπο που οι εργασίες διανέμονται στα νήματα. Για να εγγυηθεί το \emph{\en{OpenMP}} ότι κάθε διεργασία εκτελείται
μόνο μια φορά, η κατασκευή τους θα πρέπει να ενσωματωθεί σε μια οδηγία \emph{\en{single}} ή \emph{\en{master}}.

		\selectlanguage{english}
		\begin{lstlisting}[tabsize=4, basicstyle=\small, language=C++, caption={\el{Παράδειγμα κώδικα με διεργασίες}}, frame=tb]
#include <omp.h>

int main(void) {	
	#pragma omp parallel          // Beginning of parallel region
	{
		#pragma omp single
		{
			#pragma omp task
				func_task_1();    // First task creation
			#pragma omp task
				func_task_2();    // Second task creation
		}// end of single region  // Implicit barrier
	} //end of parallel region    // Implicit barrier
}
\end{lstlisting}
\selectlanguage{greek}

\subparagraph{}
Στο προηγούμενο παράδειγμα έστω ότι υπάρχουν δύο νήματα. Το ένα νήμα συναντά την οδηγία \emph{\en{single}} και αρχίζει
να εκτελεί το τμήμα κώδικα που βρίσκεται μέσα σε αυτή. Δύο διεργασίες δημιουργούνται και εισάγονται σχετική στοίβα χωρίς
να έχει ξεκινήσει όμως η εκτέλεσή τους. Ταυτόχρονα, το δεύτερο νήμα συναντά την υποκείμενη οδηγία \emph{\en{barrier}}
στο τέλος της οδηγίας \emph{\en{single}} και περιμένει εκεί εως ότου εισαχθεί μια διεργασία στη στοίβα διεργασιών προς
εκτέλεση. Τα νήματα αυτά είναι άμεσα διαθέσιμα για την εκτέλεση των διεργασιών. Το νήμα που δημιουργεί τις διεργασίες
καταλήγει επίσης στο υποκείμενο φράγμα όταν ολοκληρωθεί η διαδικασία παραγωγής διεργασιών και είναι και αυτό διαθέσιμο
για εκτέλεση των διεργασιών που απομένουν. Η σειρά εκτέλεσης των διεργασιών δεν είναι προκαθορισμένη.
\subsubsection{Η οδηγία \emph{\en{task}}}
\subparagraph{}
	Βασική οδηγία της έννοιας των διεργασιών είναι η \textbf{\en{task}} που ορίζει μια ρητή διεργασία. Το περιβάλλον
	δεδομένων της διεργασίας δημιουργείται σύμφωνα με τις φράσεις χαρακτηριστικών κοινής χρήσης δεδομένων κατά την
	κατασκευή των διεργασίων και τυχόν προεπιλογές που ισχύουν για τις φράσεις αυτές\cite{thenextstep20}.
	
\selectlanguage{english}
\begin{spacing}{1.1}
\begin{lstlisting}[language=C++, caption={\el{Σύνταξη οδηγίας} \emph{\en{task}}} , frame=tlrb]{Name}
#pragma omp task [clause[ [, ]clause] ...] 
	structured-block 
\end{lstlisting}
\begin{lstlisting}[language=C++, caption={\el{Φράσεις οδηγίας} \emph{\en{task}}} , frame=tlrb]{Name}
if([ task :] scalar-expression) 
final(scalar-expression) 
untied 
default(shared | none) 
mergeable 
private(list) 
firstprivate(list) 
shared(list) 
in_reduction(reduction-identifier : list) 
depend([depend-modifier,] dependence-type : locator-list) 
priority(priority-value) 
allocate([allocator :] list) 
affinity([aff-modifier :] locator-list) 
detach(event-handle)
\end{lstlisting}
\end{spacing}
\selectlanguage{greek}

\paragraph{Φράση \en{if}}
\subparagraph{}
Η έκφραση που δεχεται η συνθήκη \textbf{\en{if}} ως όρισμα σε μια διεργασία, αξιολογείται ως ψευδής ή αληθής. Αν η έκφραση
αξιολογείται ψευδής, απαγορεύεται η αναβολή εκτέλεσής της από το μεταγλωττιστή. Έτσι, δεν γίνονται όλοι οι
υπολογισμοί που απαιτούνται από τον μεταγλωττιστή για την κατασκευή της διεργασίας αυτής και εισαγωγής της στη στοίβα
διεργασιών προς εκτέλεση, όπως αν δεν υπήρχε η φράση \emph{\en{if}}. Με τον τρόπο αυτό, μπορούν να αποφευχθούν ορισμένοι
υπολογισμοί που πιθανόν να οδηγήσουν σε μείωση της απόδοσης. Ανεξαρτήτου αποτελέσματος της έκφρασης ωστόσο,
δημιουργείται πάντα ένα νέο περιβάλλον δεδομένων για τη διεργασία.

Ως εκ τούτου η φράση \emph{\en{if}} αποτελεί λύση σε αλγόριθμους όπως οι αναδρομικοί, όπου το υπολογιστικό κόστος
μειώνεται καθώς αυξάνεται το βάθος, αλλά το όφελος από τη δημιουργία μιας νέας διεργασίας μειώνεται λόγω του γενικού
κόστους κατασκευής της. Παρόλα αυτά, το κόστος της ρύθμισης περιβάλλοντος δεδομένων μπορεί να είναι το κυρίαρχο για τη
δημιουργία διεργασίας, διότι για λόγους ασφαλείας οι μεταβλητές που αναφέρονται σε μια οδηγία κατασκευής διεργασιών
είναι στις περισσότερες περιπτώσεις από προεπιλογή \emph{\en{firstprivate}}\cite{parallel_dist}. 

\paragraph{Φράση \en{final}}
\subparagraph{}
Η φράση \emph{\en{final}} χρησιμοποιείται για να ελέγχεται με ευκρίνεια η ευαισθησία των διεργασιών. Οταν
χρησιμοποιείται μέσα στην οδηγία \emph{\en{task}}, η έκφραση αξιολογείται κατά τη διάρκεια δημιουργίας της διεργασίας.
Αν είναι αληθής, η διεργασία θεωρείται τελική. Όλες οι διεργασίες παιδιά που δημιουργούνται μέσα σε αυτή, αγνοούνται και
εκτελούνται στο πλαίσιο της.

Υπάρχουν δύο διαφορές ανάμεσα στη φράση \emph{\en{if}} και στη \emph{\en{final}}. Οι διαφορές αυτές αφορούν:

\begin{itemize}
\item{τις διεργασίες που επηρεάζει κάθε φράση}
\item{τον τρόπο με τον οποίο  διαχειρίζεται ο μεταγλωττιστής της κατασκευασμένες διεργασίες}\cite{tasking1}.
\end{itemize}
\clearpage

\begin{table}[htbp]
\captionsetup{justification=raggedright, singlelinecheck=false} \caption{Διαφορές ανάμεσα στις φράσεις \emph{\en{if}}
και \emph{\en{final}} οταν εισάγονται σε κατασκευή διεργασίας.} \def\arraystretch{1.5}
\begin{tabular}{| p{0.5\textwidth} | p{0.5\textwidth}|}
\textbf{\en{\emph{if} clause}} \cellcolor[HTML]{D0D0D0} & \textbf{\en{\emph{final} clause}} \cellcolor[HTML]{D0D0D0} \\
\hline
Επηρεάζει την διεργασία που κατασκευάζεται από τη συγκεκριμένη οδηγιά κατασκευής διεργασιών & Επηρεάζει όλες τις
διεργασίες "απογόνους" δηλαδή όλες τις διεργασίες που πρόκειται να δημιουργηθούν μέσα από αυτή που περιείχε τη φράση
\emph{\en{final}}\\
\hline
Η οδηγία κατασκευής διεργασίας αγνοείται εν μέρει. Η διεργασία και το περιβάλλον μεταβλητών δημιουργείται ακόμα και αν η
έκφραση αξιολογηθεί ως \emph{\en{false}} & Αγνοείται εντελώς η κατασκευή διεργασιών δηλαδή δε θα δημιουργηθεί νέα
διεργασία, αλλά ούτε και νέο περιβάλλον δεδομένων.	\\
\hline
\end{tabular}
\end{table}

\paragraph{Φράση \en{mergeable}}
\subparagraph{}

Μια \emph{"συγχωνευμένη"} διεργασία είναι η διεργασία της οποίας το περιβάλλον δεδομένων είναι ίδιο με αυτό της
διεργασίας που τη δημιούργησε. Οταν εισάγεται η φράση \emph{\en{mergeable}} σε μια οδηγία \emph{\en{task}} τότε η
υλοποίηση της δημιουργεί μια συγχωνευμένη διεργασία.
\paragraph{Φράση \en{depend}}
\subparagraph{}

Η φράση \textbf{\en{depend}} επιβάλλει πρόσθετους περιορισμούς στη δημιουργία διεργασιών. Αυτοί οι περιορισμοί
δημιουργούν εξαρτήσεις μόνο μεταξύ συγγενικών διεργασιών.

\selectlanguage{english}
\begin{spacing}{1.1}
\begin{lstlisting}[language=C++, caption={\el{Φράση} \emph{\en{depend}}} , frame=tlrb]{Name}
depend([depend-modifier, ]dependency-type : locator-list)

dependence-type:
	in
	out
	inout
	mutexinoutset
	depobj	
	
depend-modifier:
	iterator(iterators-definition)
	depend([source|depend(sink: vec)])
\end{lstlisting}
\end{spacing}
\selectlanguage{greek}
\clearpage
Οι εξαρτήσεις των διεργασιών καθορίζονται από τον τύπο που ορίζεται σε μια φράση εξάρτησης και την λίστα των
περιεχόμενων αντικειμένων που ακολουθεί. Ο τύπος εξάρτησης μπορεί να είναι ένας από τους ακόλουθους\cite{oracle1}.

Για τον τύπο \textbf{\en{in(x)}}, η προκύπτουσα διεργασία (\en{y}) θα είναι εξαρτημένη όλων των προηγούμενων συγγενικών
διεργασιών που εχουν τύπο εξάρτησης \emph{\en{out}} ή \emph{\en{inout}} και το \textbf{\en{x}} αναφέρεται στην λίστα. Η
\en{y} θα ξεκινήσει μετά την ολοκλήρωση των εξαρτήσεών της.

Για τους τύπους εξάρτησης \emph{\textbf{\en{out(x)}}} και \emph{\textbf{\en{inout}}}, η προκύπτουσα διεργασία  θα
εκτελεστεί και ολοκληρωθεί πρίν από την εκτέλεση των διεργασίών τύπου \emph{\en{in} ή \en{inout}} και περιέχουν την
μεταβλητή \textbf{\en{x}} στη λίστα. \ \\

\textbf{Παράδειγμα}
\begin{itemize}
\item{Η φράση \emph{\en{depend(in:x)}} θα δημιουργήσει μια εξάρτηση με όλες της διεργασίας που δημιουργήθηκαν με τη
φράση \emph{\en{depend(out:x)} ή \en{depend(inout: x)}}}.

\item{Η φράση \emph{\en{depend(out:x)} ή \en{depend(inoout:x)}} θα δημιουργήσει μια διεργασία εξαρτημένη με όλες τις
προηγούμενες διεργασίες που αναφέρουν τη μεταβλητή \emph{\en{x}} στη φράση εξάρτησης.}
\end{itemize}

Από τα παραπάνω, προκύπτει η εξής εξάρτηση διεργασιών:
\selectlanguage{english}
\begin{spacing}{1.1}
\begin{lstlisting}[language=C++, caption={\el{Παράδειγμα εξάρτησης διεργασιών}} , frame=tlrb]{Name}
#pragma omp task depend(out:x) // task1
...
#pragma omp task depend(in:x) depend(out:y) // task2
...
#pragma omp task depend(inout:x) // task3
...
#pragma omp task depend(in:x, y) // task4
...


task1(out:x) -> task2(in:x, out:y) -> task4(in:x, y)
		|
		-> task3(inout:x)
\end{lstlisting}
\end{spacing}
\selectlanguage{greek}
\clearpage
\paragraph{Φράση \en{untied}}
\subparagraph{}
Κατά τη συνέχιση μιας διεργασίας που έχει τεθεί σε αναστολή, μια δεσμευμένη διεργασία θα πρέπει να εκτελεστεί ξανά από
το ίδιο νήμα. Με τη φράση \emph{\en{untied}}, δεν υπάρχει τέτοιος περιορισμός και η διεργασία συνεχίζεται από
οποιοδήποτε νήμα.

\subsubsection{Συγχρονισμός διεργασιών}
\subparagraph{}
Οι διεργασίες δημιουργούνται όταν το πρόγραμμα συναντήσει την οδηγία \textbf{\en{pragma omp task}}. Η στιγμή εκτέλεσης
τους δεν είναι προκαθορισμένη. Το \emph{\en{OpenMP}} εγγυάται οτι θα έχουν ολοκληρωθεί όταν ολοκληρωθεί η εκτέλεση του
προγράμματος ή όταν προκύψει κάποια οδηγία συγχρονισμού νημάτων.

Στην περίπτωση της δημιουργίας διεργασιών μέσω οδηγίας \emph{\en{\textbf{single}}} στο τέλος της οδηγίας υπάρχει
υποκείμενος φραγμός εκτέλεσης, σε αντίθεση με την οδηγία \emph{\en{\textbf{master}}}.

\begin{table}[h]
\captionsetup{justification=raggedright,
singlelinecheck=false
}
\caption{Οδηγιες συγχρονισμού διεργασιών.}
\def\arraystretch{0.5}
\begin{tabular}{| p{0.5\textwidth} | p{0.5\textwidth}|}
\hline
\textbf{Οδηγία} \cellcolor[HTML]{D0D0D0} & \textbf{Περιγραφή} \cellcolor[HTML]{D0D0D0} \\
\hline
\textbf{\en{{\#}pragma omp barrier}} & Μπορεί να υπονοείται μέσω μιας οδηγίας ή να δηλωθεί ρητά από το χρήστη. \\
\hline
\textbf{\en{{\#}pragma omp taskwait}} & Αναμένει μέχρι να ολοκληρωθούν όλες οι διεργασίες-παιδιά της συγκεκριμένης
διεργασίας.\\
\hline
\textbf{\en{{\#}pragma omp taskgroup}} & Αναμένει μέχρι να ολοκληρωθούν όλες οι διεργασίες παιδιά της συγκεκριμένης
διεργασίας αλλά και οι απόγονοι τους.\\
\hline
\end{tabular}
\end{table}

\selectlanguage{english}
\begin{spacing}{1.0}
\begin{lstlisting}[language=C++, caption={\el{Παράδειγμα} taskwait} , frame=tlrb]{Name}
#pragma omp parallel {
	#pragma omp single
	{
		#pragma omp task
		{ ... } // Task #1

		#pragma omp task
		{ ... } // Task #2

		#pragma omp taskwait
			last_to_be_executed();
	} // End of single region
} // End of parallel region
\end{lstlisting}
\end{spacing}
\selectlanguage{greek}