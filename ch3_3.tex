\subsection{\en{Tasking}}
\subparagraph{}
Η έννοια των διεργασιών (\emph{\en{Tasking}}) εισήχθει στο \en{OpenMP} το 2008 με την έκδοση 3.0\cite{parallel_dist}.
Οι διεργασίες παρέχουν τη δυνατότητα, οι αλγόριθμοι με ακανόνιστη και εξαρτώμενη από το χρόνο ροή εκτέλεσης να μπορούν να παραλληλιστούν. Ένας μηχανισμός ουράς διαχειρίζεται δυναμικά την εκχώρηση νημάτων στη διεργασία που πρέπει να εκτελεστεί. Τα νήματα παραλαμβάνουν εργασίες από την ουρά εως ότου αυτή αδειάσει.
Διεργασία (\emph{\en{task}}) είναι ενα μπλοκ κώδικα σε μια παράλληλη περιοχή που εκτελειται ταυτόχρονα με μια αλλη διεργασία στην ίδια περιοχή. 

Οι διεργασίες είναι τμήμα της παράλληλης περιοχής.
Χωρίς ειδική μέριμνα, η ίδια διεργασία μπορεί να εκτελεστεί από διαφορετικά νήματα. Αυτό αποτελεί ασάφεια, καθώς οι οδηγίες διαμοιρασμού μνήμης, καθορίζουν με σαφήνεια τον τρόπο που οι εργασίες διανέμονται στα νήματα. Για να εγγυηθεί το \en{OpenMP} ότι κάθε διεργασία εκτελείται μόνο μια φορά, η κατασκευή τους θα πρέπει να ενσωματωθούν σε μια οδηγία \emph{\en{single}} ή \emph{\en{master}}.

		\selectlanguage{english}
		\begin{lstlisting}[tabsize=4, basicstyle=\small, language=C++, caption={\el{Παράδειγμα κώδικα με διεργασίες}}, frame=tb]
#include <omp.h>

int main(void) {	
	#pragma omp parallel          // Begin of parallel region
	{
		#pragma omp single
		{
			#pragma omp task
				func_task_1();    // First task creation
			#pragma omp task
				func_task_2();    // Second task creation
		}// end of single region  // Implicit barrier
	} //end of parallel region    // Implicit barrier
}
\end{lstlisting}
\selectlanguage{greek}
\clearpage
Το παραπάνω παράδειγμα αποτελεί επεξήγηση της λειτουργίας των διεργασιών. Υπάρχουν δύο νήματα, ένα νήμα συναντά την οδηγία \emph{\en{single}} και αρχίζει να εκτελεί το μπλοκ κώδικα. Δύο διεργασίες δημιουργούνται, αλλά δεν έχουν ακόμη εκτελεστεί. Tο άλλο νήμα συναντά και περιμένει στο  \emph{\en{barrier}} στο τέλος της οδηγίας \emph{\en{single}}. Στην περίπτωση την διεργασιών, τα αδρανή νήματα δεν περιμένουν στο \emph{\en{barrier}}. Αντι αυτου, αυτά τα νήματα είναι διαθέσιμα για την εκτέλεση των διεργασιών. Το νήμα που δημιουργεί τις διεργασίες καταλήγει επίσης στο φράγμα, μόλις ολοκληρωθεί η φάση παραγωγής και μπορεί να εκτελεί και αυτό διεργασίες. Η σειρά με την οποία θα εκτελεστούν οι διεργασίες δεν καθορίζεται.

\subsubsection{Η οδηγία διεργασιών}
\subparagraph{}
	Ορίζει μια ρητή διεργασία. Το περιβάλλον δεδομένων της διεργασίας δημιουργείται σύμφωνα με τις φράσεις χαρακτηριστικών κοινής χρήσης δεδομένων κατασκευή διεργασίων και τυχόν προεπιλογές που ισχύουν\cite{thenextstep20}.
	
\selectlanguage{english}
\begin{lstlisting}[language=C++, caption={\el{Σύνταξη διεργασίας}} , frame=tlrb]{Name}
		#pragma omp task [clause[ [, ]clause] ...] 
			structured-block 
\end{lstlisting}

\begin{lstlisting}[language=C++, caption={\el{Αποδεκτές φράσεις οδηγίας} sections} , frame=tlrb]{Name}
if([ task :] scalar-expression) 
final(scalar-expression) 
untied 
default(shared | none) 
mergeable 
private(list) 
firstprivate(list) 
shared(list) 
in_reduction(reduction-identifier : list) 
depend([depend-modifier,] dependence-type : locator-list) 
priority(priority-value) 
allocate([allocator :] list) 
affinity([aff-modifier :] locator-list) 
detach(event-handle)
\end{lstlisting}
\selectlanguage{greek}

\paragraph{Φράση \en{if}}
\subparagraph{}
Η έκφραση της συνθήκης \emph{\en{if}} μιας διεργασίας, θα πρέπει να αξιολογείται ώς ψευδής ή αληθής. 
Η οδηγία των διεργασιών παρέχει μια φράση \emph{\en{if}}, που δέχεται μια έκραση ως όρισμα. Αν η έκφραση αξιολογείται ώς ψευδής, απαγορεύει η αναβολή της διεργασίας.
Ανεξαρήτου αποτελέσματος της έκφρασης, δημιουργείται πάντα ένα νέο περιβάλλον δεδομένων εργίας. Αν η έκφραση αξιολογείται ως ψευδής, δεν γίνονται όλοι οι υπολογισμοί που απετούνται για μια διεργασία αν δεν υπήρχε η φράση if, επομένως θα μπορούσαν να αποφευχθούν ορισμένα κόστη επιδόσεων.

Ως εκ τούτου η φράση \emph{\en{if}} αποτελεί λύση σε καταστάσεις, όπως οι αναδρομικοί αλγόριθμοι, όπου
το υπολογιστικό κόστος μειώνεται καθώς αυξάνεται το βάθος και το όφελος από τη δημιουργία μιας νέας διεργασίας
μειώνεται λόγω του γενικού κόστους.
Ωστόσο, για λόγους ασφαλείας \cite{parallel_dist}, οι μεταβλητές που αναφέρονται σε μια οδηγία κατασκευής διεργασιών είναι, στις περισσότερες περιπτώσεις, από προεπιλογή \emph{\en{firstprivate}}, επομένως το κόστος
της ρύθμισης περιβάλλοντος δεδομένων μπορεί να είναι το κυρίαρχο συστατικό της δημιουργίας διεργασιών.

\paragraph{Φράση \en{final}}
\subparagraph{}
Η φράση χρησιμοποιείται για να ελέγχεται με ευκρίνεια η ευαισθησία των διεργασιών. Οταν χρησιμοποιείται μέσα στην οδηγεία \emph{\en{task}}, η έκφραση αξιολογείται κατά τη διάρκεια δημιουργίας αυτού.  Αν είναι αληθής, η διεργασία θεωρείται τελική. Όλες οι διεργασίες που δημιουργούνται μέσα σε αυτή τη διεργασία, αγνοούνται και εκτελούνται στο πλαίσιο της.
\clearpage
Υπάρχουν δυο διαφορές ανάμεσα στη φράση \emph{\en{if}} και στη \emph{\en{final}}: την κατασκευή διεργασιών που επηρεάζει και τον τρόπο με τον οποίο οι κατασκευές αγνοούνται\cite{tasking1}.

\begin{table}[htbp]
\captionsetup{justification=raggedright,
singlelinecheck=false
}
\caption{Διαφορές ανάμεσα στις φράσεις \emph{\en{if}} και \emph{\en{final}} οταν εισάγονται σε κατασκευή διεργασίας.}
\def\arraystretch{1.5}
\begin{tabular}{| p{0.5\textwidth} | p{0.5\textwidth}|}
\textbf{\en{\emph{if} clause}} \cellcolor[HTML]{D0D0D0} & \textbf{\en{\emph{final} clause}} \cellcolor[HTML]{D0D0D0} \\
\hline
Επηρεάζει την διεργασία που κατασκευάζεται από τη συγκεκριμένη οδηγιά κατασκευής διεργασιών & Επηρεάζει όλες τις διεργασίες "απογόνους" δηλαδή όλες τις διεργασίες που πρόκειται να δημιουργηθούν μέσα από αυτή που περιείχε τη φράση \emph{\en{final}}\\
\hline
Η οδηγία κατασκευής διεργασίας αγνοείται εν μέρει. Η διεργασία και το περιβάλλον μεταβλητών δημιουργείται ακόμα και αν η έκφραση αξιολογηθεί ως \emph{\en{false}} & Αγνοείται εντελώς η κατασκευή διεργασιών δηλαδή δε θα δημιουργηθεί νέα εργασία ούτε νέο περιβάλλον δεδομένων.	\\
\hline
\end{tabular}
\end{table}

\paragraph{Φράση \en{mergeable}}
\subparagraph{}

Μια συγχωνευμένη διεργασία είναι η διεργασία της οποίας το περιβάλλον δεδομένων είναι ίδιο με το περιβάλλον που δημιούργησε την διεργασία. Οταν εισάγεται η φράση \emph{\en{mergeable}} σε μια οδηγία δημιουργίας διεργασίας τότε η υλοποίηση μπορεί να δημιουργήσει μια συγχωνευμένη διεργασία.

\paragraph{Φράση \en{depend}}
\subparagraph{}

Η φράση \emph{\en{depend}} επιβάλλει πρόσθετους περιορισμούς στη δημιουργία διεργασιών. Αυτοί οι περιορισμοί δημιουργούν εξαρτήσεις μόνο μεταξύ συγκενικών διεργασιών.
\clearpage

\paragraph{Φράση \en{untied}}
\subparagraph{}
Κατά την επανάληψη μιας περιοχής διεργασίας που έχει τεθεί σε αναστολή, μια δεσμευμένη διεργασία θα πρέπει να εκτελεστεί ξανά από το ίδιο νήμα. Με τη φράση \emph{\en{untied}}, δεν υπάρχει τέτοιος περιορισμός και η διεργασία συνεχίζεται από οποιοδήποτε νήμα.


\ \\
\subsubsection{Συγχρονισμός διεργασιών}
\subparagraph{}
Οι διεργασίες δημιουργούνται όταν υπάρχει μια οδηγία δημιουργίας διεργασιών. Η στιγμή εκτέλεσης τους δεν είναι καθορισμένη. Το \emph{\en{OpenMP}} εγγυάται οτι θα έχουν ολοκληρωθεί όταν ολοκληρωθεί η εκτέλεση του προγράμματος, ή όταν προκύψει κάποια οδηγία συγχρονισμού διεργασίας.
\ \\
\ \\

\begin{table}[htbp]
\captionsetup{justification=raggedright,
singlelinecheck=false
}
\caption{Οδηγιες συγχρονισμού διεργασιών.}
\def\arraystretch{1.5}
\begin{tabular}{| p{0.5\textwidth} | p{0.5\textwidth}|}
\textbf{Οδηγία} \cellcolor[HTML]{D0D0D0} & \textbf{Περιγραφή} \cellcolor[HTML]{D0D0D0} \\
\hline
\textbf{\en{{\#}pragma omp barrier}} & Μπορεί είτε να υπάρχει υπονοούμενη είτε να δηλωθεί ρητά. \\
\hline
\textbf{\en{{\#}pragma omp taskwait}} & Περιμένει μέχρι να ολοκληρωθούν όλες οι διεργασίες παιδιά της συγκεκριμένης διεργασίας.\\
\hline
\textbf{\en{{\#}pragma omp taskgroup}} & Περιμένει μέχρι να ολοκληρωθούν όλες οι διεργασίες παιδιά της συγκεκριμένης διεργασίας αλλά και οι απόγονοι τους.\\
\hline
\end{tabular}
\end{table}

Στην περίπτωση της δημιουργίας διεργασιών μέσω οδηγίας single, υπάρχει υπονοούμενο φράγμα εκτέλεσης, σε αντίθεση με την οδηγία master που δεν υπονοείται.
\clearpage
\selectlanguage{english}
\begin{lstlisting}[language=C++, caption={\el{Παράδειγμα} taskwait} , frame=tlrb]{Name}
#pragma omp parallel {
	#pragma omp single
	{
		#pragma omp task
		{
			function1();
		}//Task #1
		#pragma omp task
		{
			function2();
		} // Task #2

		#pragma omp taskwait
			last_to_be_executed();
	} // End of single region
} // End of parallel region
\end{lstlisting}
\selectlanguage{greek}