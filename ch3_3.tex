\clearpage
\subsection{\en{Tasking}}
Η έννοια των εργασιών(\textbf{\en{Tasks}}) εισήχθει στο \emph{\en{OpenMP}} το 2008, με την έκδοση
3.0\cite{parallel_dist}. Οι εργασίες παρέχουν τη δυνατότητα παραλληλοποίησης αλγορίθμων με ακανόνιστη και μη εξαρτώμενη
από το χρόνο ροή εκτέλεσης εργασιών. Ένας μηχανισμός ουράς διαχειρίζεται δυναμικά την ανάθεση εργασιών που πρέπει να εκτελεστούν στα διαθέσιμα νήματα. Τα νήματα παραλαμβάνουν εργασίες από την ουρά έως ότου αυτή αδειάσει. Η εργασία είναι ένα
τμήμα κώδικα που ορίζεται εντός της παράλληλης περιοχής και εκτελείται ασύγχρονα και ταυτόχρονα με τις υπόλοιπες εργασίες της ίδιας περιοχής. 

Οι εργασίες αποτελούν τμήμα της παράλληλης περιοχής κώδικα. Χωρίς ειδική μέριμνα, κάθε εργασία μπορεί να
εκτελεστεί από κάθε νήμα. Αυτό αποτελεί πρόβλημα, καθώς οι οδηγίες διαμοιρασμού καθορίζουν με σαφήνεια
τον τρόπο που οι εργασίες διανέμονται στα νήματα. Για να εγγυηθεί το \emph{\en{OpenMP}} ότι κάθε εργασία εκτελείται
μόνο μια φορά, η κατασκευή τους θα πρέπει να ενσωματωθεί σε μια οδηγία \emph{\en{single}} ή \emph{\en{master}}.
		\selectlanguage{english}
\begin{lstlisting}[tabsize=4, basicstyle=\small, language=C++, caption={\el{Χρήση εργασιών}}, frame=tb]
#include <omp.h>

int main(void) {	
	#pragma omp parallel          // Beginning of parallel region
	{
		#pragma omp single
		{
			#pragma omp task
				func_task_1();    // First task creation
			#pragma omp task
				func_task_2();    // Second task creation
		}// end of single region  // Implicit barrier
	} //end of parallel region    // Implicit barrier
}
\end{lstlisting}
\selectlanguage{greek}
\clearpage

\mbox{}
Στο προηγούμενο παράδειγμα έστω ότι υπάρχουν δύο νήματα. Το ένα νήμα συναντά την οδηγία \emph{\en{single}} και εκτελεί το τμήμα κώδικα που βρίσκεται εντός αυτής. Δύο εργασίες δημιουργούνται χωρίς
να έχει ξεκινήσει όμως η εκτέλεσή τους. Ταυτόχρονα, το δεύτερο νήμα συναντά την υποκείμενη οδηγία \emph{\en{barrier}}
στο τέλος της οδηγίας \emph{\en{single}} και περιμένει εκεί έως ότου εισαχθεί μια εργασία στην ουρά εργασιών προς
εκτέλεση. Τα νήματα που καταλήγουν στο υποκείμενο φράγμα, είναι διαθέσιμα για την εκτέλεση των εργασιών. Ως συνέπεια, το νήμα που δημιουργεί τις εργασίες
καταλήγει επίσης στο υποκείμενο φράγμα όταν ολοκληρωθεί η διαδικασία παραγωγής εργασιών και είναι και αυτό διαθέσιμο
για εκτέλεση των εργασιών που απομένουν. Η σειρά εκτέλεσης των εργασιών δεν είναι προκαθορισμένη.
\subsubsection{Η οδηγία \emph{\en{task}}}
Βασική οδηγία της έννοιας των εργασιών είναι η \textbf{\en{pragma omp task}} που ορίζει μια ρητή εργασία. Το περιβάλλον δεδομένων της εργασίας δημιουργείται σύμφωνα με τις \hyperref[sec:link]{\emph{φράσεις διαμοιρασμού δεδομένων}} κατά την
κατασκευή των εργασιών και τυχόν προεπιλογές που ισχύουν για τις φράσεις αυτές\cite{thenextstep20}.
	
\selectlanguage{english}
\begin{spacing}{1.1}
\begin{lstlisting}[caption={\el{Γραμματική οδηγίας} \emph{\en{task}}} , frame=tlrb]{Name}
#pragma omp task [clause[ [, ]clause] ...] 
	structured-block 
\end{lstlisting}
\begin{lstlisting}[language=C++, caption={\el{Φράσεις οδηγίας} \emph{\en{task}}} , frame=tlrb]{Name}
if([ task :] scalar-expression) 
final(scalar-expression) 
untied 
default(shared | none) 
mergeable 
private(list) 
firstprivate(list) 
shared(list) 
in_reduction(reduction-identifier : list) 
depend([depend-modifier,] dependence-type : locator-list) 
priority(priority-value) 
allocate([allocator :] list) 
affinity([aff-modifier :] locator-list) 
detach(event-handle)
\end{lstlisting}
\end{spacing}
\selectlanguage{greek}
% !TeX spellcheck = en_US 
\paragraph{Φράση \en{if}}
\ \\
Η έκφραση που δέχεται η συνθήκη \textbf{\en{if}} ως όρισμα σε μια εργασία, αξιολογείται ως ψευδής ή αληθής. Όταν η έκφραση αξιολογείται ως ψευδής, το νήμα \textbf{A} που δημιουργεί τη νέα εργασία αναστέλλεται έως ότου αυτή ολοκληρωθεί. Η εργασία του νήματος \textbf{Α} δε μπορεί να συνεχιστεί από κάποιο άλλο νήμα, ακόμα και αν αυτό είναι \en{untied})\cite{if_clause}.
Ανεξαρτήτου αποτελέσματος της έκφρασης ωστόσο,
δημιουργείται πάντα ένα νέο περιβάλλον δεδομένων για τη εργασία.

Ως εκ τούτου η φράση \emph{\en{if}} αποτελεί λύση σε αλγόριθμους όπως οι αναδρομικοί, όπου το υπολογιστικό κόστος
μειώνεται καθώς αυξάνεται το βάθος, αλλά το όφελος από τη δημιουργία μιας νέας εργασίας μειώνεται λόγω του γενικού
κόστους κατασκευής της. Παρόλα αυτά, το κόστος της ρύθμισης περιβάλλοντος δεδομένων μπορεί να είναι το κυρίαρχο για τη
δημιουργία εργασίας, διότι για λόγους ασφαλείας οι μεταβλητές που αναφέρονται σε μια οδηγία κατασκευής εργασιών
είναι στις περισσότερες περιπτώσεις από προεπιλογή \emph{\en{firstprivate}}\cite{parallel_dist}. 

\paragraph{Φράση \en{final}}
\ \\
Όταν η φράση \emph{\en{final}} 
χρησιμοποιείται μέσα στην οδηγία \emph{\en{task}}, η έκφραση αξιολογείται κατά τη διάρκεια δημιουργίας της εργασίας.
Αν είναι αληθής, η εργασία θεωρείται τελική. Όλες οι εργασίες παιδιά που δημιουργούνται μέσα σε αυτή, αγνοούνται και
εκτελούνται στο πλαίσιο της. Επί της ουσίας πρόκειται για άμεση σειριακή εκτέλεση του κώδικα εντός της περιοχής της νέας εργασίας από το ίδιο νήμα\cite{if_clause}.

Υπάρχουν δύο διαφορές ανάμεσα στη φράση \emph{\en{if}} και στη \emph{\en{final}}. Οι διαφορές αυτές αφορούν:

\begin{itemize}
\item{τις εργασίες που επηρεάζει κάθε φράση}
\item{τον τρόπο με τον οποίο  διαχειρίζεται ο μεταγλωττιστής της κατασκευασμένες εργασίες}\cite{tasking1}.
\end{itemize}
\clearpage

\begin{table}[htbp]
\captionsetup{justification=raggedright, singlelinecheck=false} \caption{Διαφορές ανάμεσα στις φράσεις \emph{\en{if}}
και \emph{\en{final}} όταν εισάγονται σε κατασκευή εργασίας.} \def\arraystretch{1.5}
\begin{tabular}{| p{0.5\textwidth} | p{0.5\textwidth}|}
\textbf{\en{\emph{if} clause}} \cellcolor[HTML]{D0D0D0} & \textbf{\en{\emph{final} clause}} \cellcolor[HTML]{D0D0D0} \\
\hline
Επηρεάζει την εργασία που κατασκευάζεται από τη συγκεκριμένη οδηγία κατασκευής εργασιών & Επηρεάζει όλες τις
εργασίες "απογόνους" δηλαδή όλες τις εργασίες που πρόκειται να δημιουργηθούν μέσα από αυτή που περιείχε τη φράση
\emph{\en{final}}\\
\hline
Η οδηγία κατασκευής εργασίας αγνοείται εν μέρει. Η εργασία και το περιβάλλον μεταβλητών δημιουργείται ακόμα και αν η
έκφραση αξιολογηθεί ως \emph{\en{false}} & Αγνοείται εντελώς η κατασκευή εργασιών δηλαδή δε θα δημιουργηθεί νέα
εργασία, αλλά ούτε και νέο περιβάλλον δεδομένων.	\\
\hline
\end{tabular}
\end{table}

\paragraph{Φράση \en{mergeable}}
\mbox{}

Μια \emph{"συγχωνευμένη"} εργασία είναι αυτή της οποίας το περιβάλλον δεδομένων είναι ίδιο με αυτό της
εργασίας που τη δημιούργησε. Όταν εισάγεται η φράση \emph{\en{mergeable}} σε μια οδηγία \emph{\en{task}} τότε η
υλοποίηση της δημιουργεί μια συγχωνευμένη εργασία.
\paragraph{Φράση \en{depend}}
\mbox{}

Η φράση \textbf{\en{depend}} επιβάλλει πρόσθετους περιορισμούς στη δημιουργία εργασιών. Αυτοί οι περιορισμοί
δημιουργούν εξαρτήσεις μόνο μεταξύ συγγενικών εργασιών.

\selectlanguage{english}
\begin{spacing}{1.1}
\begin{lstlisting}[language=C++, caption={\el{Φράση} \emph{\en{depend}}} , frame=tlrb]{Name}
depend([depend-modifier, ]dependency-type : locator-list)

dependence-type:
	in
	out
	inout
	mutexinoutset
	depobj	
	
depend-modifier:
	iterator(iterators-definition)
	depend([source|depend(sink: vec)])
\end{lstlisting}
\end{spacing}
\selectlanguage{greek}
\clearpage
Οι εξαρτήσεις των εργασιών καθορίζονται από τον τύπο που ορίζεται σε μια φράση εξάρτησης και την λίστα των
περιεχόμενων αντικειμένων που ακολουθεί. Ο τύπος εξάρτησης μπορεί να είναι ένας από τους ακόλουθους\cite{oracle1}.

Για τον τύπο \textbf{\en{in(x)}}, η προκύπτουσα εργασία (\en{y}) θα είναι εξαρτημένη όλων των προηγούμενων συγγενικών
εργασιών που έχουν τύπο εξάρτησης \emph{\en{out}} ή \emph{\en{inout}} και το \textbf{\en{x}} αναφέρεται στην λίστα. Η
\en{y} θα ξεκινήσει μετά την ολοκλήρωση των εξαρτήσεών της.

Για τους τύπους εξάρτησης \emph{\textbf{\en{out(x)}}} και \emph{\textbf{\en{inout}}}, η προκύπτουσα εργασία  θα
εκτελεστεί και ολοκληρωθεί πριν από την εκτέλεση των εργασιών τύπου \emph{\en{in} ή \en{inout}} και περιέχουν την
μεταβλητή \textbf{\en{x}} στη λίστα. \ \\

\textbf{Παράδειγμα}
\begin{itemize}
\item{Η φράση \emph{\en{depend(in:x)}} θα δημιουργήσει μια εξάρτηση με όλες τις εργασίες που δημιουργήθηκαν με τη
φράση \emph{\en{depend(out:x)} ή \en{depend(inout: x)}}}.

\item{Η φράση \emph{\en{depend(out:x)} ή \en{depend(inoout:x)}} θα δημιουργήσει μια εργασία εξαρτημένη με όλες τις
προηγούμενες εργασίες που αναφέρουν τη μεταβλητή \emph{\en{x}} στη φράση εξάρτησης.}
\end{itemize}

Από τα παραπάνω, προκύπτει η εξής εξάρτηση εργασιών:
\selectlanguage{english}
\begin{spacing}{1.1}
\begin{lstlisting}[language=C++, caption={\el{Παράδειγμα εξάρτησης εργασιών}} , frame=tlrb]{Name}
#pragma omp task depend(out:x) // task1
...
#pragma omp task depend(in:x) depend(out:y) // task2
...
#pragma omp task depend(inout:x) // task3
...
#pragma omp task depend(in:x, y) // task4
...


task1(out:x) -> task2(in:x, out:y) -> task4(in:x, y)
		|
		-> task3(inout:x)
\end{lstlisting}
\end{spacing}
\selectlanguage{greek}
\clearpage
\paragraph{Φράση \en{untied}}
\ \\
Κατά τη συνέχιση μιας εργασίας που έχει τεθεί σε αναστολή, μια δεσμευμένη εργασία θα πρέπει να εκτελεστεί ξανά από
το ίδιο νήμα. Με τη φράση \emph{\en{untied}}, δεν υπάρχει τέτοιος περιορισμός και η εργασία συνεχίζεται από
οποιοδήποτε νήμα.
 
\subsubsection{Συγχρονισμός εργασιών}
Οι εργασίες δημιουργούνται όταν το πρόγραμμα συναντήσει την οδηγία \textbf{\en{pragma omp task}}. Η στιγμή εκτέλεσης
τους δεν είναι προκαθορισμένη. Το \emph{\en{OpenMP}} εγγυάται ότι θα έχουν ολοκληρωθεί όταν ολοκληρωθεί η εκτέλεση του
προγράμματος ή όταν προκύψει κάποια οδηγία συγχρονισμού νημάτων.

Στην περίπτωση της δημιουργίας εργασιών μέσω οδηγίας \emph{\en{\textbf{single}}} στο τέλος της οδηγίας υπάρχει
υποκείμενο φράγμα εκτέλεσης, σε αντίθεση με την οδηγία \emph{\en{\textbf{master}}}.

\begin{table}[h]
\captionsetup{justification=raggedright,
singlelinecheck=false
}
\caption{Οδηγίες συγχρονισμού εργασιών.}
\def\arraystretch{0.5}
\begin{tabular}{| p{0.3\textwidth} | p{0.6\textwidth}|}
\hline
\textbf{Οδηγία} \cellcolor[HTML]{D0D0D0} & \textbf{Περιγραφή} \cellcolor[HTML]{D0D0D0} \\
\hline
\textbf{\en{{\#}pragma omp barrier}} & Μπορεί να υπονοείται μέσω μιας οδηγίας ή να δηλωθεί ρητά από το χρήστη. \\
\hline
\textbf{\en{{\#}pragma omp taskwait}} & Αναμένει μέχρι να ολοκληρωθούν όλες οι εργασίες-παιδιά της συγκεκριμένης
εργασίας.\\
\hline
\textbf{\en{{\#}pragma omp taskgroup}} & Αναμένει μέχρι να ολοκληρωθούν όλες οι εργασίες παιδιά της συγκεκριμένης
εργασίας αλλά και οι απόγονοι τους.\\
\hline
\end{tabular}
\end{table}

\selectlanguage{english}
\begin{spacing}{1.0}
\begin{lstlisting}[language=C++, caption={\el{Χρήση} taskwait} , frame=tlrb]{Name}
#pragma omp parallel {
	#pragma omp single
	{
		#pragma omp task
		{ ... } // Task #1

		#pragma omp task
		{ ... } // Task #2

		#pragma omp taskwait
			last_to_be_executed();
	} // End of single region
} // End of parallel region
\end{lstlisting}
\end{spacing}
\selectlanguage{greek}