\subsection{Πρόσθεση διανυσμάτων αριθμών μικρής ακρίβειας - \emph{\en{SAXPY}}}
\subparagraph{}
Μια από τις λειτουργίες που κατέχει θεμελιώδη θέση σε εφαρμογές της γραμμικής άλγεβρας, αποτελεί η πρόσθεση διανυσμάτων
δεκαδικών αριθμών μικρής ακρίβειας (\emph{\en{floats}}), γνωστή ως \textbf{\en{SAXPY}}.

Στο παράδειγμα \emph{\en{SAXPY}}, ο λόγος του μεγέθους υπολογισμών προς το μέγεθος των δεδομένων που τελούν υπό επεξεργασία
είναι μικρός. Ως εκτότου, αποτελεί πρόβλημα περιορισμένης επεκτασιμότητας. Παρόλα αυτά, πρόκειται για ένα χρήσιμο
παράδειγμα που ανήκει στην κατηγορία προβλημάτων παραλληλοποίησης τύπου \emph{\en{map}} και των εννοιών
\emph{\en{uniform}} και \emph{\en{varying parameters}}\cite{patters}.
\\
\subsubsection{Περιγραφή προβλήματος}
Η λειτουργία \en{SAXPY} δέχεται ως δεδομένα δυο διάνυσματα δεκαδικών αριθμών. Το πρώτο διάνυσμα πολλαπλασιάζεται με μια
σταθερά \emph{\en{a}} και το αποτέλεσμα προστίθεται στο δεύτερο διάνυσμα \emph{\en{y}}. Τα διανύσματα \emph{\en{x,y}}
πρέπει να έχουν το ίδιο μέγεθος.

Ο υπολογισμός αυτός εμφανίζεται συχνά στη γραμμική άλγεβρα, όπως για παράδειγμα στη διαγραφή σειρών για την απαλοιφή
\textbf{\en{Gauss}}. Το όνομα \emph{\en{SAXPY}} δόθηκε από την βιβλιοθήκη \en{\textbf{BLAS} (\emph{"Basic Linear Algrbra
Subprograms"})} για δεκαδικούς αρθμούς μικρής ακρίβειας (\emph{\en{floats}}). Ο αντίστοιχος αλγόριθμος διπλής ακρίβειας
ονομάζεται \emph{\en{DAXPY}}, ενώ για μιγαδικούς αριθμούς ονομάζεται \emph{\en{CAXPY}}.
Η μαθηματική διατύπωση του \emph{\en{SAXPY}} είναι:
                              $$\en{\textbf{y} = a*\textbf{x} + \textbf{y}}$$ όπου το διάνυσμα \emph{\en{x}}
χρησιμοποιείται ως είσοδος, το \en{\emph{y}} ως είσοδος και έξοδος. Δηλαδή το αρχικό διάνυσμα \emph{\en{y}}
τροποποιείται. Εναλλακτικά, η λειτουργία \emph{\en{SAXPY}} μπορεί να περιγραφεί ως συνάρτηση που δρα σε μεμονωμένα
στοιχεία, οπως φαίνεται παρακάτω: 
\begin{equation*}\label{eq:pareto mle2}
\begin{aligned}
f(t, p, q) = tp + q\\
\forall _i : y_i \leftarrow f(a, x_i, y_i)
\end{aligned}
\end{equation*}
\clearpage
Οι συναρτήσεις τύπου \emph{\en{f}} δεχονται ως ορίσματα, δύο είδη παραμέτρων. Τις παραμέτρους όπως την \en{\emph{a}} που
παραμένουν σταθερές και ονομάζονται \emph{\en{uniform}}, οι παράμετροι που είναι μεταβλητές σε κάθε κλίση της
\emph{\en{f}} ονομάζονται \emph{\en{varying}}. To μοτίβο \emph{\en{map}} καλεί τη συνάρτηση \emph{\en{f}} τόσες φορές
όσες και ο αριθμός των στοιχείων του διανύσματος.\cite{patters}.

\subsection{Περιγραφή κεντρικού τμήματος προβλήματος \en{SAXPY}}
\subparagraph{}
Το πρόβλημα ξεκινάει δημιουργώντας ένα στοιχείο τύπου \emph{\en{Containers}}, που περιέχει τα διανύσματα που εισάγονται
στον αλγόριθμο \en{SAXPY}. Ο ρόλος του \en{Containers} είναι για την διαχείριση της \en{\emph{heap}} μνήμης. Τα
διανύσματα και η σταθερά \en{cons} αρχικοποιούνται με τυχαίους αριθμούς μικρής ακρίβειας. Το μέγεθος των
διανυσμάτων (μέγεθος προβλήματος) ορίζεται από τον χρήστη μέσω της γραμμής εντολών. Στη συνέχεια, καλείται ο αλγόριθμος
\en{SAXPY} μόλις τελειώσει γίνεται επαλήθευση των αποτελεσμάτων, όπου αν επαληθευτούν σωστά, γίνεται εκτύπωση του
χρόνουν εκτέλεσης της παραλλαγής.
\\
\selectlanguage{english}
\begin{spacing}{1.0}
\begin{lstlisting}[language=C++, caption={\el{Κεντρικός κώδικας προβλήματος \en{SAXPY}}} , frame=tlrb]{Name}
int main(int argc, char **argv) {
    Opts o;
    parseArgs(argc, argv, o);
    Containers c(o.size);
    c.setRandomValues();
    float cons = float(rand()) / float(RAND_MAX);
    auto start = omp_get_wtime();
    saxpy(c.m_size, cons, c.m_a, c.m_b);
    auto end = omp_get_wtime();
    verify(c.m_size, cons, c.m_a, c.m_b, c.m_verification);
    std::cout << "Execution Time : " << std::fixed
         << end - start << std::setprecision(5);
    std::cout << " sec " << std::endl;
    return 0;
}
   
\end{lstlisting}
\end{spacing}
\selectlanguage{greek}
\clearpage
\selectlanguage{english}
\begin{spacing}{0.8}
\begin{lstlisting}[language=C++, caption={\el{Κλάση \emph{\en{Containers}}}} , frame=tlrb]{Name}
struct Containers {
    explicit Containers(size_t containers_size);
    ~Containers();

    size_t m_size;
    float *m_a;
    float *m_verification;
    float *m_b;
};     

Containers::Containers(size_t containers_size)
    : m_size(containers_size) {
    srand(time(nullptr));
    m_a = new float[containers_size];
    m_verification = new float[containers_size];
    m_b = new float[containers_size];
}

Containers::~Containers() {
    delete []m_a;
    delete []m_b;
    delete []m_verification;
}

Containers::setRandomValues() {
    fill_random_arr(m_a, m_size);
    fill_random_arr(m_b, m_size);
}
\end{lstlisting}
\end{spacing}
\selectlanguage{greek}

\selectlanguage{english}
\begin{spacing}{1}
\begin{lstlisting}[language=C++, caption={\el{Συνάρτηση επαλήθευσης}} , frame=tlrb]{Name}
static void verify(size_t size, float c, float *a, float *b, 
                    float *verification) {
    for (size_t i = 0; i < size; ++i) {
        if (abs(c * a[i] + verification[i] - b[i]) >= 10e-6) {
            std::cout << "Failed index: " << i <<
             ". " << c * a[i] + verification[i] << 
             " =! " << b[i] << std::endl;
            exit(1);
        }
    }
}
\end{lstlisting}
\end{spacing}
\selectlanguage{greek}

\selectlanguage{english}
\begin{spacing}{1}
\begin{lstlisting}[language=C++, caption={\el{Συνάρτηση αρχικοποίησης τιμών}} , frame=tlrb]{Name}
static void fill_random_arr(float *arr, size_t size) {
    for (size_t k = 0; k < size; ++k) {
        arr[k] = (float)(rand()) / RAND_MAX;
    }
}       
\end{lstlisting}
\end{spacing}
\selectlanguage{greek}

\clearpage
\subsection{Σειριακή εκτέλεση}
\subparagraph{}
Η υλοποίηση της σειριακής παραλλαγής της συνάρτησης \en{saxpy} περιλαμβάνει έναν επαναληπτικό 
βρόγχο στον οποίο γίνεται ο υπολογισμός για κάθε στοιχείο των διανυσμάτων.
\selectlanguage{english}
\begin{spacing}{0.8}
\begin{lstlisting}[language=C++, caption={\el{Σειριακή υλοποίηση της \en{SAXPY}}} , frame=tlrb]{Name}
void saxpy(size_t n, float a, const float *x, float *y) {
    for (size_t i = 0; i < n; ++i) {
        y[i] = a * x[i] + y[i];
    }
}   
\end{lstlisting}
\end{spacing}
\selectlanguage{greek}

Οι χρόνοι εκτέλεσης του αλγορίθμοι συναρτήσει του μεγέθους του προβλήματος παρατίθενται στον παρακάτων πίνακα.
Το πρόγραμμα μεταγλωττίστηκε με επιλογή -O3 kai -O0.


\begin{table}[h]
    \centering
    \caption{Καταγραφή χρόνων εκτέλεσης}
    \label{my-label}
    \begin{tabular}{|p{0.30\textwidth}
    | >{\centering\arraybackslash}p{0.12\textwidth}
    | >{\centering\arraybackslash}p{0.12\textwidth}
    | >{\centering\arraybackslash}p{0.12\textwidth}
    |}
    \hline
    \multirow{2}{*}{\textbf{Μέγεθος προβλήματος}} & \multicolumn{3}{|c|}{\textbf{Χρόνοι εκτέλεσης \en{(sec)}}} \\ \cline{2-4} 
               & \textbf{-Ο0} & \textbf{-Ο3} & \en{\textbf{-O3 -fno-tree-vectorize}}\\ \hline
     100000    &  0.0012 & 0.00011 & 0.0002 \\ \cline{1-4} 
     1000000   &  0.0118 & 0.00171 & 0.0021 \\ \cline{1-4} 
     10000000  &  0.1179 & 0.01631 & 0.0203 \\ \cline{1-4} 
     100000000 &  1.1821 & 0.16124 & 0.2397 \\ \cline{1-4} 
     200000000 &  2.3612 & 0.31867 & 0.3981 \\ \cline{1-4} 
     300000000 &  3.5510 & 0.42806 & 0.6052 \\ \cline{1-4} 
     400000000 &  4.7291 & 0.6339  & 0.7884 \\ \cline{1-4} 

    \end{tabular}
\end{table}

\begin{figure}[h]
\begin{tabular}{*{2}{>{\centering\arraybackslash}b{\dimexpr0.5\linewidth-2\tabcolsep\relax}}}
\resizebox{0.35\textwidth}{!} {
\begin{tikzpicture}[state/.append style={minimum size=7mm}]
     \begin{axis}[
         title={Χρόνοι εκτέλεσης \en{SAXPY}},
         xlabel={Μέγεθος διανυσμάτων},
         ylabel={Χρόνος εκτέλεσης},
         xmin=100000, xmax=300000000,
         ymin=0, ymax=4,
         xtick={ 100000000, 200000000, 300000000},
         ytick={ 0, 1, 2, 3, 4 },
         legend pos=north west,
        % ymajorgrids=true,
        % grid style=dashed,
     ]
    
     \addplot[ color=blue, mark=square,]
      coordinates {
          (100000,0.0012)(1000000,0.0118)(10000000,0.1179)
          (100000000,1.1821)(200000000,2.3612)(300000000,3.5510)
 	};
     \addlegendentry{-O0}
     
     \addplot[ color=red, mark=square,]
      coordinates {
          (100000,0.00011)(1000000,0.00171)(10000000,0.01631)
          (100000000,0.16124)(200000000,0.31867)(300000000,0.42806)
 	};
 	\addlegendentry{-O3}
 	
 	\addplot[ color=green, mark=square,]
      coordinates {
          (100000,0.0002)(1000000,0.0021)(10000000,0.0203)
          (100000000,0.2397)(200000000,0.3981)(300000000,0.6052)
 	};
 	\addlegendentry{-O3 \en{no-vec}}
     \end{axis}
 \end{tikzpicture}}

\caption{Σύγκριση αποτελεσμάτων}
    &
\renewcommand{\arraystretch}{1.1}
\begin{tabular}{c|c}
Μέγεθος & Επιτάχυνση (\%)  \\
\hline
100000    & 90.8 \\
1000000   & 85.5 \\
10000000  & 86.2  \\
100000000 & 86.4  \\
200000000 & 86.5 \\
300000000 & 87.9
\end{tabular}
\captionof{table}{Ποσοστό μείωσης με -Ο3}
\end{tabular}
\end{figure}

\clearpage
\subsection{Παραλλαγή με οδηγία \en{parallel for}}
\subparagraph{}
Η πρώτη υλοποίηση παραλλαγής με παραλληλισμό της συνάρτησης \en{saxpy} περιλαμβάνει τον επαναληπτικό βρόγχο ενσωματωμένο στην οδηγία \emph\en{parallel for} στον οποίο γίνεται ο υπολογισμός για κάθε στοιχείο των διανυσμάτων. Τα αποτελέσματα φαίνονται στον ακολουθούμενο πίνακα.
\selectlanguage{english}
\begin{spacing}{1.0}
\begin{lstlisting}[language=C++, caption={\el{Υλοποίηση παραλλαγής με \en{parallel for}}} , frame=tlrb]{Name}
void saxpy(size_t n, float a, const float *x, float *y) {
    #pragma omp parallel for
    for (size_t i = 0; i < n; ++i) {
        y[i] = a * x[i] + y[i];
    }
} 
\end{lstlisting}
\end{spacing}
\selectlanguage{greek}

\begin{table}[h]
    \centering
    \caption{Καταγραφή χρόνων εκτέλεσης}
    \label{my-label}
    \begin{tabular}{|p{0.30\textwidth}| >{\centering\arraybackslash}p{0.25\textwidth}| >{\centering\arraybackslash}p{0.25\textwidth}|}
    \hline
    \multirow{2}{*}{\textbf{Μέγεθος προβλήματος}} & \multicolumn{2}{|c|}{\textbf{Χρόνοι εκτέλεσης \en{(sec)}}} \\ \cline{2-3} 
               & \textbf{-Ο0} & \textbf{-Ο3} \\ \hline
     100000    &  0.003 & 0.004 \\ \cline{1-3} 
     1000000   &  0.009 & 0.004 \\ \cline{1-3} 
     10000000  &  0.018 & 0.014 \\ \cline{1-3} 
     100000000 &  0.133 & 0.126 \\ \cline{1-3} 
     200000000 &  0.255 & 0.244 \\ \cline{1-3} 
     300000000 &  0.380 & 0.364 \\ \cline{1-3} 
     400000000 &  0.480 & 0.460 \\ \cline{1-3} 
     500000000 &  0.489 & 0.408 \\ \cline{1-3} 


    \end{tabular}
\end{table}

\begin{figure}[h]
\begin{tabular}{*{2}{>{\centering\arraybackslash}b{\dimexpr0.5\linewidth-2\tabcolsep\relax}}}
\resizebox{0.42\textwidth}{!} {
\begin{tikzpicture}[state/.append style={minimum size=7mm}]
     \begin{axis}[
         title={Χρόνοι εκτέλεσης \en{SAXPY}},
         xlabel={Μέγεθος διανυσμάτων},
         ylabel={Χρόνος εκτέλεσης},
         xmin=100000, xmax=700000000,
         ymin=0, ymax=1.5,
         xtick={ 100000000, 200000000, 300000000, 400000000,
          500000000, 600000000, 700000000},
         ytick={ 0, 0.25, 0.5, 0.75, 1, 1.5 },
         legend pos=north west,
        % ymajorgrids=true,
        % grid style=dashed,
     ]
    
     \addplot[ color=blue, mark=square,]
      coordinates {
          (100000, 0.003804)(1000000,0.009033)(10000000,0.018875)
          (100000000,0.133915)(200000000,0.255251)(300000000,0.380445)
          (400000000, 0.480862)(500000000, 0.489234)(600000000, 0.471 )
 	};
     
     \addplot[ color=red, mark=square,]
      coordinates {
          (100000,0.004391)(1000000,0.004197)(10000000,0.014912)
          (100000000,0.126548)(200000000,0.2425)(300000000,0.364)
          (400000000,0.460)(500000000,0.408)(600000000, 0.391)(700000000, 0.4275)
 	};

	\addplot[ color=green, mark=square,]
      coordinates {
          (100000,0.00011)(1000000,0.00171)(10000000,0.01631)
          (100000000,0.16124)(200000000,0.31867)(300000000,0.42806)
          (400000000, 0.645)(500000000,0.851) (600000000, 1.055)(700000000, 1.1977)
 	};
     \legend{-O0, -O3, \en{serial} -O3}
     \end{axis}
 \end{tikzpicture}}

\caption{Σύγκριση αποτελεσμάτων}
    &
\renewcommand{\arraystretch}{1.1}
\begin{tabular}{c|c}
Μέγεθος & Επιτάχυνση (\%)  \\
\hline
100000    & -33.3 \\
1000000   &  55.5\\
10000000  &  22.22 \\
100000000 & 5.26\\
200000000 & 4.31 \\
300000000 & 4.2\\
400000000 & 4.16\\
500000000 & 16.56
\end{tabular}
\captionof{table}{Ποσοστό μείωσης με -Ο3}
\end{tabular}
\end{figure}

\subsubsection{Παρατηρήσεις}
\subparagraph{}
Με τη χρήση της οδηγίας \emph{\en{pragma omp parallel for}} επετέφχθει μείωση του χρόνου εκτέλεσης του αλγορίθμου όταν το πρόγραμμα μεταγλωττίζεται με -Ο0 σε σύγκριση με την αντίστοιχη σειριακή. Οι χρόνοι εκτέλεσης της συγκεκριμένης περίπτωσης ωστόσο, μοιάζουν με τους αντίστοιχους της μεταγλώττισης σειριακού κώδικα με -Ο3 για διανύσματα μέχρι 3\en{e}8 στοιχείων, ενώ για μεγαλύτερες τιμές η παράλληλη εκτέλεση έχει καλύτερες επιδόσεις. Τέλος, από το παραπάνω διάγραμμα δε προκύπτει κάποια διαφοροποίηση ανάμεσα στη μεταγλώττιση με -Ο0 και -Ο3.


\subsection{Παραλλαγή με \emph{\en{parallel for}} και \en{padding}}
\subparagraph{}
Σε αυτή την περίπτωση, ο αλγόριθμος παραμένει ίδιος, χρησιμοποιείται δηλαδή η οδηγία \emph{\en{pragma omp parallel for}}. Ωστόσο στη συνάρτηση εισάγονται ως ορίσματα δομές που εμπεριέχουν μία μεταβλητή αριθμού μικρής ακρίβειας και ένα τεχνητό κενό \emph{\en{padding}}. Το μέγεθος της είναι 64\en{bytes} και έχει ως στόχο την αποφυγή του φαινομένου \textbf{\en{false sharing}}.

\selectlanguage{english}
\begin{spacing}{0.9}
\begin{lstlisting}[language=C++, caption={\el{Υλοποίηση παραλλαγής με \en{parallel for}}} , frame=tlrb]{Name}
void saxpy(size_t n, float a, const float64 *x, float64 *y) {
    #pragma omp parallel for
    for (size_t i = 0; i < n; ++i) {
        y[i].val = a * x[i].val + y[i].val;
    }
} 
\end{lstlisting}
\end{spacing}
\selectlanguage{greek}

\begin{table}[h]
    \centering
    \caption{Καταγραφή χρόνων εκτέλεσης}
    \label{my-label}
    \begin{tabular}{|p{0.30\textwidth}| >{\centering\arraybackslash}p{0.25\textwidth}| >{\centering\arraybackslash}p{0.25\textwidth}|}
    \hline
    \multirow{2}{*}{\textbf{Μέγεθος προβλήματος}} & \multicolumn{2}{|c|}{\textbf{Χρόνοι εκτέλεσης \en{(sec)}}} \\ \cline{2-3} 
               & \textbf{-Ο0} & \textbf{-Ο3} \\ \hline
     100000    &  0.004 & 0.0046 \\ \cline{1-3} 
     1000000   &  0.021 & 0.0196 \\ \cline{1-3} 
     10000000  &  0.182 & 0.185  \\ \cline{1-3} 
     100000000 &  \en{killed} & \en{killed} \\ \cline{1-3} 
    \end{tabular}
\end{table}

\subsubsection{Παρατηρήσεις}
\subparagraph{}
Το πρόβλημα \emph{\en{false sharing}} δεν ήταν δυνατό να εντοπισθεί στη
συγκεκριμένη παραλλαγή. Μάλιστα, 
ο έλεγχος για μεγέθη διανυσμάτων μεγαλύτερων των 1\en{e}8 στοιχείων, ήταν ανεπιτυχής λόγω έλλειψης υπολογιστικών πόρων.
\clearpage

\subsection{Παραλλαγή με \emph{\en{omp simd}}}
\subparagraph{}
\selectlanguage{english}
\begin{spacing}{0.9}
\begin{lstlisting}[language=C++, caption={\el{Υλοποίηση παραλλαγής με \en{omp simd}}} , frame=tlrb]{Name}
void saxpy(size_t n, float a, const float *x, float *y) {
    #pragma omp simd
    for (size_t i = 0; i < n; ++i) {
        y[i] = a * x[i] + y[i];
    }
}
\end{lstlisting}
\end{spacing}
\selectlanguage{greek}

\begin{table}[h]
    \centering
    \caption{Καταγραφή χρόνων εκτέλεσης}
    \label{my-label}
    \begin{tabular}{| >{\centering\arraybackslash}p{0.15\textwidth}| 
    >{\centering\arraybackslash}p{0.1\textwidth}|
    >{\centering\arraybackslash}p{0.25\textwidth}|
    >{\centering\arraybackslash}p{0.1\textwidth}|
    >{\centering\arraybackslash}p{0.25\textwidth}|}
    \hline
    \multirow{2}{*}{\textbf{\shortstack{\\Μέγεθος \\ προβλήματος}}} & \multicolumn{4}{|c|}					{\textbf{Χρόνοι εκτέλεσης \en{(sec)}}} \\ \cline{2-5} 
        & \textbf{-Ο0}
        & \textbf{\en{\shortstack{\\-O0\\ -fno-tree-vectorize}}} 
        & \textbf{\en{-O3}}
        & \textbf{\en{\shortstack{\\-O3\\ -fno-tree-vectorize}}} 
\\ \hline
     100000    & 0.001 & 0.001 & 0.000 & 0.003 \\ \cline{1-5} 
     1000000   & 0.011 & 0.011 & 0.002 & 0.002 \\ \cline{1-5} 
     10000000  & 0.113 & 0.114 & 0.016 & 0.020 \\ \cline{1-5} 
     100000000 & 1.126 & 1.142 & 0.162 & 0.198 \\ \cline{1-5} 
     200000000 & 2.263 & 2.271 & 0.320 & 0.380 \\ \cline{1-5} 
     300000000 & 3.417 & 3.414 & 0.481 & 0.527 \\ \cline{1-5} 
     400000000 & 4.548 & 4.513 & 0.634 & 0.787 \\ \cline{1-5} 

    \end{tabular}
\end{table}

θελω να συγκρινω το vectorizatin με την σεριαλ εκδοση. Να ξαναδω τη σειριαλ, για 
να παρε αυτο με την επιλογη -fno-tree-vectorize. Να φτιαξω διαγραμματα και να κανω πειραματα.

\begin{figure}[h]
\centering 
\resizebox{0.5\textwidth}{!} {
\begin{tikzpicture}   
    \begin{axis}[
         title={Χρόνοι εκτέλεσης με \en{omp simd}},
         xlabel={Μέγεθος διανυσμάτων},
         ylabel={Χρόνος εκτέλεσης},
         xmin=100000000, xmax=400000000,
         ymin=0, ymax=5,
         xtick={ 100000000, 200000000, 300000000, 400000000},
         ytick={ 0, 0.5, 1, 1.5, 2, 2.5, 3, 3.5, 4},
         legend pos=north west,
        % ymajorgrids=true,
        % grid style=dashed,
     ]
    
     \addplot[ color=blue, mark=square,]
      coordinates {
          (100000000, 1.126)(200000000, 2.263)(300000000, 3.417)
          (400000000, 4.548)
 	};
     
     \addplot[ color=red, mark=square,]
      coordinates {
          (100000000,1.142)(200000000,2.271)(300000000,3.414)
          (400000000,4.513)
 	};

	\addplot[ color=green, mark=square,]
      coordinates {
          (100000000,0.162)(200000000,0.320)(300000000,0.481)
          (400000000, 0.634)
 	};
 	
 	\addplot[ color=black, mark=square,]
      coordinates {
          (100000000,0.198)(200000000,0.380)(300000000,0.527)
          (400000000, 0.787)
 	};
 	\legend{-O0, \en{-O0 no-vec}, -O3, \en{-O3 no-vec}}
    \end{axis}
χ`\end{tikzpicture}}% NO EMPTY LINE HERE!!!! 
\resizebox{0.5\textwidth}{!} {
\begin{tikzpicture}
    \begin{axis}[
         title={Σύγκριση σειριακής με \en{omp simd}},
         xlabel={Μέγεθος διανυσμάτων},
         ylabel={Χρόνος εκτέλεσης},
         xmin=100000000, xmax=400000000,
         ymin=0, ymax=5,
         xtick={ 100000000, 200000000, 300000000, 400000000},
         ytick={ 0, 0.5, 1, 1.5, 2, 2.5, 3, 3.5, 4},
         legend pos=north west,
        % ymajorgrids=true,
        % grid style=dashed,
     ]
    
	\addplot[ color=blue, mark=square,]
      coordinates {
          (100000000, 1.182)(200000000, 2.361)(300000000, 3.55)
          (400000000, 4.73)
 	};
 	\addlegendentry{-O0 \en{serial}}
 	
 	\addplot[ color=red, mark=square,]
      coordinates {
          (100000000,1.142)(200000000,2.271)(300000000,3.414)
          (400000000,4.513)
 	};
 	\addlegendentry{-O0 \en{no-vec}}
 	
 	\addplot[ color=green, mark=square,]
      coordinates {
          (100000000,0.2397)(200000000, 0.3981)(300000000,0.6052)
          (400000000,0.7884)
 	};
 	\addlegendentry{-O3 \en{ serial no-vec}}
 	
 	 	\addplot[ color=black, mark=square,]
      coordinates {
          (100000000,0.198)(200000000,0.380)(300000000,0.527)
          (400000000,0.787)
 	};
 	\addlegendentry{-O3 \en{simd no-vec}}


    \end{axis}
\end{tikzpicture}} 
\caption{Left: No Interaction. Right: Interaction} \label{fig:M}  
\end{figure}

\subsection{Παραλλαγή με \emph{\en{omp declare simd uniform}}}
\subparagraph{} 

\selectlanguage{english}
\begin{spacing}{0.9}
\begin{lstlisting}[language=C++, caption={\el{Υλοποίηση παραλλαγής με \en{omp declare simd uniform}}} , frame=tlrb]{Name}
#pragma omp declare simd uniform(a)
float do_work(float a, float b, float c)
{
    return a * b + c;
}

void saxpy(size_t n, float a, const float *x, float *y) {
    #pragma omp simd
    for (size_t i = 0; i < n; ++i) {
        y[i] = do_work(a, x[i], y[i]);
    }
}
\end{lstlisting}
\end{spacing}
\selectlanguage{greek}

\begin{table}[h]
    \centering
    \caption{Καταγραφή χρόνων εκτέλεσης}
    \label{my-label}
    \begin{tabular}{| >{\centering\arraybackslash}p{0.15\textwidth}| 
    >{\centering\arraybackslash}p{0.1\textwidth}|
    >{\centering\arraybackslash}p{0.25\textwidth}|
    >{\centering\arraybackslash}p{0.1\textwidth}|
    >{\centering\arraybackslash}p{0.25\textwidth}|}
    \hline
    \multirow{2}{*}{\textbf{\shortstack{\\Μέγεθος \\ προβλήματος}}} & \multicolumn{4}{|c|}					{\textbf{Χρόνοι εκτέλεσης \en{(sec)}}} \\ \cline{2-5} 
        & \textbf{-Ο0}
        & \textbf{\en{\shortstack{\\-O0\\ -fno-tree-vectorize}}} 
        & \textbf{\en{-O3}}
        & \textbf{\en{\shortstack{\\-O3\\ -fno-tree-vectorize}}} 
\\ \hline
     100000    & 0.001 & 0.001 & 0.001 & 0.0003\\ \cline{1-5} 
     1000000   & 0.017 & 0.014 & 0.002 & 0.002 \\ \cline{1-5} 
     10000000  & 0.143 & 0.139 & 0.017 & 0.022 \\ \cline{1-5} 
     100000000 & 1.424 & 1.421 & 0.148 & 0.193 \\ \cline{1-5} 
     200000000 & 2.792 & 2.78  & 0.294 & 0.422 \\ \cline{1-5} 
     300000000 & 4.206 & 4.174 & 0.459 & 0.630 \\ \cline{1-5} 
     400000000 & 5.705 & 5.59  & 0.651 & 0.833 \\ \cline{1-5} 
    \end{tabular}
\end{table}

\en{TODO} Μπορω να βαλω 4 διαγραμματα που να συγκρινουν το προηγουμενο με αυτο. για να δειξω οτι η πιθανη καθυστερηση σε αυτο ειναι το overhead απο το function call.
\clearpage
\subsection{Παραλλαγή με \emph{\en{omp declare simd uniform notinbranch}}}
\subparagraph{} 
\selectlanguage{english}
\begin{spacing}{0.9}
\begin{lstlisting}[language=C++, caption={\el{Υλοποίηση παραλλαγής με \en{omp declare simd uniform
 notinbranch}}} , frame=tlrb]{Name}
#pragma omp declare simd uniform(a) notinbranch
float do_work(float a, float b, float c)
{
    return a * b + c;
}

void saxpy(size_t n, float a, const float *x, float *y) {
    #pragma omp simd
    for (size_t i = 0; i < n; ++i) {
        y[i] = a * x[i] + y[i]; //TODO
    }
}
\end{lstlisting}
\end{spacing}
\selectlanguage{greek}

\begin{table}[h]
    \centering
    \caption{Καταγραφή χρόνων εκτέλεσης}
    \label{my-label}
    \begin{tabular}{| >{\centering\arraybackslash}p{0.15\textwidth}| 
    >{\centering\arraybackslash}p{0.1\textwidth}|
    >{\centering\arraybackslash}p{0.25\textwidth}|
    >{\centering\arraybackslash}p{0.1\textwidth}|
    >{\centering\arraybackslash}p{0.25\textwidth}|}
    \hline
    \multirow{2}{*}{\textbf{\shortstack{\\Μέγεθος \\ προβλήματος}}} & \multicolumn{4}{|c|}					{\textbf{Χρόνοι εκτέλεσης \en{(sec)}}} \\ \cline{2-5} 
        & \textbf{-Ο0}
        & \textbf{\en{\shortstack{\\-O0\\ -fno-tree-vectorize}}} 
        & \textbf{\en{-O3}}
        & \textbf{\en{\shortstack{\\-O3\\ -fno-tree-vectorize}}} 
\\ \hline
     100000    & 0.001 & 0.001 & 0.001 & 0.001 \\ \cline{1-5} 
     1000000   & 0.012 & 0.011 & 0.002 & 0.002 \\ \cline{1-5} 
     10000000  & 0.115 & 0.115 & 0.016 & 0.021 \\ \cline{1-5} 
     100000000 & 1.154 & 1.145 & 0.161 & 0.212 \\ \cline{1-5} 
     200000000 & 2.308 & 2.299 & 0.322 & 0.378 \\ \cline{1-5} 
     300000000 & 3.473 & 3.446 & 0.479 & 0.633 \\ \cline{1-5} 
     400000000 & 4.621 & 4.604 & 0.632 & 0.827 \\ \cline{1-5} 
    \end{tabular}
\end{table}

Να αναφερω οτι δε βλέπω κάποια διαφορα σε σχέση με το προηγούμενο παραλλαγή.
\clearpage


\subsection{Παραλλαγή με \emph{\en{omp parallel for simd}}}
\subparagraph{} 
\selectlanguage{english}
\begin{spacing}{0.9}
\begin{lstlisting}[language=C++, caption={\el{Υλοποίηση παραλλαγής με \en{omp parallel for simd}}} , frame=tlrb]{Name}
void saxpy(size_t n, float a, const float *x, float *y) {
    #pragma omp parallel for simd
    for (size_t i = 0; i < n; ++i) {
        y[i] = a * x[i] + y[i];
    }
}
\end{lstlisting}
\end{spacing}
\selectlanguage{greek}

\begin{table}[h]
    \centering
    \caption{Καταγραφή χρόνων εκτέλεσης}
    \label{my-label}
    \begin{tabular}{| >{\centering\arraybackslash}p{0.15\textwidth}| 
    >{\centering\arraybackslash}p{0.1\textwidth}|
    >{\centering\arraybackslash}p{0.25\textwidth}|
    >{\centering\arraybackslash}p{0.1\textwidth}|
    >{\centering\arraybackslash}p{0.25\textwidth}|}
    \hline
    \multirow{2}{*}{\textbf{\shortstack{\\Μέγεθος \\ προβλήματος}}} & \multicolumn{4}{|c|}					{\textbf{Χρόνοι εκτέλεσης \en{(sec)}}} \\ \cline{2-5} 
        & \textbf{-Ο0}
        & \textbf{\en{\shortstack{\\-O0\\ -fno-tree-vectorize}}} 
        & \textbf{\en{-O3}}
        & \textbf{\en{\shortstack{\\-O3\\ -fno-tree-vectorize}}} 
\\ \hline
     100000    & 0.004 & 0.005 & 0.002 & 0.005 \\ \cline{1-5} 
     1000000   & 0.007 & 0.007 & 0.001 & 0.004 \\ \cline{1-5} 
     10000000  & 0.019 & 0.019 & 0.017 & 0.014 \\ \cline{1-5} 
     100000000 & 0.127 & 0.125 & 0.125 & 0.124 \\ \cline{1-5} 
     200000000 & 0.256 & 0.249 & 0.247 & 0.250 \\ \cline{1-5} 
     300000000 & 0.372 & 0.383 & 0.369 & 0.365 \\ \cline{1-5} 
     400000000 & 0.494 & 0.525 & 0.482 & 0.485 \\ \cline{1-5} 
    \end{tabular}
\end{table}

να αναφέρω οτι δεν παιζει ρολο το -O3 optimization :(
Επίσης να συγκρίνω με την parallel for χωρις simd!
\clearpage
\subsection{Παραλλαγή με \emph{\en{target map}}}
\subparagraph{}
\selectlanguage{english}
\begin{spacing}{0.9}
\begin{lstlisting}[language=C++, caption={\el{Υλοποίηση παραλλαγής με \en{target map}}} , frame=tlrb]{Name}
void saxpy(size_t n, float a, const float *x, float *y) {
    #pragma omp target map(tofrom: y[0:n]) map(to: x[0:n])
    for (size_t i = 0; i < n; ++i) {
        y[i] = a * x[i] + y[i];
    }
}
\end{lstlisting}
\end{spacing}
\selectlanguage{greek}

\begin{table}[h]
    \centering
    \caption{Καταγραφή χρόνων εκτέλεσης}
    \label{my-label}
    \begin{tabular}{| >{\centering\arraybackslash}p{0.25\textwidth}| 
    >{\centering\arraybackslash}p{0.25\textwidth}|
    >{\centering\arraybackslash}p{0.25\textwidth}|}
    \hline
    \multirow{2}{*}{\textbf{\shortstack{\\Μέγεθος \\ προβλήματος}}} & \multicolumn{2}{|c|}					{\textbf{Χρόνοι εκτέλεσης \en{(sec)}}} \\ \cline{2-3} 
        & \textbf{-Ο0}
        & \textbf{-O3} 

\\ \hline
     100000    & 0.005 & 0.004 \\ \cline{1-3} 
     1000000   & 0.016 & 0.006 \\ \cline{1-3} 
     10000000  & 0.123 & 0.019 \\ \cline{1-3} 
     100000000 & 1.187 & 0.269 \\ \cline{1-3} 
     200000000 & 2.381 & 0.334 \\ \cline{1-3} 
     300000000 & 3.555 & 0.512 \\ \cline{1-3} 
     400000000 & 4.731 & 0.622 \\ \cline{1-3} 
    \end{tabular}
\end{table}

Συγκριση με σειριακή.

\clearpage
\subsection{Παραλλαγή με \emph{\en{target simd map}}}
\subparagraph{}
\selectlanguage{english}
\begin{spacing}{0.9}
\begin{lstlisting}[language=C++, caption={\el{Υλοποίηση παραλλαγής με \en{target simd map}}} , frame=tlrb]{Name}
void saxpy(size_t n, float a, const float *x, float *y) {
    #pragma omp target simd map(tofrom: y[0:n]) map(to: x[0:n])
    for (size_t i = 0; i < n; ++i) {
        y[i] = a * x[i] + y[i];
    }
}

\end{lstlisting}
\end{spacing}
\selectlanguage{greek}

\begin{table}[h]
    \centering
    \caption{Καταγραφή χρόνων εκτέλεσης}
    \label{my-label}
    \begin{tabular}{| >{\centering\arraybackslash}p{0.15\textwidth}| 
    >{\centering\arraybackslash}p{0.15\textwidth}|
	>{\centering\arraybackslash}p{0.20\textwidth}|
	>{\centering\arraybackslash}p{0.15\textwidth}|
    >{\centering\arraybackslash}p{0.20\textwidth}|}
    \hline
    \multirow{4}{*}{\textbf{\shortstack{\\Μέγεθος \\ προβλήματος}}} & \multicolumn{2}{|c|}					{\textbf{Χρόνοι εκτέλεσης \en{(sec)}}} \\ \cline{2-3} 
        & \textbf{-Ο0}
        & \textbf{\en{-O0 -fopenmp-simd}}
        & \textbf{-O3} 
        & \textbf{\en{-O3 -fopenmp-simd}}

\\ \hline
     100000    & 0.005 & 0.005 & 0.004 & 0.004 \\ \cline{1-5} 
     1000000   & 0.015 & 0.016 & 0.006 & 0.006\\ \cline{1-5} 
     10000000  & 0.120 & 0.120 & 0.020 & 0.021\\ \cline{1-5} 
     100000000 & 1.154 & 1.161 & 0.157 & 0.167\\ \cline{1-5} 
     200000000 & 2.298 & 2.314 & 0.332 & 0.333\\ \cline{1-5} 
     300000000 & 3.485 & 3.491 & 0.486 & 0.490\\ \cline{1-5} 
     400000000 & 4.620 & 4.613 & 0.650 & 0.654\\ \cline{1-5} 
    \end{tabular}
\end{table}

Συγκριση με σιμδ.
\clearpage
\subsection{Παραλλαγή με \emph{\en{target parallel for}}}
\subparagraph{}
\selectlanguage{english}
\begin{spacing}{0.9}
\begin{lstlisting}[language=C++, caption={\el{Υλοποίηση παραλλαγής με \en{target parallel for}}} , frame=tlrb]{Name}
void saxpy(size_t n, float a, const float *x, float *y) {
#pragma omp target parallel for map(tofrom: y[0:n]) map(to: x[0:n])
    for (size_t i = 0; i < n; ++i) {
        y[i] = a * x[i] + y[i];
    }
}
\end{lstlisting}
\end{spacing}
\selectlanguage{greek}
\begin{table}[h]
    \centering
    \caption{Καταγραφή χρόνων εκτέλεσης}
    \label{my-label}
    \begin{tabular}{| >{\centering\arraybackslash}p{0.25\textwidth}| 
    >{\centering\arraybackslash}p{0.25\textwidth}|
    >{\centering\arraybackslash}p{0.25\textwidth}|}
    \hline
    \multirow{2}{*}{\textbf{\shortstack{\\Μέγεθος \\ προβλήματος}}} & \multicolumn{2}{|c|}					{\textbf{Χρόνοι εκτέλεσης \en{(sec)}}} \\ \cline{2-3} 
        & \textbf{-Ο0}
        & \textbf{-O3} 

\\ \hline
     100000    & 0.011 & 0.011 \\ \cline{1-3} 
     1000000   & 0.009 & 0.013 \\ \cline{1-3} 
     10000000  & 0.035 & 0.021 \\ \cline{1-3} 
     100000000 & 0.150 & 0.127 \\ \cline{1-3} 
     200000000 & 0.264 & 0.251 \\ \cline{1-3} 
     300000000 & 0.387 & 0.369 \\ \cline{1-3} 
     400000000 & 0.500 & 0.490 \\ \cline{1-3} 
    \end{tabular}
\end{table}
\clearpage
\subsection{Παραλλαγή με \emph{\en{target parallel for simd}}}
\subparagraph{}
\selectlanguage{english}
\begin{spacing}{0.9}
\begin{lstlisting}[basicstyle=\small, language=C++, caption={\el{Υλοποίηση παραλλαγής με \en{target parallel for simd}}} , frame=tlrb]{Name}
void saxpy(size_t n, float a, const float *x, float *y) {
#pragma omp target parallel for simd map(tofrom: y[0:n]) map(to: x[0:n])
    for (size_t i = 0; i < n; ++i) {
        y[i] = a * x[i] + y[i];
    }
}
\end{lstlisting}
\end{spacing}
\selectlanguage{greek}

\begin{table}[h]
    \centering
    \caption{Καταγραφή χρόνων εκτέλεσης}
    \label{my-label}
    \begin{tabular}{| >{\centering\arraybackslash}p{0.15\textwidth}| 
    >{\centering\arraybackslash}p{0.15\textwidth}|
	>{\centering\arraybackslash}p{0.20\textwidth}|
	>{\centering\arraybackslash}p{0.15\textwidth}|
    >{\centering\arraybackslash}p{0.20\textwidth}|}
    \hline
    \multirow{4}{*}{\textbf{\shortstack{\\Μέγεθος \\ προβλήματος}}} & \multicolumn{2}{|c|}					{\textbf{Χρόνοι εκτέλεσης \en{(sec)}}} \\ \cline{2-3} 
        & \textbf{-Ο0}
        & \textbf{\en{-O0 -fopenmp-simd}}
        & \textbf{-O3} 
        & \textbf{\en{-O3 -fopenmp-simd}}

\\ \hline
     100000    & 0.010 & 0.010 & 0.011 & 0.011 \\ \cline{1-5} 
     1000000   & 0.012 & 0.014 & 0.011 & 0.009 \\ \cline{1-5} 
     10000000  & 0.027 & 0.026 & 0.020 & 0.020 \\ \cline{1-5} 
     100000000 & 0.140 & 0.154 & 0.128 & 0.130 \\ \cline{1-5} 
     200000000 & 0.257 & 0.271 & 0.249 & 0.247 \\ \cline{1-5} 
     300000000 & 0.378 & 0.385 & 0.370 & 0.365 \\ \cline{1-5} 
     400000000 & 0.513 & 0.505 & 0.450 & 0.489 \\ \cline{1-5} 
    \end{tabular}
\end{table}

\clearpage
\subsection{Παραλλαγή με \emph{\en{target teams map}}}
\subparagraph{}
\selectlanguage{english}
\begin{spacing}{0.9}
\begin{lstlisting}[language=C++, caption={\el{Υλοποίηση παραλλαγής με \en{target teams map}}} , frame=tlrb]{Name}
void saxpy(size_t n, float a, const float *x, float *y) {
    #pragma omp target teams map(tofrom: y[0:n]) map(to: x[0:n])
    for (size_t i = 0; i < n; ++i) {
        y[i] = a * x[i] + y[i];
    }
}
\end{lstlisting}
\end{spacing}
\selectlanguage{greek}
\begin{table}[h]
    \centering
    \caption{Καταγραφή χρόνων εκτέλεσης}
    \label{my-label}
    \begin{tabular}{| >{\centering\arraybackslash}p{0.25\textwidth}| 
    >{\centering\arraybackslash}p{0.25\textwidth}|
    >{\centering\arraybackslash}p{0.25\textwidth}|}
    \hline
    \multirow{2}{*}{\textbf{\shortstack{\\Μέγεθος \\ προβλήματος}}} & \multicolumn{2}{|c|}					{\textbf{Χρόνοι εκτέλεσης \en{(sec)}}} \\ \cline{2-3} 
        & \textbf{-Ο0}
        & \textbf{-O3} 

\\ \hline
     100000    & 0.005 & 0.004  \\ \cline{1-3} 
     1000000   & 0.016 & 0.006 \\ \cline{1-3} 
     10000000  & 0.132 & 0.025 \\ \cline{1-3} 
     100000000 & 1.186 & 0.221 \\ \cline{1-3} 
     200000000 & 2.374 & 0.420 \\ \cline{1-3} 
     300000000 & 3.543 & 0.652 \\ \cline{1-3} 
     400000000 & 4.729 & 0.821 \\ \cline{1-3} 
    \end{tabular}
\end{table}

\clearpage
\subsection{Παραλλαγή με \emph{\en{target teams distribute map}}}
\subparagraph{}
\selectlanguage{english}
\begin{spacing}{0.9}
\begin{lstlisting}[language=C++, caption={\el{Υλοποίηση παραλλαγής με \en{target teams distribute map}}} , frame=tlrb]{Name}
void saxpy(size_t n, float a, const float *x, float *y) {
#pragma omp target teams distribute map(from: y[0:n]) map(to: x[0:n])
    for (size_t i = 0; i < n; ++i) {
        y[i] = a * x[i] + y[i];
    }
}

\end{lstlisting}
\end{spacing}
\selectlanguage{greek}
\begin{table}[h]
    \centering
    \caption{Καταγραφή χρόνων εκτέλεσης}
    \label{my-label}
    \begin{tabular}{| >{\centering\arraybackslash}p{0.25\textwidth}| 
    >{\centering\arraybackslash}p{0.25\textwidth}|
    >{\centering\arraybackslash}p{0.25\textwidth}|}
    \hline
    \multirow{2}{*}{\textbf{\shortstack{\\Μέγεθος \\ προβλήματος}}} & \multicolumn{2}{|c|}					{\textbf{Χρόνοι εκτέλεσης \en{(sec)}}} \\ \cline{2-3} 
        & \textbf{-Ο0}
        & \textbf{-O3} 

\\ \hline
     100000    & 0.005 & 0.004 \\ \cline{1-3} 
     1000000   & 0.015 & 0.006 \\ \cline{1-3} 
     10000000  & 0.117 & 0.025 \\ \cline{1-3} 
     100000000 & 1.142 & 0.214 \\ \cline{1-3} 
     200000000 & 2.278 & 0.428 \\ \cline{1-3} 
     300000000 & 3.433 & 0.625 \\ \cline{1-3} 
     400000000 & 4.575 & 0.851 \\ \cline{1-3} 
    \end{tabular}
\end{table}

\clearpage
\subsection{Παραλλαγή με \emph{\en{target teams distribute parallel for map}}}
\subparagraph{}
\selectlanguage{english}
\begin{spacing}{0.9}
\begin{lstlisting}[basicstyle=\footnotesize, language=C++, caption={\el{Υλοποίηση παραλλαγής με \en{target teams distribute parallel for}}} , frame=tlrb]{Name}
void saxpy(size_t n, float a, const float *x, float *y) {
#pragma omp target teams distribute parallel for map(from: y[0:n]) map(to: x[0:n])
    for (size_t i = 0; i < n; ++i) {
        y[i] = a * x[i] + y[i];
    }
}
\end{lstlisting}
\end{spacing}
\selectlanguage{greek}
\begin{table}[h]
    \centering
    \caption{Καταγραφή χρόνων εκτέλεσης}
    \label{my-label}
    \begin{tabular}{| >{\centering\arraybackslash}p{0.25\textwidth}| 
    >{\centering\arraybackslash}p{0.25\textwidth}|
    >{\centering\arraybackslash}p{0.25\textwidth}|}
    \hline
    \multirow{2}{*}{\textbf{\shortstack{\\Μέγεθος \\ προβλήματος}}} & \multicolumn{2}{|c|}					{\textbf{Χρόνοι εκτέλεσης \en{(sec)}}} \\ \cline{2-3} 
        & \textbf{-Ο0}
        & \textbf{-O3} 

\\ \hline
     100000    & 0.010 & 0.011 \\ \cline{1-3} 
     1000000   & 0.016 & 0.011 \\ \cline{1-3} 
     10000000  & 0.023 & 0.020 \\ \cline{1-3} 
     100000000 & 0.139 & 0.127 \\ \cline{1-3} 
     200000000 & 0.257 & 0.250 \\ \cline{1-3} 
     300000000 & 0.389 & 0.369 \\ \cline{1-3} 
     400000000 & 0.511 & 0.490 \\ \cline{1-3} 
    \end{tabular}
\end{table}
\clearpage
\subsection{Παραλλαγή με \emph{\en{target teams distribute simd map}}}
\subparagraph{}
\selectlanguage{english}
\begin{spacing}{0.9}
\begin{lstlisting}[basicstyle=\small, language=C++, caption={\el{Υλοποίηση παραλλαγής με \en{target teams distribute simd map}}} , frame=tlrb]{Name}
void saxpy(size_t n, float a, const float *x, float *y) {
#pragma omp target teams distribute simd map(from: y[0:n]) map(to: x[0:n])
    for (size_t i = 0; i < n; ++i) {
        y[i] = a * x[i] + y[i];
    }
}

\end{lstlisting}
\end{spacing}
\selectlanguage{greek}

\begin{table}[h]
    \centering
    \caption{Καταγραφή χρόνων εκτέλεσης}
    \label{my-label}
    \begin{tabular}{| >{\centering\arraybackslash}p{0.15\textwidth}| 
    >{\centering\arraybackslash}p{0.15\textwidth}|
	>{\centering\arraybackslash}p{0.20\textwidth}|
	>{\centering\arraybackslash}p{0.15\textwidth}|
    >{\centering\arraybackslash}p{0.20\textwidth}|}
    \hline
    \multirow{4}{*}{\textbf{\shortstack{\\Μέγεθος \\ προβλήματος}}} & \multicolumn{2}{|c|}					{\textbf{Χρόνοι εκτέλεσης \en{(sec)}}} \\ \cline{2-3} 
        & \textbf{-Ο0}
        & \textbf{\en{-O0 -fopenmp-simd}}
        & \textbf{-O3} 
        & \textbf{\en{-O3 -fopenmp-simd}}

\\ \hline
     100000    & 0.005 & 0.050 & 0.004 & 0.004 \\ \cline{1-5} 
     1000000   & 0.015 & 0.015 & 0.006 & 0.006 \\ \cline{1-5} 
     10000000  & 0.120 & 0.119 & 0.021 & 0.021 \\ \cline{1-5} 
     100000000 & 1.159 & 1.153 & 0.159 & 0.168 \\ \cline{1-5} 
     200000000 & 2.311 & 2.305 & 0.326 & 0.327 \\ \cline{1-5} 
     300000000 & 3.487 & 3.472 & 0.482 & 0.495 \\ \cline{1-5} 
     400000000 & 4.620 & 4.610 & 0.647 & 0.644 \\ \cline{1-5} 
    \end{tabular}
\end{table}

\clearpage
\subsection{Παραλλαγή με \emph{\en{target teams distribute parallel for simd map}}}
\subparagraph{}
\selectlanguage{english}
\begin{spacing}{0.9}
\begin{lstlisting}[basicstyle=\footnotesize, language=C++, caption={\el{Υλοποίηση παραλλαγής με \en{teams distribute parallel for simd\
			map}}} , frame=tlrb]{Name}
void saxpy(size_t n, float a, const float *x, float *y) {
#pragma omp target teams distribute parallel for simd\
			map(from: y[0:n]) map(to: x[0:n])
    for (size_t i = 0; i < n; ++i) {
        y[i] = a * x[i] + y[i];
    }
}
\end{lstlisting}
\end{spacing}
\selectlanguage{greek}

\begin{table}[h]
    \centering
    \caption{Καταγραφή χρόνων εκτέλεσης}
    \label{my-label}
    \begin{tabular}{| >{\centering\arraybackslash}p{0.15\textwidth}| 
    >{\centering\arraybackslash}p{0.15\textwidth}|
	>{\centering\arraybackslash}p{0.20\textwidth}|}
    \hline
    \multirow{2}{*}{\textbf{\shortstack{\\Μέγεθος \\ προβλήματος}}} & \multicolumn{2}{|c|}					{\textbf{Χρόνοι εκτέλεσης \en{(sec)}}} \\ \cline{2-3} 
        & \textbf{-Ο0}
        & \textbf{-O3} 

\\ \hline
     100000    & 0.010 & 0.010 \\ \cline{1-3} 
     1000000   & 0.012 & 0.010 \\ \cline{1-3} 
     10000000  & 0.025 & 0.021 \\ \cline{1-3} 
     100000000 & 0.146 & 0.127 \\ \cline{1-3} 
     200000000 & 0.263 & 0.246 \\ \cline{1-3} 
     300000000 & 0.389 & 0.371 \\ \cline{1-3} 
     400000000 & 0.519 & 0.489 \\ \cline{1-3} 
    \end{tabular}
\end{table}