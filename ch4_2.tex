\subsection{Πρόσθεση διανυσμάτων αριθμών μικρής ακρίβειας - \emph{\en{SAXPY}}}
\subparagraph{}
Μια από τις λειτουργίες που κατέχει θεμελιώδη θέση σε εφαρμογές της γραμμικής άλγεβρας, αποτελεί η πρόσθεση διανυσμάτων
δεκαδικών αριθμών μικρής ακρίβειας (\emph{\en{floats}}), γνωστή ως \textbf{\en{SAXPY}}.

Στο παράδειγμα \emph{\en{SAXPY}}, ο λόγος του μεγέθους υπολογισμών προς το μέγεθος των δεδομένων που τελούν υπό επεξεργασία
είναι μικρός. Ως εκτότου, αποτελεί πρόβλημα περιορισμένης επεκτασιμότητας. Παρόλα αυτά, πρόκειται για ένα χρήσιμο
παράδειγμα που ανήκει στην κατηγορία προβλημάτων παραλληλοποίησης τύπου \emph{\en{map}} και των εννοιών
\emph{\en{uniform}} και \emph{\en{varying parameters}}\cite{patters}.
\\
\subsubsection{Περιγραφή προβλήματος}
Η λειτουργία \en{SAXPY} δέχεται ως δεδομένα δυο διάνυσματα δεκαδικών αριθμών. Το πρώτο διάνυσμα πολλαπλασιάζεται με μια
σταθερά \emph{\en{a}} και το αποτέλεσμα προστίθεται στο δεύτερο διάνυσμα \emph{\en{y}}. Τα διανύσματα \emph{\en{x,y}}
πρέπει να έχουν το ίδιο μέγεθος.

Ο υπολογισμός αυτός εμφανίζεται συχνά στη γραμμική άλγεβρα, όπως για παράδειγμα στη διαγραφή σειρών για την απαλοιφή
\textbf{\en{Gauss}}. Το όνομα \emph{\en{SAXPY}} δόθηκε από την βιβλιοθήκη \en{\textbf{BLAS} (\emph{"Basic Linear Algrbra
Subprograms"})} για δεκαδικούς αρθμούς μικρής ακρίβειας (\emph{\en{floats}}). Ο αντίστοιχος αλγόριθμος διπλής ακρίβειας
ονομάζεται \emph{\en{DAXPY}}, ενώ για μιγαδικούς αριθμούς ονομάζεται \emph{\en{CAXPY}}.
Η μαθηματική διατύπωση του \emph{\en{SAXPY}} είναι:
                              $$\en{\textbf{y} = a*\textbf{x} + \textbf{y}}$$ όπου το διάνυσμα \emph{\en{x}}
χρησιμοποιείται ως είσοδος, το \en{\emph{y}} ως είσοδος και έξοδος. Δηλαδή το αρχικό διάνυσμα \emph{\en{y}}
τροποποιείται. Εναλλακτικά, η λειτουργία \emph{\en{SAXPY}} μπορεί να περιγραφεί ως συνάρτηση που δρα σε μεμονωμένα
στοιχεία, οπως φαίνεται παρακάτω: 
\begin{equation*}\label{eq:pareto mle2}
\begin{aligned}
f(t, p, q) = tp + q\\
\forall _i : y_i \leftarrow f(a, x_i, y_i)
\end{aligned}
\end{equation*}
\clearpage
Οι συναρτήσεις τύπου \emph{\en{f}} δεχονται ως ορίσματα, δύο είδη παραμέτρων. Τις παραμέτρους όπως την \en{\emph{a}} που
παραμένουν σταθερές και ονομάζονται \emph{\en{uniform}}, οι παράμετροι που είναι μεταβλητές σε κάθε κλίση της
\emph{\en{f}} ονομάζονται \emph{\en{varying}}. To μοτίβο \emph{\en{map}} καλεί τη συνάρτηση \emph{\en{f}} τόσες φορές
όσες και ο αριθμός των στοιχείων του διανύσματος.\cite{patters}.

\subsection{Περιγραφή κεντρικού τμήματος προβλήματος \en{SAXPY}}
\subparagraph{}
Το πρόβλημα ξεκινάει δημιουργώντας ένα στοιχείο τύπου \emph{\en{Containers}}, που περιέχει τα διανύσματα που εισάγονται
στον αλγόριθμο \en{SAXPY}. Ο ρόλος του \en{Containers} είναι για την διαχείριση της \en{\emph{heap}} μνήμης. Τα
διανύσματα και η σταθερά \en{cons} αρχικοποιούνται με τυχαίους αριθμούς μικρής ακρίβειας. Το μέγεθος των
διανυσμάτων (μέγεθος προβλήματος) ορίζεται από τον χρήστη μέσω της γραμμής εντολών. Στη συνέχεια, καλείται ο αλγόριθμος
\en{SAXPY} μόλις τελειώσει γίνεται επαλήθευση των αποτελεσμάτων, όπου αν επαληθευτούν σωστά, γίνεται εκτύπωση του
χρόνουν εκτέλεσης της παραλλαγής.
\\
\selectlanguage{english}
\begin{spacing}{1.0}
\begin{lstlisting}[language=C++, caption={\el{Κεντρικός κώδικας προβλήματος \en{SAXPY}}} , frame=tlrb]{Name}
int main(int argc, char **argv) {
    Opts o;
    parseArgs(argc, argv, o);
    Containers c(o.size);
    c.setRandomValues();
    float cons = float(rand()) / float(RAND_MAX);
    auto start = omp_get_wtime();
    saxpy(c.m_size, cons, c.m_a, c.m_b);
    auto end = omp_get_wtime();
    verify(c.m_size, cons, c.m_a, c.m_b, c.m_verification);
    std::cout << "Execution Time : " << std::fixed
         << end - start << std::setprecision(5);
    std::cout << " sec " << std::endl;
    return 0;
}
   
\end{lstlisting}
\end{spacing}
\selectlanguage{greek}
\clearpage
\selectlanguage{english}
\begin{spacing}{0.8}
\begin{lstlisting}[language=C++, caption={\el{Κλάση \emph{\en{Containers}}}} , frame=tlrb]{Name}
struct Containers {
    explicit Containers(size_t containers_size);
    ~Containers();

    size_t m_size;
    float *m_a;
    float *m_verification;
    float *m_b;
};     

Containers::Containers(size_t containers_size)
    : m_size(containers_size) {
    srand(time(nullptr));
    m_a = new float[containers_size];
    m_verification = new float[containers_size];
    m_b = new float[containers_size];
}

Containers::~Containers() {
    delete []m_a;
    delete []m_b;
    delete []m_verification;
}

Containers::setRandomValues() {
    fill_random_arr(m_a, m_size);
    fill_random_arr(m_b, m_size);
}
\end{lstlisting}
\end{spacing}
\selectlanguage{greek}

\selectlanguage{english}
\begin{spacing}{1}
\begin{lstlisting}[language=C++, caption={\el{Συνάρτηση επαλήθευσης}} , frame=tlrb]{Name}
static void verify(size_t size, float c, float *a, float *b, 
                    float *verification) {
    for (size_t i = 0; i < size; ++i) {
        if (abs(c * a[i] + verification[i] - b[i]) >= 10e-6) {
            std::cout << "Failed index: " << i <<
             ". " << c * a[i] + verification[i] << 
             " =! " << b[i] << std::endl;
            exit(1);
        }
    }
}
\end{lstlisting}
\end{spacing}
\selectlanguage{greek}

\selectlanguage{english}
\begin{spacing}{1}
\begin{lstlisting}[language=C++, caption={\el{Συνάρτηση αρχικοποίησης τιμών}} , frame=tlrb]{Name}
static void fill_random_arr(float *arr, size_t size) {
    for (size_t k = 0; k < size; ++k) {
        arr[k] = (float)(rand()) / RAND_MAX;
    }
}       
\end{lstlisting}
\end{spacing}
\selectlanguage{greek}

\clearpage
\subsection{Σειριακή εκτέλεση}
\subparagraph{}
Η υλοποίηση της σειριακής παραλλαγής της συνάρτησης \en{saxpy} περιλαμβάνει έναν επαναληπτικό 
βρόγχο στον οποίο γίνεται ο υπολογισμός για κάθε στοιχείο των διανυσμάτων.
\selectlanguage{english}
\begin{spacing}{0.8}
\begin{lstlisting}[language=C++, caption={\el{Σειριακή υλοποίηση της \en{SAXPY}}} , frame=tlrb]{Name}
void saxpy(size_t n, float a, const float *x, float *y) {
    for (size_t i = 0; i < n; ++i) {
        y[i] = a * x[i] + y[i];
    }
}   
\end{lstlisting}
\end{spacing}
\selectlanguage{greek}

Η μεταγλώττιση έγινε με τους παρακάτω τρόπους και οι χρόνοι εκτέλεσης καταγράφονται στους πίνακες που ακολουθούν. Ο δεύτερος τρόπος μεταγλώττισης περιλαμβάνει διανυσματικοποίηση από τον μεταγλωττιστή ενώ η πρώτη όχι.
\begin{table}[h]
    \centering
    \caption{Επιλογές μεταγλώττισης}
    \label{my-label}
    \begin{tabular}{
    |p{0.1\textwidth}
    | >{\centering\arraybackslash}p{0.9\textwidth}
    |}
    \hline
 {\textbf{\en{Label}}} & \textbf{\en{Options}} \\ \hline
     \textbf{\en{Alt1}} & \en{ -fopt-info-vec=info.log -fno-inline -fno-tree-vectorize -fopenmp -Wall  -Wextra -std=c++14 -O2} \\ \hline
     \textbf{\en{Alt2}} & \en{ -fopt-info-vec=info.log -fno-inline -ftree-vectorize -fopenmp -Wall  -Wextra -std=c++14 -O2} \\ \hline
          \textbf{\en{Alt3}} & \en{ -fopt-info-vec=info.log -fno-inline -fopenmp -Wall  -Wextra -std=c++14 -O2} \\ \hline
    \end{tabular}
\end{table}

\begin{table}[h]
    \centering
    \caption{Καταγραφή χρόνων εκτέλεσης}
    \label{my-label}
    \resizebox{0.7\textwidth}{!} {
    \begin{tabular}{|p{0.30\textwidth}
    | >{\centering\arraybackslash}p{0.12\textwidth}
    | >{\centering\arraybackslash}p{0.12\textwidth}
    | >{\centering\arraybackslash}p{0.12\textwidth}
    |}
    \hline
    \multirow{2}{*}{\textbf{Μέγεθος προβλήματος}} & \multicolumn{3}{|c|}{\textbf{Χρόνοι εκτέλεσης \en{(sec)}}} \\ \cline{2-4} 
               & \textbf{\en{Alt1}} & \textbf{\en{Alt2}} & \textbf{\en{Alt3}} \\ \hline
     100000    & 0.0004 & 0.0001 & 0.0001\\ \cline{1-4} 
     1000000   & 0.002  & 0.002  & 0.002\\ \cline{1-4} 
     10000000  & 0.021  & 0.016  & 0.020\\ \cline{1-4} 
     100000000 & 0.200  & 0.167  & 0.199 \\ \cline{1-4} 
     200000000 & 0.398  & 0.334  & 0.397\\ \cline{1-4} 
     300000000 & 0.583  & 0.502  & 0.599\\ \cline{1-4} 
     400000000 & 0.773  & 0.617  & 0.788\\ \cline{1-4} 
     500000000 & 0.989  & 0.818  & 0.991\\ \cline{1-4} 

    \end{tabular}}
\end{table}
\clearpage
\begin{figure}[h]
\begin{tabular}{*{2}{>{\centering\arraybackslash}b{\dimexpr0.5\linewidth-2\tabcolsep\relax}}}
\resizebox{0.5\textwidth}{!} {
\begin{tikzpicture}[state/.append style={minimum size=7mm}]
     \begin{axis}[
         title={Χρόνοι εκτέλεσης \en{Alt1 - Alt2 - Alt3}},
         xlabel={Μέγεθος διανυσμάτων},
         ylabel={Χρόνος εκτέλεσης},
         xmin=100000, xmax=400000000,
         ymin=0, ymax=1,
         xtick={ 100000000, 200000000, 300000000, 400000000},
         ytick={ 0, 0.25, 0.5, 0.75, 1 },
         legend pos=north west,
        % ymajorgrids=true,
        % grid style=dashed,
     ]
    
     \addplot[ color=blue, mark=square,]
      coordinates {
          (100000,0.0004)(1000000,0.002)(10000000,0.021)
          (100000000,0.200)(200000000,0.398)(300000000,0.583)(400000000, 0.773)
			(5e8, 0.989) 	
 	};
     \addlegendentry{\en{Alt1}}
     
 	
 	\addplot[ color=green, mark=square,]
      coordinates {
          (100000,0.0001)(1000000,0.002)(10000000,0.016)
          (100000000,0.167)(200000000,0.334)(300000000,0.502)(400000000, 0.617)
			(5e8, 0.818) 	
 	};
 	\addlegendentry{\en{Alt2}}
 	
 	 	\addplot[ color=red, mark=square,]
      coordinates {
          (100000,0.0001)(1000000,0.002)(10000000,0.020)
          (1e8,0.199)(2e8,0.397)(3e8,0.599)(4e8, 0.788)
			(5e8,0.991) 	
 	};
 	\addlegendentry{\en{Alt3}}
     \end{axis}
 \end{tikzpicture}}

\caption{Σύγκριση αποτελεσμάτων}
    &
\renewcommand{\arraystretch}{1.1}
\resizebox{0.4\textwidth}{!} {
\begin{tabular}{c|c}
Μέγεθος & Επιτάχυνση (\%)  \\
\hline
100000    & - \\
1000000   & - \\
10000000  & 23 \\
100000000 & 17 \\
200000000 & 16 \\
300000000 & 13.9\\
400000000 & 20 \\
\end{tabular}}
\captionof{table}{Ποσοστιαία σύγκριση μεταξύ \en{Alt1} και \en{Alt2}}
\end{tabular}
\end{figure}

Η επιλογή για διανυσματικοποίηση κατά τη μεταγλώττιση επιφέρει περίπου 20\% βελτίωση στις χρονικές επιδόσεις των εκτελέσεων με διαφορετικά μεγέθη διανύσματος. Ακόμη, η επιλογή μεταγλώττισης με \emph{\en{-fno-tree-vectorize}} έχει τις ίδιες επιδόσεις με την παραλλαγή \en{Alt3}, πράγμα που επιβεβαιώνει ότι η -Ο2 δεν υπονοεί αυτόματη διανυσματικοποίηση. Τα
αποτελέσματα αυτά θα χρησιμοποιηθούν για συγκρίσεις με τις παραλλαγές που θα ακολουθήσουν. 

\subsection{Παραλλαγή με οδηγία \en{parallel for}}
\subparagraph{}
Η πρώτη υλοποίηση παραλλαγής με παραλληλισμό της συνάρτησης \en{saxpy} περιλαμβάνει τον επαναληπτικό βρόγχο ενσωματωμένο στην οδηγία \emph{\en{parallel for}} στον οποίο γίνεται ο υπολογισμός για κάθε στοιχείο των διανυσμάτων. Τα αποτελέσματα φαίνονται στον ακολουθούμενο πίνακα.
\\
\selectlanguage{english}
\begin{spacing}{1.0}
\begin{lstlisting}[language=C++, caption={\el{Υλοποίηση παραλλαγής με \en{parallel for}}} , frame=tlrb]{Name}
void saxpy(size_t n, float a, const float *x, float *y) {
    #pragma omp parallel for
    for (size_t i = 0; i < n; ++i) {
        y[i] = a * x[i] + y[i];
    }
} 
\end{lstlisting}
\end{spacing}
\selectlanguage{greek}
\clearpage

\begin{table}[h]
    \centering
    \caption{Επιλογές μεταγλώττισης}
    \label{my-label}
    \begin{tabular}{
    |p{0.1\textwidth}
    | >{\centering\arraybackslash}p{0.9\textwidth}
    |}
    \hline
 {\textbf{\en{Label}}} & \textbf{\en{Options}} \\ \hline
     \textbf{\en{Alt4}} & \en{ -fopt-info-vec=info.log -fno-inline -fno-tree-vectorize -fopenmp -Wall  -Wextra -std=c++14 -O2} \\ \hline
     \textbf{\en{Alt5}} & \en{ -fopt-info-vec=info.log -fno-inline -ftree-vectorize -fopenmp -Wall  -Wextra -std=c++14 -O2} \\ \hline
    \end{tabular}
\end{table}

\begin{table}[h]
    \centering
    \caption{Καταγραφή χρόνων εκτέλεσης}
    \label{my-label}
    \begin{tabular}{|p{0.30\textwidth}| >{\centering\arraybackslash}p{0.25\textwidth}| >{\centering\arraybackslash}p{0.25\textwidth}|}
    \hline
    \multirow{2}{*}{\textbf{Μέγεθος προβλήματος}} & \multicolumn{2}{|c|}{\textbf{Χρόνοι εκτέλεσης \en{(sec)}}} \\ \cline{2-3} 
               & \textbf{\en{Alt4}} & \textbf{\en{Alt5}} \\ \hline
     100000    & 0.005 & 0.005  \\ \cline{1-3} 
     1000000   & 0.006 & 0.003 \\ \cline{1-3} 
     10000000  & 0.016 & 0.015 \\ \cline{1-3} 
     100000000 & 0.122 & 0.126 \\ \cline{1-3} 
     200000000 & 0.245 & 0.243 \\ \cline{1-3} 
     300000000 & 0.370 & 0.379 \\ \cline{1-3} 
     400000000 & 0.485 & 0.488 \\ \cline{1-3} 
     500000000 & 0.557 & 0.547 \\ \cline{1-3} 
     600000000 & 0.526 & 0.531 \\ \cline{1-3} 
     700000000 & 0.485 & 0.487 \\ \cline{1-3} 
     800000000 & 0.482 & 0.483 \\ \cline{1-3} 
    \end{tabular}
\end{table}

\begin{center}
\resizebox{0.7\textwidth}{!} {
\begin{tikzpicture}[state/.append style={minimum size=7mm}]
     \begin{axis}[
         title={Χρόνοι εκτέλεσης \en{Alt2 - Alt4 - Alt5}},
         xlabel={Μέγεθος διανυσμάτων},
         ylabel={Χρόνος εκτέλεσης},
         xmin=100000, xmax=800000000,
         ymin=0, ymax=0.9,
         xtick={ 100000000, 200000000, 300000000, 400000000,
          500000000, 6e8, 7e8, 8e8},
         ytick={ 0, 0.15, 0.3, 0.45, 0.6, 0.75, 0.9},
         legend pos=north west,
        % ymajorgrids=true,
        % grid style=dashed,
     ]
    
     \addplot[ color=blue, mark=square,]
      coordinates {
          (100000, 0.005)(1000000, 0.006)(10000000, 0.016)
          (100000000, 0.122)(200000000, 0.245)(300000000, 0.37)
          (400000000, 0.485)(500000000, 0.557)(600000000, 0.526)
          (700000000, 0.485)(800000000, 0.482)
 	};
 	\addlegendentry{\en{Alt4}}
 	
     \addplot[ color=red, mark=square,]
      coordinates {
          (100000, 0.005)(1000000, 0.003)(10000000,0.015)
          (100000000, 0.126)(200000000,0.243)(300000000,0.379)
          (400000000, 0.488)(500000000, 0.547)(600000000, 0.531)
          (700000000, 0.487)(800000000, 0.483 )
 	};
 	\addlegendentry{\en{Alt5}}
 	
 	\addplot[ color=green, mark=square,]
      coordinates {
          (100000,0.0001)(1000000,0.002)(10000000,0.016)
          (100000000,0.167)(200000000,0.334)(300000000,0.502)(400000000, 0.617)
          (5e8, 0.752)(6e8, 1.068)
 	};
 	\addlegendentry{\en{Alt2}}

     \end{axis}
\end{tikzpicture}}
\end{center}


\clearpage
\subsubsection{Παρατηρήσεις}
\subparagraph{}
Με τη χρήση της οδηγίας \emph{\en{pragma omp parallel for}} επετέφχθει μείωση του χρόνου εκτέλεσης του αλγορίθμου σε σύγκριση με την αντίστοιχη σειριακή παραλλαγή. Δεν προκύπτει καμία διαφοροποίηση της μεταγλώττισης με εντολή διανυσματικοποίησης ή χωρίς. Σύμφωνα ωστόσο με τον ορισμό του \en{false sharing}, υπάρχει πιθανότητα περαιτέρω βελτίωσης της παραλλαγής με οδηγία \en{parallel for}, κάτι που εξετάζεται στην επόμενη ενότητα.

\subsection{Παραλλαγή με \emph{\en{parallel for}} και \en{padding}}
\subparagraph{}
Σε αυτή την περίπτωση, ο αλγόριθμος παραμένει ίδιος, χρησιμοποιείται δηλαδή η οδηγία \emph{\en{pragma omp parallel for}}. Ωστόσο στη συνάρτηση εισάγονται ως ορίσματα δομές που εμπεριέχουν μία μεταβλητή αριθμού μικρής ακρίβειας και ένα τεχνητό κενό \emph{\en{padding}}. Το μέγεθος της είναι 64\en{bytes} και έχει ως στόχο την αποφυγή του φαινομένου \textbf{\en{false sharing}}.

\selectlanguage{english}
\begin{spacing}{0.9}
\begin{lstlisting}[language=C++, caption={\el{Υλοποίηση παραλλαγής με \en{parallel for}}} , frame=tlrb]{Name}
void saxpy(size_t n, float a, const float64 *x, float64 *y) {
    #pragma omp parallel for
    for (size_t i = 0; i < n; ++i) {
        y[i].val = a * x[i].val + y[i].val;
    }
} 
\end{lstlisting}
\end{spacing}
\selectlanguage{greek}
\begin{table}[h]
    \centering
    \caption{Επιλογές μεταγλώττισης}
    \label{my-label}
    \begin{tabular}{
    |p{0.1\textwidth}
    | >{\centering\arraybackslash}p{0.9\textwidth}
    |}
    \hline
 {\textbf{\en{Label}}} & \textbf{\en{Options}} \\ \hline
     \textbf{\en{Alt6}} & \en{ -fopt-info-vec=info.log -fno-inline -fno-tree-vectorize -fopenmp -Wall  -Wextra -std=c++14 -O2} \\ \hline
     \textbf{\en{Alt7}} & \en{ -fopt-info-vec=info.log -fno-inline -ftree-vectorize -fopenmp -Wall  -Wextra -std=c++14 -O2} \\ \hline
    \end{tabular}
\end{table}


\begin{table}[h]
    \centering
    \caption{Καταγραφή χρόνων εκτέλεσης}
    \label{my-label}
    \begin{tabular}{|p{0.30\textwidth}| >{\centering\arraybackslash}p{0.25\textwidth}| >{\centering\arraybackslash}p{0.25\textwidth}|}
    \hline
    \multirow{2}{*}{\textbf{Μέγεθος προβλήματος}} & \multicolumn{2}{|c|}{\textbf{Χρόνοι εκτέλεσης \en{(sec)}}} \\ \cline{2-3} 
               & \textbf{\en{Alt6}} & \textbf{\en{Alt7}} \\ \hline
     100000    & 0.006  & 0.006 \\ \cline{1-3} 
     1000000   & 0.023  & 0.025 \\ \cline{1-3} 
     10000000  & 0.193  & 0.198  \\ \cline{1-3} 
     100000000 &  \en{killed} & \en{killed} \\ \cline{1-3} 
    \end{tabular}
\end{table}

\subsubsection{Παρατηρήσεις}
\subparagraph{}
Το πρόβλημα \emph{\en{false sharing}} δεν ήταν δυνατό να εντοπισθεί στη
συγκεκριμένη παραλλαγή. Μάλιστα, 
ο έλεγχος για μεγέθη διανυσμάτων μεγαλύτερων των 1\en{e}8 στοιχείων, ήταν ανεπιτυχής λόγω έλλειψης υπολογιστικών πόρων.

\subsection{Παραλλαγές με χρήση οδηγίας \emph{\en{SIMD}}}
\subparagraph{}
Η συγκεκριμένη ενότητα καθώς και ορισμένες που ακολουθούν αφορούν την επίλυση του προβλήματος \en{\emph{SAXPY}} με χρήση της οδηγίας διανυσματικοποίησης \textbf{\en{SIMD}}. Οπως προαναφέρθηκε, η οδηγία δεν έχει ως στόχο την παραλληλοποίηση τμήματος κώδικα, αλλά την ταυτόχρονη εκτέλεση εντολών ως μία εντολή \en{\textbf{SIMD}}. Στη περίπτωση του \en{g++}, γίνεται αυτόματη προσπάθεια διανυσματικοποίησης χρησιμοποιείται η επιλογή -Ο3. Αυτός ειναι και ο λόγος που στα προηγούμενα παραδείγματα χρησιμοποιήθηκε επιλογή μεταγλώττισης -Ο2 με ταυτόχρονη χρήση των εντολών \en{\textbf{-fno-tree-vectorize}} και 
\en{\textbf{ftree-vectorize}} που επιτρέπουν στο χρήστη να επιλέγει μεταγλώττιση με διανυσματικοποίηση ή χωρίς. Παράλληλα, χρησιμοποιούνται οι εντολές \en{\textbf{-fno-inline}} και \en{\textbf{-fopt-info-vec}}. Η πρώτη απογορεύει στον μεταγγλωτιστή το \en{loop-unrolling} ενώ η δεύτερη δημιουργεί ένα αρχείο με χρήσιμα μηνύματα κατά τη διάρκεια της μεταγλώττισης.

\subsubsection{Παραλλαγή με \emph{\en{omp simd}}}
\subparagraph{}
Η εντολή \en{\emph{pragma omp simd}} αποσκοπεί στη διανυσματικοποίηση του τμήματος κώδικα που ακολουθεί, χωρίς ωστόσο να γίνεται διαμοιρασμός των επαναλήψεων του βρόγχου σε διαφορετικά νήματα όπως θα γίνόταν με την οδηγία \en{\emph{pragma omp parallel for simd}}.
\selectlanguage{english}
\begin{spacing}{0.9}
\begin{lstlisting}[language=C++, caption={\el{Υλοποίηση παραλλαγής με \en{omp simd}}} , frame=tlrb]{Name}
void saxpy(size_t n, float a, const float *x, float *y) {
    #pragma omp simd
    for (size_t i = 0; i < n; ++i) {
        y[i] = a * x[i] + y[i];
    }
}
\end{lstlisting}
\end{spacing}
\selectlanguage{greek}

\begin{table}[h]
    \centering
    \caption{Επιλογές μεταγλώττισης}
    \label{my-label}
    \begin{tabular}{
    |p{0.1\textwidth}
    | >{\centering\arraybackslash}p{0.9\textwidth}
    |}
    \hline
 {\textbf{\en{Label}}} & \textbf{\en{Options}} \\ \hline
     \textbf{\en{Alt8}} & \en{ -fopt-info-vec=info.log -fno-inline -fno-tree-vectorize -fopenmp -Wall  -Wextra -std=c++14 -O2} \\ \hline
     \textbf{\en{Alt9}} & \en{ -fopt-info-vec=info.log -fno-inline -ftree-vectorize -fopenmp -Wall  -Wextra -std=c++14 -O2} \\ \hline
     \textbf{\en{Alt10}} & \en{ -fopt-info-vec=info.log -fno-inline -fopenmp -Wall  -Wextra -std=c++14 -O2} \\ \hline
    \end{tabular}
\end{table}

\begin{table}[h]
    \centering
    \caption{Καταγραφή χρόνων εκτέλεσης}
    \label{my-label}
    \begin{tabular}{|p{0.30\textwidth}
    | >{\centering\arraybackslash}p{0.12\textwidth}
    | >{\centering\arraybackslash}p{0.12\textwidth}
    | >{\centering\arraybackslash}p{0.12\textwidth}
|}
    \hline
    \multirow{2}{*}{\textbf{Μέγεθος προβλήματος}} & \multicolumn{3}{|c|}{\textbf{Χρόνοι εκτέλεσης \en{(sec)}}} \\ \cline{2-4} 
      & \textbf{\en{Alt8}} & \textbf{\en{Alt9}} & \textbf{\en{Alt10}} \\ \hline
     100000    & 0.001 & 0.001 & 0.001\\ \cline{1-4} 
     1000000   & 0.002 & 0.002 & 0.002 \\ \cline{1-4} 
     10000000  & 0.021 & 0.017 & 0.017\\ \cline{1-4} 
     100000000 & 0.202 & 0.165 & 0.145\\ \cline{1-4} 
     200000000 & 0.401 & 0.329 & 0.296\\ \cline{1-4} 
     300000000 & 0.592 & 0.503 & 0.496\\ \cline{1-4} 
     400000000 & 0.783 & 0.639 & 0.642\\ \cline{1-4} 
     500000000 & 0.995 & 0.827 & 0.844\\ \cline{1-4} 
    \end{tabular}
\end{table}



\begin{figure}[h]
\centering 
\resizebox{0.5\textwidth}{!} {
\begin{tikzpicture}   
    \begin{axis}[
         title={Χρόνοι εκτέλεσης με \en{Alt1 - Alt8}},
         xlabel={Μέγεθος διανυσμάτων},
         ylabel={Χρόνος εκτέλεσης},
         xmin=1e8, xmax=5e8,
         ymin=0, ymax=1,
         xtick={ 1e8, 2e8, 3e8, 4e8, 5e8},
         ytick={0, 0.2, 0.4, 0.6, 0.8, 1},
         legend pos=north west,
        % ymajorgrids=true,
        % grid style=dashed,
     ]
    
     \addplot[ color=blue, mark=square,]
      coordinates {
          (1e8, 0.202)(2e8,0.401)(3e8, 0.592)
          (4e8, 0.783)(5e8, 0.995)
 	};
  	\addlegendentry{\en{Alt8}}

          \addplot[ color=red, mark=square,]
      coordinates {
          (100000000,0.200)(200000000,0.398)
          (300000000,0.583)(400000000, 0.773)
          (5e8, 0.989)
 	};
     \addlegendentry{\en{Alt1}}

    \end{axis}
\end{tikzpicture}}% NO EMPTY LINE HERE!!!!
\resizebox{0.5\textwidth}{!} {
\begin{tikzpicture}
    \begin{axis}[
         title={Χρόνοι εκτέλεσης με \en{Alt2 - Alt8}},
         xlabel={Μέγεθος διανυσμάτων},
         ylabel={Χρόνος εκτέλεσης},
         xmin=1e8, xmax=5e8,
         ymin=0, ymax=1,
         xtick={ 1e8, 2e8, 3e8, 4e8, 5e8},
         ytick={0, 0.2, 0.4, 0.6, 0.8, 1},
         legend pos=north west,
        % ymajorgrids=true,
        % grid style=dashed,
     ]
    
 	\addplot[ color=green, mark=square,]
      coordinates {
          (100000,0.0001)(1000000,0.002)(10000000,0.016)
          (100000000,0.167)(200000000,0.334)(300000000,0.502)(400000000, 0.617)
          (5e8, 0.818)
 	};
 	\addlegendentry{\en{Alt2}}

     
     \addplot[ color=red, mark=square,]
      coordinates {
          (1e8, 0.165)(2e8, 0.329)(3e8, 0.503)
          (4e8, 0.639)(5e8, 0.827)
 	};
 	\addlegendentry{\en{Alt9}}
    \end{axis}
\end{tikzpicture}} 
\caption{Left: No Interaction. Right: Interaction} \label{fig:M}  
\end{figure}
\clearpage
\begin{center}
\resizebox{0.7\textwidth}{!} {
\begin{tikzpicture}   
    \begin{axis}[
         title={Σύγκριση \en{Alt3} με \en{Alt10}},
         xlabel={Μέγεθος διανυσμάτων},
         ylabel={Χρόνος εκτέλεσης},
         xmin=1e8, xmax=5e8,
         ymin=0, ymax=1,
         xtick={ 1e8, 2e8, 3e8, 4e8, 5e8},
         ytick={0, 0.2, 0.4, 0.6, 0.8, 1},
         legend pos=north west,
        % ymajorgrids=true,
        % grid style=dashed,
     ]
    
     \addplot[ color=blue, mark=square,]
      coordinates {
          (1e8, 0.199)(2e8,0.397)(3e8, 0.599)
          (4e8, 0.788)(5e8, 0.991)
 	};
  	\addlegendentry{\en{Alt3}}

          \addplot[ color=red, mark=square,]
      coordinates {
          (100000000,0.145)(200000000,0.296)
          (300000000,0.496)(400000000, 0.642)
          (5e8, 0.844)
 	};
     \addlegendentry{\en{Alt10}}

    \end{axis}
\end{tikzpicture}}% NO EMPTY LINE HERE!!!! 
\end{center}

Από τα παραπάνω διαγράμματα διαφαίνεται ότι η μεταγλώττιση με επιλογή \emph{\en{-fno-tree-vectorize}} και \emph{\en{-ftree-vectorize}} παρακάμπτει την οδηγία \en{omp simd} και εκτελείται ως σειριακή, με διανυσματικοποίηση ή χωρίς, ανάλογα με την επιλογή. Στη περίπτωση που δε δοθεί η επιλογή ωστόσο, τότε διανυσματικοποίηση εφαρμόζεται μέσω του \emph{\en{OpenMP}} και της οδηγίας \emph{\en{omp simd}} αν υπάρχει
\begin{center}
\begin{table}[h]
    \centering
    \caption{Συνοπτικός πίνακας εφαρμογών διανθσματικοποίησης}
    \label{my-label}
    \begin{tabular}{
    |p{0.3\textwidth}
    | >{\centering\arraybackslash}p{0.25\textwidth}
        | >{\centering\arraybackslash}p{0.25\textwidth}
    |}
    \hline
 \textbf{Επιλογή μεταγλώττισης} & \textbf{Σειριακή} & \en{\textbf{OpenMP - omp simd}}\\ \hline
     \textbf{\en{-fno-tree-vectorize}} & Οχι & Οχι \\ \hline
     \textbf{\en{-ftree-vectorize}}    & Ναι & Ναι\\ \hline
     \textbf{\en{None}}                & Οχι & Ναι\\ \hline
    \end{tabular}
\end{table}
\end{center}

\clearpage
\subsubsection{Παραλλαγή με \emph{\en{omp parallel for simd}}}
\subparagraph{}
Στην παραλλαγή αυτής της ενότητας χρησιμοποιείται ο συνδυασμός παραλληλοποίησης μέσω της οδηγίας \emph{\en{parallel for}} με διανυσματικοποίηση μέσω \emph{\en{simd}}. 
\selectlanguage{english}
\begin{spacing}{0.9}
\begin{lstlisting}[language=C++, caption={\el{Υλοποίηση παραλλαγής με \en{omp parallel for simd}}} , frame=tlrb]{Name}
void saxpy(size_t n, float a, const float *x, float *y) {
    #pragma omp parallel for simd
    for (size_t i = 0; i < n; ++i) {
        y[i] = a * x[i] + y[i];
    }
}
\end{lstlisting}
\end{spacing}
\selectlanguage{greek}

\begin{table}[h]
    \centering
    \caption{Επιλογές μεταγλώττισης}
    \label{my-label}
    \begin{tabular}{
    |p{0.1\textwidth}
    | >{\centering\arraybackslash}p{0.8\textwidth}
    |}
    \hline
 {\textbf{\en{Label}}} & \textbf{\en{Options}} \\ \hline
     \textbf{\en{Alt11}} & \en{ -fopt-info-vec=info.log -fno-inline -fno-tree-vectorize -fopenmp -Wall  -Wextra -std=c++14 -O2} \\ \hline
     \textbf{\en{Alt12}} & \en{ -fopt-info-vec=info.log -fno-inline -ftree-vectorize -fopenmp -Wall  -Wextra -std=c++14 -O2} \\ \hline
     \textbf{\en{Alt13}} & \en{ -fopt-info-vec=info.log -fno-inline -fopenmp -Wall  -Wextra -std=c++14 -O2} \\ \hline
    \end{tabular}
\end{table}

\begin{table}[h]
    \centering
    \caption{Καταγραφή χρόνων εκτέλεσης}
    \label{my-label}
    \begin{tabular}{|p{0.30\textwidth}
    | >{\centering\arraybackslash}p{0.12\textwidth}
    | >{\centering\arraybackslash}p{0.12\textwidth}
    | >{\centering\arraybackslash}p{0.12\textwidth}
|}
    \hline
    \multirow{2}{*}{\textbf{Μέγεθος προβλήματος}} & \multicolumn{3}{|c|}{\textbf{Χρόνοι εκτέλεσης \en{(sec)}}} \\ \cline{2-4} 
      & \textbf{\en{Alt11}} & \textbf{\en{Alt12}} & \textbf{\en{Alt13}} \\ \hline
     100000    & 0.006 & 0.002 & 0.005 \\ \cline{1-4} 
     1000000   & 0.006 & 0.002 & 0.004 \\ \cline{1-4} 
     10000000  & 0.015 & 0.016 & 0.016 \\ \cline{1-4} 
     100000000 & 0.129 & 0.123 & 0.124 \\ \cline{1-4} 
     200000000 & 0.247 & 0.251 & 0.246 \\ \cline{1-4} 
     300000000 & 0.368 & 0.368 & 0.366 \\ \cline{1-4} 
     400000000 & 0.489 & 0.486 & 0.486 \\ \cline{1-4} 
     500000000 & 0.578 & 0.576 & 0.577 \\ \cline{1-4}
     500000000 & 0.578 & 0.576 & 0.577 \\ \cline{1-4} 
     600000000 & 0.515 & 0.458 & 0.525 \\ \cline{1-4} 
     700000000 & 0.496 & 0.496 & 0.460 \\ \cline{1-4} 
     800000000 & 0.487 & 0.500 & 0.483 \\ \cline{1-4} 

    \end{tabular}
\end{table}

Από τις παραπάνω εκτελέσεις του αλγόριθμου προκύπτει το συμπέρασμα ότι η οδηγία διανυσματικοποίησης μέσω \en{simd} δεν λαμβάνεται υπόψη όταν εφαρμόζεται σε συνδυασμό με την οδηγία \en{parallel for}.

\clearpage
\begin{figure}[h]
\centering 
\resizebox{0.7\textwidth}{!} {
\begin{tikzpicture}   
    \begin{axis}[
         title={Χρόνοι εκτέλεσης με \en{Alt5 - Alt13}},
         xlabel={Μέγεθος διανυσμάτων},
         ylabel={Χρόνος εκτέλεσης},
         xmin=1e8, xmax=5e8,
         ymin=0, ymax=1,
         xtick={ 1e8, 2e8, 3e8, 4e8, 5e8},
         ytick={0, 0.2, 0.4, 0.6, 0.8, 1},
         legend pos=north west,
        % ymajorgrids=true,
        % grid style=dashed,
     ]
    
    \addplot[ color=red, mark=square,]
      coordinates {
          (1e8,0.124)(2e8,0.246)
          (3e8,0.366)(4e8,0.486)
          (5e8,0.577)(6e8, 0.525) (7e8, 0.460 )(8e8, 0.483)
 	};
  	\addlegendentry{\en{Alt13}}

    \addplot[ color=green, mark=square,]
      coordinates {
          (1e8,0.126)(2e8,0.243)
          (3e8,0.379)(4e8, 0.488)
          (5e8, 0.547)(6e8, 0.531) (7e8, 0.487)(8e8, 0.483 )
 	};
     \addlegendentry{\en{Alt5}}

    \end{axis}
\end{tikzpicture}}% NO EMPTY LINE HERE!!!!
\end{figure}
\clearpage
\subsubsection{Παραλλαγή με \emph{\en{omp declare simd uniform}}}
\subparagraph{} 
Υλοποίηση παραλλαγής με χρήση της φράσης \emph{\en{uniform}}. Θεωρητικά δεν υπάρχει κέρδος στις επιδόσεις του αλγορίθμου σε σχέση με της προηγούμενες παραλλαγές. Η εναλλακτική αυτη χρησιμοποιείται για την επαλήθευση της διανυσματικοποίησης μέσω \emph{\en{OpenMP}}.
\selectlanguage{english}
\begin{spacing}{0.9}
\begin{lstlisting}[language=C++, caption={\el{Υλοποίηση παραλλαγής με \en{omp declare simd uniform}}} , frame=tlrb]{Name}
#pragma omp declare simd uniform(a)
float do_work(float a, float b, float c)
{
    return a * b + c;
}

void saxpy(size_t n, float a, const float *x, float *y) {
    #pragma omp simd
    for (size_t i = 0; i < n; ++i) {
        y[i] = do_work(a, x[i], y[i]);
    }
}
\end{lstlisting}
\end{spacing}
\selectlanguage{greek}
\begin{table}[h]
    \centering
    \caption{Επιλογές μεταγλώττισης}
    \label{my-label}
    \begin{tabular}{
    |p{0.1\textwidth}
    | >{\centering\arraybackslash}p{0.8\textwidth}
    |}
    \hline
 {\textbf{\en{Label}}} & \textbf{\en{Options}} \\ \hline
     \textbf{\en{Alt14}} & \en{ -fopt-info-vec=info.log -fno-inline -fno-tree-vectorize -fopenmp -Wall  -Wextra -std=c++14 -O2} \\ \hline
     \textbf{\en{Alt15}} & \en{ -fopt-info-vec=info.log -fno-inline -ftree-vectorize -fopenmp -Wall  -Wextra -std=c++14 -O2} \\ \hline
     \textbf{\en{Alt16}} & \en{ -fopt-info-vec=info.log -fno-inline -fopenmp -Wall  -Wextra -std=c++14 -O2} \\ \hline
    \end{tabular}
\end{table}

\begin{table}[h]
    \centering
    \caption{Καταγραφή χρόνων εκτέλεσης}
    \label{my-label}
    \begin{tabular}{|p{0.30\textwidth}
    | >{\centering\arraybackslash}p{0.15\textwidth}
    | >{\centering\arraybackslash}p{0.15\textwidth}
    | >{\centering\arraybackslash}p{0.15\textwidth}
|}
    \hline
    \multirow{2}{*}{\textbf{Μέγεθος προβλήματος}} & \multicolumn{3}{|c|}{\textbf{Χρόνοι εκτέλεσης \en{(sec)}}} \\ \cline{2-4} 
      & \textbf{\en{Alt14}} & \textbf{\en{Alt15}} & \textbf{\en{Alt16}} \\ \hline
     100000    & 0.0003 & 0.0001 & 0.0003 \\ \cline{1-4} 
     1000000   & 0.003 & 0.002 & 0.002 \\ \cline{1-4} 
     10000000  & 0.035 & 0.018 & 0.018 \\ \cline{1-4} 
     100000000 & 0.346 & 0.180 & 0.179 \\ \cline{1-4} 
     200000000 & 0.700 & 0.339 & 0.318 \\ \cline{1-4} 
     300000000 & 1.030 & 0.479 & 0.469 \\ \cline{1-4} 
     400000000 & 1.384 & 0.707 & 0.637 \\ \cline{1-4}
     500000000 & 1.707 & 0.944 & 0.911 \\ \cline{1-4} 
    \end{tabular}
\end{table}

\begin{figure}[h]
\centering
\begin{tikzpicture}   
    \begin{axis}[
         title={Χρόνοι εκτέλεσης με \en{Alt14 - Alt8}},
         xlabel={Μέγεθος διανυσμάτων},
         ylabel={Χρόνος εκτέλεσης},
         xmin=1e8, xmax=5e8,
         ymin=0, ymax=1.8,
         xtick={ 1e8, 2e8, 3e8, 4e8, 5e8},
         ytick={0, 0.2, 0.4, 0.6, 0.8, 1, 1.2, 1.4, 1.6, 1.8},
         legend pos=north west,
        % ymajorgrids=true,
        % grid style=dashed,
     ]
    
     \addplot[ color=red, mark=square,]
      coordinates {
          (1e8, 0.346)(2e8,0.7)(3e8,1.030)
          (4e8,1.384)(5e8, 1.707)
 	};
  	\addlegendentry{\en{Alt14}}

     \addplot[ color=blue, mark=square,]
      coordinates {
          (1e8,0.202 )(2e8,0.401)(3e8,0.592)
          (4e8,0.783)(5e8,0.995)
 	};
     \addlegendentry{\en{Alt8}}
    \end{axis}
\end{tikzpicture}% NO EMPTY LINE HERE!!!!
\begin{tikzpicture}
    \begin{axis}[
         title={Χρόνοι εκτέλεσης με \en{Alt15 - Alt9}},
         xlabel={Μέγεθος διανυσμάτων},
         ylabel={Χρόνος εκτέλεσης},
         xmin=1e8, xmax=5e8,
         ymin=0, ymax=1,
         xtick={ 1e8, 2e8, 3e8, 4e8, 5e8},
         ytick={0, 0.2, 0.4, 0.6, 0.8, 1},
         legend pos=north west,
        % ymajorgrids=true,
        % grid style=dashed,
     ]
    
 	\addplot[ color=green, mark=square,]
      coordinates {
          (100000000,0.180)(200000000,0.339)(300000000,0.479)(400000000,0.707 )
          (5e8, 0.944)
 	};
 	\addlegendentry{\en{Alt15}}

     
     \addplot[ color=red, mark=square,]
      coordinates {
          (1e8, 0.165)(2e8,0.329)(3e8,0.503)
          (4e8, 0.639 )(5e8, 0.827)
 	};
 	\addlegendentry{\en{Alt9}}
    \end{axis}
\end{tikzpicture}
\end{figure}

Από τα διαγράμματα προκύπτει ότι ο υπολογισμός \en{SAXPY} μέσω συνάρτησης, προσδίδει εξτρά καθυστέρηση στην συνολική απόδοση, που πιθανόν οφείλεται στη μετακίνηση της διεύθυνσης μνήμης καθώς κατά τη μεταγλώττιση απενεργοποιείται το \emph{\en{loop unrolling}}. Στη περίπτωση της \en{explicit} εντολής διανυσματικοποίησης κατά τη μεταγλώττιση, οι επιδόσεις δείχνουν να μοιάζουν, με λίγο καλύτερες επιδόσεις για την εκτέλεση χωρίς ενδιάμεση κλήση συνάρτησης.
\subsubsection{Παραλλαγή με \emph{\en{omp declare simd uniform notinbranch}}}
\subparagraph{} 
Στη περίπτωση που ακολουθεί, γίνεται προσπάθεια εξαγωγής συμπερασμάτων σχετικών με τη φράση \emph{\en{notinbranch}} και το κέρδος που μπορεί να επιφέρει σε σύγκριση με τη προηγούμενη παραλλαγή.
\selectlanguage{english}
\begin{spacing}{0.9}
\begin{lstlisting}[language=C++, caption={\el{Υλοποίηση παραλλαγής με \en{omp declare simd uniform
 notinbranch}}} , frame=tlrb]{Name}
#pragma omp declare simd uniform(a) notinbranch
float do_work(float a, float b, float c)
{
    return a * b + c;
}

void saxpy(size_t n, float a, const float *x, float *y) {
    #pragma omp simd
    for (size_t i = 0; i < n; ++i) {
        y[i] = do_work(a, x[i], y[i]);
    }
}
\end{lstlisting}
\end{spacing}
\selectlanguage{greek}
\begin{table}[h]
    \centering
    \caption{Επιλογές μεταγλώττισης}
    \label{my-label}
    \begin{tabular}{
    |p{0.1\textwidth}
    | >{\centering\arraybackslash}p{0.8\textwidth}
    |}
    \hline
 {\textbf{\en{Label}}} & \textbf{\en{Options}} \\ \hline
     \textbf{\en{Alt17}} & \en{ -fopt-info-vec=info.log -fno-inline -fno-tree-vectorize -fopenmp -Wall  -Wextra -std=c++14 -O2} \\ \hline
     \textbf{\en{Alt18}} & \en{ -fopt-info-vec=info.log -fno-inline -ftree-vectorize -fopenmp -Wall  -Wextra -std=c++14 -O2} \\ \hline
     \textbf{\en{Alt19}} & \en{ -fopt-info-vec=info.log -fno-inline -fopenmp -Wall  -Wextra -std=c++14 -O2} \\ \hline
    \end{tabular}
\end{table}
\begin{table}[h]
    \centering
    \caption{Καταγραφή χρόνων εκτέλεσης}
    \label{my-label}
    \begin{tabular}{|p{0.30\textwidth}
    | >{\centering\arraybackslash}p{0.12\textwidth}
    | >{\centering\arraybackslash}p{0.12\textwidth}
    | >{\centering\arraybackslash}p{0.12\textwidth}
|}
    \hline
    \multirow{2}{*}{\textbf{Μέγεθος προβλήματος}} & \multicolumn{3}{|c|}{\textbf{Χρόνοι εκτέλεσης \en{(sec)}}} \\ \cline{2-4} 
      & \textbf{\en{Alt17}} & \textbf{\en{Alt18}} & \textbf{\en{Alt19}} \\ \hline
     100000    & 0.001 & 0.001 & 0.001 \\ \cline{1-4} 
     1000000   & 0.003 & 0.002 & 0.002 \\ \cline{1-4} 
     10000000  & 0.035 & 0.018 & 0.018 \\ \cline{1-4} 
     100000000 & 0.343 & 0.180 & 0.175 \\ \cline{1-4} 
     200000000 & 0.689 & 0.359 & 0.319 \\ \cline{1-4} 
     300000000 & 1.025 & 0.536 & 0.469 \\ \cline{1-4} 
     400000000 & 1.375 & 0.702 & 0.619 \\ \cline{1-4} 
     500000000 & 1.723 & 0.909 & 0.809 \\ \cline{1-4} 

    \end{tabular}
\end{table}


\begin{figure}[h]
\centering 
\resizebox{0.5\textwidth}{!} {
\begin{tikzpicture}   
    \begin{axis}[
         title={Χρόνοι εκτέλεσης με \en{Alt17 - Alt14}},
         xlabel={Μέγεθος διανυσμάτων},
         ylabel={Χρόνος εκτέλεσης},
         xmin=1e8, xmax=5e8,
         ymin=0, ymax=2,
         xtick={ 1e8, 2e8, 3e8, 4e8, 5e8},
         ytick={0, 0.2, 0.4, 0.6, 0.8, 1, 1.2, 1.4, 1.6, 1.8, 2},
         legend pos=north west,
        % ymajorgrids=true,
        % grid style=dashed,
     ]
    
     \addplot[ color=red, mark=square,]
      coordinates {
          (1e8, 0.346)(2e8,0.7)(3e8,1.030)
          (4e8,1.384)(5e8, 1.707)
 	};
  	\addlegendentry{\en{Alt14}}

          \addplot[ color=blue, mark=square,]
      coordinates {
          (100000000,0.343)(200000000,0.689)
          (300000000,1.025)(400000000, 1.375)
          (5e8, 1.723)
 	};
     \addlegendentry{\en{Alt17}}

    \end{axis}
\end{tikzpicture}}% NO EMPTY LINE HERE!!!!
\resizebox{0.5\textwidth}{!} {
\begin{tikzpicture}
    \begin{axis}[
         title={Χρόνοι εκτέλεσης με \en{Alt15 - Alt18}},
         xlabel={Μέγεθος διανυσμάτων},
         ylabel={Χρόνος εκτέλεσης},
         xmin=1e8, xmax=5e8,
         ymin=0, ymax=1,
         xtick={ 1e8, 2e8, 3e8, 4e8, 5e8},
         ytick={0, 0.2, 0.4, 0.6, 0.8, 1},
         legend pos=north west,
        % ymajorgrids=true,
        % grid style=dashed,
     ]
    
 	\addplot[ color=green, mark=square,]
      coordinates {
          (1e8,0.180)(2e8,0.359)(3e8,0.536)(4e8,0.702)
          (5e8, 0.909)
 	};
 	\addlegendentry{\en{Alt18}}

     
     \addplot[ color=red, mark=square,]
      coordinates {
          (1e8, 0.180)(2e8, 0.339)(3e8, 0.479)
          (4e8, 0.707)(5e8, 0.944)
 	};
 	\addlegendentry{\en{Alt15}}
    \end{axis}
\end{tikzpicture}} 
\end{figure}
\clearpage
\clearpage
\begin{figure}[h]
\centering 
\resizebox{0.7\textwidth}{!} {
\begin{tikzpicture}   
    \begin{axis}[
         title={Χρόνοι εκτέλεσης με \en{Alt16 - Alt19}},
         xlabel={Μέγεθος διανυσμάτων},
         ylabel={Χρόνος εκτέλεσης},
         xmin=1e8, xmax=5e8,
         ymin=0, ymax=1,
         xtick={ 1e8, 2e8, 3e8, 4e8, 5e8},
         ytick={0, 0.2, 0.4, 0.6, 0.8, 1},
         legend pos=north west,
        % ymajorgrids=true,
        % grid style=dashed,
     ]
    
    \addplot[ color=red, mark=square,]
      coordinates {
          (1e8,0.179)(2e8,0.318)
          (3e8,0.469)(4e8,0.637)
          (5e8,0.911)
 	};
  	\addlegendentry{\en{Alt16}}

    \addplot[ color=green, mark=square,]
      coordinates {
          (1e8,0.175)(2e8,0.319)
          (3e8,0.469)(4e8, 0.619)
          (5e8, 0.809)
 	};
     \addlegendentry{\en{Alt19}}

    \end{axis}
\end{tikzpicture}}% NO EMPTY LINE HERE!!!!
\end{figure}
\clearpage
\subsection{Παραλλαγές με \emph{\en{offloading}}}
\subparagraph{}

Η ομάδα παραλλαγών αυτής της ενότητας, αφορά τη μεταφορά του αλγορίθμου σε άλλο μέσο για την επίλυση του. Το βασικότερο τμήμα του \emph{\en{SAXPY}}, εκτελείται στη μονάδα επεξεργασίας κάρτας γραφικών - \emph{\en{GPU}}. Στα πλαίσια της μεταφοράς, συμπεριλαμβάνεται και η μεταφορά των μεταβλητών από τη μνήμη της κεντρικής μονάδας στη μνήμη της κάρτας γραφικών. Αυτή η μεταφορά γίνεται με διάφορους τρόπου και τεχνικές που αναφέρονται στις επόμενες παραγράφους.

\subsubsection{Παραλλαγή με \emph{\en{target map}}}
\subparagraph{}
Στη συγκεκριμένη παραλλαγή γίνεται απλή μεταφορά του κώδικα και των μεταβλητών στην κάρτα γραφικών για την εκτέλεση του \emph{\en{SAXPY}} σε αυτή. 
\selectlanguage{english}
\begin{spacing}{0.9}
\begin{lstlisting}[language=C++, caption={\el{Υλοποίηση παραλλαγής με \en{target map}}} , frame=tlrb]{Name}
void saxpy(size_t n, float a, const float *x, float *y) {
    #pragma omp target map(tofrom: y[0:n]) map(to: x[0:n])
    for (size_t i = 0; i < n; ++i) {
        y[i] = a * x[i] + y[i];
    }
}
\end{lstlisting}
\end{spacing}
\selectlanguage{greek}
\begin{table}[h]
    \centering
    \caption{Επιλογές μεταγλώττισης}
    \label{my-label}
    \begin{tabular}{
    |p{0.1\textwidth}
    | >{\centering\arraybackslash}p{0.8\textwidth}
    |}
    \hline
 {\textbf{\en{Label}}} & \textbf{\en{Options}} \\ \hline
     \textbf{\en{Alt20}} & \en{ -fopt-info-vec=info.log -fno-inline -fno-tree-vectorize -fopenmp -Wall  -Wextra -std=c++14 -O2} \\ \hline
     \textbf{\en{Alt21}} & \en{ -fopt-info-vec=info.log -fno-inline -ftree-vectorize -fopenmp -Wall  -Wextra -std=c++14 -O2} \\ \hline
    \end{tabular}
\end{table}

\begin{table}[h]
    \centering
    \caption{Καταγραφή χρόνων εκτέλεσης}
    \label{my-label}
    \begin{tabular}{|p{0.30\textwidth}
    | >{\centering\arraybackslash}p{0.12\textwidth}
    | >{\centering\arraybackslash}p{0.12\textwidth}
|}
    \hline
    \multirow{2}{*}{\textbf{Μέγεθος προβλήματος}} & \multicolumn{2}{|c|}{\textbf{Χρόνοι εκτέλεσης \en{(sec)}}} \\ \cline{2-3} 
      & \textbf{\en{Alt20}} & \textbf{\en{Alt21}}  \\ \hline
     100000    & 0.881 & 0.873 \\ \cline{1-3} 
     1000000   & 1.200 & 1.169 \\ \cline{1-3} 
     10000000  & 3.848 & 3.752 \\ \cline{1-3} 
     100000000 & 29.984 & 29.937\\ \cline{1-3} 
     200000000 & 59.055 & 59.894 \\ \cline{1-3} 
    \end{tabular}
\end{table}
\clearpage
Παρόλο που η συγκρεκριμένη παραλλαγή προσομοιώνει σειριακή εκτέλεση στη μονάδα επεξεργασίας της κάρτας γραφικών, είναι προφανές ότι η δημιουργία ενός νέου περιβάλλοντος δεδομένων, με τα απαραίτητα δεδομένα για την εκτέλεση του προβλήματος και η εκτέλεσή του στη μονάδα επεξεργασίας γραφικών μέσω \emph{\en{OpenMP}}, αποτελεί μια χρονοβόρα διαδικασία, πολύ πιο χρονοβόρα από τη σειριακή εκτέλεση στη κεντρική μονάδα επεξεργασίας \emph{\en{CPU}}.


\subsection{Παραλλαγή με \emph{\en{target simd map}}}
\subparagraph{}
\selectlanguage{english}
\begin{spacing}{0.9}
\begin{lstlisting}[language=C++, caption={\el{Υλοποίηση παραλλαγής με \en{target simd map}}} , frame=tlrb]{Name}
void saxpy(size_t n, float a, const float *x, float *y) {
    #pragma omp target simd map(tofrom: y[0:n]) map(to: x[0:n])
    for (size_t i = 0; i < n; ++i) {
        y[i] = a * x[i] + y[i];
    }
}

\end{lstlisting}
\end{spacing}
\selectlanguage{greek}
\begin{table}[h]
    \centering
    \caption{Επιλογές μεταγλώττισης}
    \label{my-label}
    \begin{tabular}{
    |p{0.1\textwidth}
    | >{\centering\arraybackslash}p{0.8\textwidth}
    |}
    \hline
 {\textbf{\en{Label}}} & \textbf{\en{Options}} \\ \hline
     \textbf{\en{Alt22}} & \en{-fopt-info-vec=builds/alt22.log -O2 -fno-tree-vectorize -fno-inline -fno-stack-protector -foffload=nvptx-none="-O2 -fno-tree-vectorize -fno-inline" -fopenmp -o ./builds/Alt22} \\ \hline
     \textbf{\en{Alt23}} & \en{-fopt-info-vec=builds/alt23.log -O2 -ftree-vectorize -fno-inline -fno-stack-protector -foffload=nvptx-none="-O2 -ftree-vectorize -fno-inline" -fopenmp -o ./builds/Alt23} \\ \hline
     \textbf{\en{Alt24}} & \en{-fopt-info-vec=builds/alt24.log -O2  -fno-inline -fno-stack-protector -foffload=nvptx-none="-O2  -fno-inline" -fopenmp -o ./builds/Alt24} \\ \hline
    \end{tabular}
\end{table}

\begin{table}[h]
    \centering
    \caption{Καταγραφή χρόνων εκτέλεσης}
    \label{my-label}
    \begin{tabular}{|p{0.30\textwidth}
    | >{\centering\arraybackslash}p{0.12\textwidth}
    | >{\centering\arraybackslash}p{0.12\textwidth}
    | >{\centering\arraybackslash}p{0.12\textwidth}
|}
    \hline
    \multirow{2}{*}{\textbf{Μέγεθος προβλήματος}} & \multicolumn{3}{|c|}{\textbf{Χρόνοι εκτέλεσης \en{(sec)}}} \\ \cline{2-4} 
      & \textbf{\en{Alt22}} & \textbf{\en{Alt23}} & \textbf{\en{Alt24}} \\ \hline
     100000    & 0.888 & 0.809 & 0.837 \\ \cline{1-4} 
     1000000   & 0.841 & 0.837 & 0.833 \\ \cline{1-4} 
     10000000  & 1.018 & 1.018 & 0.997 \\ \cline{1-4} 
     100000000 & 2.359 & 2.406 & 2.392 \\ \cline{1-4} 
     200000000 & 4.387 & 5.149 & 5.157 \\ \cline{1-4} 
     300000000 & 5.883 & 7.256 & 7.494 \\ \cline{1-4} 
     400000000 & 7.189 & 6.969 & 7.155 	\\ \cline{1-4} 

    \end{tabular}
\end{table}

\clearpage

\begin{figure}[h]
\centering 
\resizebox{0.5\textwidth}{!} {
\begin{tikzpicture}   
    \begin{axis}[
         title={Χρόνοι εκτέλεσης με \en{Alt22 - Alt8}},
         xlabel={Μέγεθος διανυσμάτων},
         ylabel={Χρόνος εκτέλεσης},
         xmin=1e8, xmax=5e8,
         ymin=0, ymax=10,
         xtick={ 1e8, 2e8, 3e8, 4e8, 5e8},
         ytick={0, 2, 4, 6, 8, 10},
         legend pos=north west,
        % ymajorgrids=true,
        % grid style=dashed,
     ]
    
     \addplot[ color=red, mark=square,]
      coordinates {
          (1e8,2.359)(2e8,4.387)
          (3e8,5.883)(4e8,7.189)
 	};
  	\addlegendentry{\en{Alt22}}

          \addplot[ color=blue, mark=square,]
      coordinates {
          (100000000,0.202)(200000000,0.401)
          (300000000,0.592)(400000000, 0.783)
        
 	};
     \addlegendentry{\en{Alt8}}

    \end{axis}
\end{tikzpicture}}% NO EMPTY LINE HERE!!!!
\resizebox{0.5\textwidth}{!} {
\begin{tikzpicture}
    \begin{axis}[
         title={Χρόνοι εκτέλεσης με \en{Alt23 - Alt9}},
         xlabel={Μέγεθος διανυσμάτων},
         ylabel={Χρόνος εκτέλεσης},
         xmin=1e8, xmax=5e8,
         ymin=0, ymax=10,
         xtick={ 1e8, 2e8, 3e8, 4e8, 5e8},
         ytick={0, 2, 4, 6, 8, 10},
         legend pos=north west,
        % ymajorgrids=true,
        % grid style=dashed,
     ]
    
 	\addplot[ color=green, mark=square,]
      coordinates {
          (1e8,2.406)(2e8,5.149)(3e8,7.256)(4e8,6.969)(5e8, 8.433)
 	};
 	\addlegendentry{\en{Alt23}}

 	\addplot[ color=blue, mark=square,]
      coordinates {
          (1e8,0.165)(2e8,0.329)(3e8,0.503)(4e8,0.639)(5e8, 0.827)
 	};
 	\addlegendentry{\en{Alt9}}
    \end{axis}
\end{tikzpicture}} 
\end{figure}

\begin{figure}[h]
\centering 
\resizebox{0.5\textwidth}{!} {
\begin{tikzpicture}   
    \begin{axis}[
         title={Χρόνοι εκτέλεσης με \en{Alt24 - Alt19}},
         xlabel={Μέγεθος διανυσμάτων},
         ylabel={Χρόνος εκτέλεσης},
         xmin=1e8, xmax=5e8,
         ymin=0, ymax=10,
         xtick={ 1e8, 2e8, 3e8, 4e8, 5e8},
         ytick={0, 2, 3, 6, 8, 10},
         legend pos=north west,
        % ymajorgrids=true,
        % grid style=dashed,
     ]
    
    \addplot[ color=red, mark=square,]
      coordinates {
          (1e8,2.392)(2e8,5.157)
          (3e8,7.494)(4e8,7.155)(5e8, 8.476)
 	};
  	\addlegendentry{\en{Alt24}}

    \addplot[ color=green, mark=square,]
      coordinates {
          (1e8,0.145)(2e8,0.296)
          (3e8,0.496)(4e8, 0.642)(5e8, 0.844)
 	};
     \addlegendentry{\en{Alt10}}

    \end{axis}
\end{tikzpicture}}% NO EMPTY LINE HERE!!!!
\end{figure}
\subsection{Παραλλαγή με \emph{\en{target parallel for}}}
\subparagraph{}
\selectlanguage{english}
\begin{spacing}{0.9}
\begin{lstlisting}[language=C++, caption={\el{Υλοποίηση παραλλαγής με \en{target parallel for}}} , frame=tlrb]{Name}
void saxpy(size_t n, float a, const float *x, float *y) {
#pragma omp target parallel for map(tofrom: y[0:n]) map(to: x[0:n])
    for (size_t i = 0; i < n; ++i) {
        y[i] = a * x[i] + y[i];
    }
}
\end{lstlisting}
\end{spacing}
\selectlanguage{greek}
\begin{table}[h]
    \centering
    \caption{Καταγραφή χρόνων εκτέλεσης}
    \label{my-label}
    \begin{tabular}{| >{\centering\arraybackslash}p{0.25\textwidth}| 
    >{\centering\arraybackslash}p{0.25\textwidth}|
    >{\centering\arraybackslash}p{0.25\textwidth}|}
    \hline
    \multirow{2}{*}{\textbf{\shortstack{\\Μέγεθος \\ προβλήματος}}} & \multicolumn{2}{|c|}					{\textbf{Χρόνοι εκτέλεσης \en{(sec)}}} \\ \cline{2-3} 
        & \textbf{-Ο0}
        & \textbf{-O3} 

\\ \hline
     100000    & 0.011 & 0.011 \\ \cline{1-3} 
     1000000   & 0.009 & 0.013 \\ \cline{1-3} 
     10000000  & 0.035 & 0.021 \\ \cline{1-3} 
     100000000 & 0.150 & 0.127 \\ \cline{1-3} 
     200000000 & 0.264 & 0.251 \\ \cline{1-3} 
     300000000 & 0.387 & 0.369 \\ \cline{1-3} 
     400000000 & 0.500 & 0.490 \\ \cline{1-3} 
    \end{tabular}
\end{table}
\clearpage
\subsection{Παραλλαγή με \emph{\en{target parallel for simd}}}
\subparagraph{}
\selectlanguage{english}
\begin{spacing}{0.9}
\begin{lstlisting}[basicstyle=\small, language=C++, caption={\el{Υλοποίηση παραλλαγής με \en{target parallel for simd}}} , frame=tlrb]{Name}
void saxpy(size_t n, float a, const float *x, float *y) {
#pragma omp target parallel for simd map(tofrom: y[0:n]) map(to: x[0:n])
    for (size_t i = 0; i < n; ++i) {
        y[i] = a * x[i] + y[i];
    }
}
\end{lstlisting}
\end{spacing}
\selectlanguage{greek}

\begin{table}[h]
    \centering
    \caption{Καταγραφή χρόνων εκτέλεσης}
    \label{my-label}
    \begin{tabular}{| >{\centering\arraybackslash}p{0.15\textwidth}| 
    >{\centering\arraybackslash}p{0.15\textwidth}|
	>{\centering\arraybackslash}p{0.20\textwidth}|
	>{\centering\arraybackslash}p{0.15\textwidth}|
    >{\centering\arraybackslash}p{0.20\textwidth}|}
    \hline
    \multirow{4}{*}{\textbf{\shortstack{\\Μέγεθος \\ προβλήματος}}} & \multicolumn{2}{|c|}					{\textbf{Χρόνοι εκτέλεσης \en{(sec)}}} \\ \cline{2-3} 
        & \textbf{-Ο0}
        & \textbf{\en{-O0 -fopenmp-simd}}
        & \textbf{-O3} 
        & \textbf{\en{-O3 -fopenmp-simd}}

\\ \hline
     100000    & 0.010 & 0.010 & 0.011 & 0.011 \\ \cline{1-5} 
     1000000   & 0.012 & 0.014 & 0.011 & 0.009 \\ \cline{1-5} 
     10000000  & 0.027 & 0.026 & 0.020 & 0.020 \\ \cline{1-5} 
     100000000 & 0.140 & 0.154 & 0.128 & 0.130 \\ \cline{1-5} 
     200000000 & 0.257 & 0.271 & 0.249 & 0.247 \\ \cline{1-5} 
     300000000 & 0.378 & 0.385 & 0.370 & 0.365 \\ \cline{1-5} 
     400000000 & 0.513 & 0.505 & 0.450 & 0.489 \\ \cline{1-5} 
    \end{tabular}
\end{table}

\clearpage
\subsection{Παραλλαγή με \emph{\en{target teams map}}}
\subparagraph{}
\selectlanguage{english}
\begin{spacing}{0.9}
\begin{lstlisting}[language=C++, caption={\el{Υλοποίηση παραλλαγής με \en{target teams map}}} , frame=tlrb]{Name}
void saxpy(size_t n, float a, const float *x, float *y) {
    #pragma omp target teams map(tofrom: y[0:n]) map(to: x[0:n])
    for (size_t i = 0; i < n; ++i) {
        y[i] = a * x[i] + y[i];
    }
}
\end{lstlisting}
\end{spacing}
\selectlanguage{greek}
\begin{table}[h]
    \centering
    \caption{Καταγραφή χρόνων εκτέλεσης}
    \label{my-label}
    \begin{tabular}{| >{\centering\arraybackslash}p{0.25\textwidth}| 
    >{\centering\arraybackslash}p{0.25\textwidth}|
    >{\centering\arraybackslash}p{0.25\textwidth}|}
    \hline
    \multirow{2}{*}{\textbf{\shortstack{\\Μέγεθος \\ προβλήματος}}} & \multicolumn{2}{|c|}					{\textbf{Χρόνοι εκτέλεσης \en{(sec)}}} \\ \cline{2-3} 
        & \textbf{-Ο0}
        & \textbf{-O3} 

\\ \hline
     100000    & 0.005 & 0.004  \\ \cline{1-3} 
     1000000   & 0.016 & 0.006 \\ \cline{1-3} 
     10000000  & 0.132 & 0.025 \\ \cline{1-3} 
     100000000 & 1.186 & 0.221 \\ \cline{1-3} 
     200000000 & 2.374 & 0.420 \\ \cline{1-3} 
     300000000 & 3.543 & 0.652 \\ \cline{1-3} 
     400000000 & 4.729 & 0.821 \\ \cline{1-3} 
    \end{tabular}
\end{table}

\clearpage
\subsection{Παραλλαγή με \emph{\en{target teams distribute map}}}
\subparagraph{}
\selectlanguage{english}
\begin{spacing}{0.9}
\begin{lstlisting}[language=C++, caption={\el{Υλοποίηση παραλλαγής με \en{target teams distribute map}}} , frame=tlrb]{Name}
void saxpy(size_t n, float a, const float *x, float *y) {
#pragma omp target teams distribute map(from: y[0:n]) map(to: x[0:n])
    for (size_t i = 0; i < n; ++i) {
        y[i] = a * x[i] + y[i];
    }
}

\end{lstlisting}
\end{spacing}
\selectlanguage{greek}
\begin{table}[h]
    \centering
    \caption{Καταγραφή χρόνων εκτέλεσης}
    \label{my-label}
    \begin{tabular}{| >{\centering\arraybackslash}p{0.25\textwidth}| 
    >{\centering\arraybackslash}p{0.25\textwidth}|
    >{\centering\arraybackslash}p{0.25\textwidth}|}
    \hline
    \multirow{2}{*}{\textbf{\shortstack{\\Μέγεθος \\ προβλήματος}}} & \multicolumn{2}{|c|}					{\textbf{Χρόνοι εκτέλεσης \en{(sec)}}} \\ \cline{2-3} 
        & \textbf{-Ο0}
        & \textbf{-O3} 

\\ \hline
     100000    & 0.005 & 0.004 \\ \cline{1-3} 
     1000000   & 0.015 & 0.006 \\ \cline{1-3} 
     10000000  & 0.117 & 0.025 \\ \cline{1-3} 
     100000000 & 1.142 & 0.214 \\ \cline{1-3} 
     200000000 & 2.278 & 0.428 \\ \cline{1-3} 
     300000000 & 3.433 & 0.625 \\ \cline{1-3} 
     400000000 & 4.575 & 0.851 \\ \cline{1-3} 
    \end{tabular}
\end{table}

\clearpage
\subsection{Παραλλαγή με \emph{\en{target teams distribute parallel for map}}}
\subparagraph{}
\selectlanguage{english}
\begin{spacing}{0.9}
\begin{lstlisting}[basicstyle=\footnotesize, language=C++, caption={\el{Υλοποίηση παραλλαγής με \en{target teams distribute parallel for}}} , frame=tlrb]{Name}
void saxpy(size_t n, float a, const float *x, float *y) {
#pragma omp target teams distribute parallel for map(from: y[0:n]) map(to: x[0:n])
    for (size_t i = 0; i < n; ++i) {
        y[i] = a * x[i] + y[i];
    }
}
\end{lstlisting}
\end{spacing}
\selectlanguage{greek}
\begin{table}[h]
    \centering
    \caption{Καταγραφή χρόνων εκτέλεσης}
    \label{my-label}
    \begin{tabular}{| >{\centering\arraybackslash}p{0.25\textwidth}| 
    >{\centering\arraybackslash}p{0.25\textwidth}|
    >{\centering\arraybackslash}p{0.25\textwidth}|}
    \hline
    \multirow{2}{*}{\textbf{\shortstack{\\Μέγεθος \\ προβλήματος}}} & \multicolumn{2}{|c|}					{\textbf{Χρόνοι εκτέλεσης \en{(sec)}}} \\ \cline{2-3} 
        & \textbf{-Ο0}
        & \textbf{-O3} 

\\ \hline
     100000    & 0.010 & 0.011 \\ \cline{1-3} 
     1000000   & 0.016 & 0.011 \\ \cline{1-3} 
     10000000  & 0.023 & 0.020 \\ \cline{1-3} 
     100000000 & 0.139 & 0.127 \\ \cline{1-3} 
     200000000 & 0.257 & 0.250 \\ \cline{1-3} 
     300000000 & 0.389 & 0.369 \\ \cline{1-3} 
     400000000 & 0.511 & 0.490 \\ \cline{1-3} 
    \end{tabular}
\end{table}
\clearpage
\subsection{Παραλλαγή με \emph{\en{target teams distribute simd map}}}
\subparagraph{}
\selectlanguage{english}
\begin{spacing}{0.9}
\begin{lstlisting}[basicstyle=\small, language=C++, caption={\el{Υλοποίηση παραλλαγής με \en{target teams distribute simd map}}} , frame=tlrb]{Name}
void saxpy(size_t n, float a, const float *x, float *y) {
#pragma omp target teams distribute simd map(from: y[0:n]) map(to: x[0:n])
    for (size_t i = 0; i < n; ++i) {
        y[i] = a * x[i] + y[i];
    }
}

\end{lstlisting}
\end{spacing}
\selectlanguage{greek}

\begin{table}[h]
    \centering
    \caption{Καταγραφή χρόνων εκτέλεσης}
    \label{my-label}
    \begin{tabular}{| >{\centering\arraybackslash}p{0.15\textwidth}| 
    >{\centering\arraybackslash}p{0.15\textwidth}|
	>{\centering\arraybackslash}p{0.20\textwidth}|
	>{\centering\arraybackslash}p{0.15\textwidth}|
    >{\centering\arraybackslash}p{0.20\textwidth}|}
    \hline
    \multirow{4}{*}{\textbf{\shortstack{\\Μέγεθος \\ προβλήματος}}} & \multicolumn{2}{|c|}					{\textbf{Χρόνοι εκτέλεσης \en{(sec)}}} \\ \cline{2-3} 
        & \textbf{-Ο0}
        & \textbf{\en{-O0 -fopenmp-simd}}
        & \textbf{-O3} 
        & \textbf{\en{-O3 -fopenmp-simd}}

\\ \hline
     100000    & 0.005 & 0.050 & 0.004 & 0.004 \\ \cline{1-5} 
     1000000   & 0.015 & 0.015 & 0.006 & 0.006 \\ \cline{1-5} 
     10000000  & 0.120 & 0.119 & 0.021 & 0.021 \\ \cline{1-5} 
     100000000 & 1.159 & 1.153 & 0.159 & 0.168 \\ \cline{1-5} 
     200000000 & 2.311 & 2.305 & 0.326 & 0.327 \\ \cline{1-5} 
     300000000 & 3.487 & 3.472 & 0.482 & 0.495 \\ \cline{1-5} 
     400000000 & 4.620 & 4.610 & 0.647 & 0.644 \\ \cline{1-5} 
    \end{tabular}
\end{table}

\clearpage
\subsection{Παραλλαγή με \emph{\en{target teams distribute parallel for simd map}}}
\subparagraph{}
\selectlanguage{english}
\begin{spacing}{0.9}
\begin{lstlisting}[basicstyle=\footnotesize, language=C++, caption={\el{Υλοποίηση παραλλαγής με \en{teams distribute parallel for simd\
			map}}} , frame=tlrb]{Name}
void saxpy(size_t n, float a, const float *x, float *y) {
#pragma omp target teams distribute parallel for simd\
			map(from: y[0:n]) map(to: x[0:n])
    for (size_t i = 0; i < n; ++i) {
        y[i] = a * x[i] + y[i];
    }
}
\end{lstlisting}
\end{spacing}
\selectlanguage{greek}

\begin{table}[h]
    \centering
    \caption{Καταγραφή χρόνων εκτέλεσης}
    \label{my-label}
    \begin{tabular}{| >{\centering\arraybackslash}p{0.15\textwidth}| 
    >{\centering\arraybackslash}p{0.15\textwidth}|
	>{\centering\arraybackslash}p{0.20\textwidth}|}
    \hline
    \multirow{2}{*}{\textbf{\shortstack{\\Μέγεθος \\ προβλήματος}}} & \multicolumn{2}{|c|}					{\textbf{Χρόνοι εκτέλεσης \en{(sec)}}} \\ \cline{2-3} 
        & \textbf{-Ο0}
        & \textbf{-O3} 

\\ \hline
     100000    & 0.010 & 0.010 \\ \cline{1-3} 
     1000000   & 0.012 & 0.010 \\ \cline{1-3} 
     10000000  & 0.025 & 0.021 \\ \cline{1-3} 
     100000000 & 0.146 & 0.127 \\ \cline{1-3} 
     200000000 & 0.263 & 0.246 \\ \cline{1-3} 
     300000000 & 0.389 & 0.371 \\ \cline{1-3} 
     400000000 & 0.519 & 0.489 \\ \cline{1-3} 
    \end{tabular}
\end{table}