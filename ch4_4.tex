\subsection{Παράδειγμα υπολογισμού $\pi$}
\subparagraph{}
Στο επόμενο παράδειγμα ακολουθεί ο υπολογισμός του αριθμού $\pi$.
Το πρόβλημα ανάγεται στον υπολογισμό του παρακάτω ολοκληρώματος, με τη χρήση αριθμητικών μεθόδων: $$\pi = \int_{0}^{1} \frac{4.0} {(1+x^2)}\, dx$$
που υπολογίζεται αριθμητικά ως:
$$\pi \approx \sum_{k=1}^N F(xi)\Delta x$$ Το πρόβλημα δέχεται ως παράμετρο τον αριθμό των βημάτων της αριθμητικής ολοκλήρωσης. Όσο πιο μεγάλος είναι ο αριθμός των βημάτων, τόσο πιο ακριβής είναι και ο υπολογισμός του $\pi$.

\clearpage
\subsubsection{Σειριακή εκτέλεση}
\subparagraph{}
Η σειριακή υλοποίηση του υπολογισμού $\pi$ με χρήση αριθμητικών μεθόδων, αποτελείται από ένα βρόγχο επανάληψης. Σε κάθε επανάληψη του οποίο υπολογίζεται ένα μικρό τμήμα του συνολικού ολοκληρώματος, το ίχνος του οποίο είναι ίσο με $1/num\_steps$, όπως φαίνεται παρακάτω:

\selectlanguage{english}
\begin{lstlisting}[language=C++, caption={\el{Υλοποίηση σειριακής έκδοσης υπολογισμού $\pi$}} , frame=tlrb]{Name}
double pi(long num_steps) {
    int upper_limit = 1;
    double step = upper_limit/(double)num_steps;
    double sum = .0, pi = .0;

    for (int i = 0; i < num_steps; ++i)
    {
        double x = (i + 0.5) * step;
        sum += 4.0 / (1.0 + x*x);
    }
    pi = step * sum;
    
    return pi;
}
\end{lstlisting}
\selectlanguage{greek}

\selectlanguage{greek}

\begin{table}[htbp]
\centering
\captionsetup{justification=raggedright,
singlelinecheck=false
}
\caption{ \emph{Καταγραφή χρόνων εκτέλεσης παραδειγμάτων}}
\def\arraystretch{1.5}
\begin{tabular}{| p{0.35\textwidth} | p{0.35\textwidth}|}
 \textbf{Αριθμός Βημάτων\cellcolor[HTML]{D0D0D0}} & \textbf{Χρόνος Υπολογισμού (\emph{\en{sec}}) }\cellcolor[HTML]{D0D0D0} \\
\hline
 100000000 &  2.812\\
\hline
 200000000 &  5.633\\
\hline
 300000000 &  8.469\\
\hline
 400000000 &  11.592  \\
 \hline
\end{tabular}
\end{table}

\paragraph{Σχόλιο:}
\subparagraph{}
Ο συγκεκριμένος αλγόριθμος λόγω του βρόγχου επανάληψης και του αθροίσματος των δεδομένων σε μια μεταβλητή, επιδέχεται πολλών παραλλαγών που υλοποιούνται στις επόμενες παραγράφους.


\clearpage
\subsubsection{Παραλλαγή 1\textsuperscript{η}}
\subparagraph{}
Στη συγκεκριμένη περίπτωση, ο αλγόριθμος παραλληλοποιείται με τη χρήση της οδηγίας \emph{\en{pragma omp parallel}}. Η επαναλήψεις του βρόγχου διαμοιράζονται στα νήματα και τα απότελεσματα των υπολογισμών του κάθε νήματος αποθηκεύονται στη σχετική θέση ενός διανύσματος. Ο τελικός υπολογισμός γίνεται σειριακά, με τη χρήση του διανύσματος αυτού.
\selectlanguage{english}
\begin{lstlisting}[language=C++, caption={\el{Υπολογισμός παραλλαγής 1}} , frame=tlrb]{Name}
double pi(long num_steps) {
    double pi = .0;
    int num_threads = 0;
#pragma omp parallel shared(num_threads)
    {
            int id = omp_get_thread_num();
            if (id == 0) {
                    num_threads = omp_get_num_threads();
            }
    }
    particle *sum = new particle[num_threads];
    for (int i = 0; i < num_threads; ++i) {
        sum[i].val = 0.0;
    }
    double step= 1.0/(double)num_steps;
#pragma omp parallel 
    {
        int thread_num = omp_get_thread_num();
        int numthreads = omp_get_num_threads();

        int low = num_steps * thread_num / numthreads;
        int high = num_steps * (thread_num + 1)/ numthreads;

        for (int i = low; i < high; ++i) {
            double x = (i + 0.5)*step;
            sum[thread_num].val += 4.0/(1.0 + x*x);
        }
    }

    for (int i = 0; i < num_threads; ++i) {
        pi += sum[i].val * step;
    }
    delete []sum;
    return pi;
}

\end{lstlisting}
\selectlanguage{greek}


\clearpage
\selectlanguage{english}
\begin{lstlisting}[language=C++, caption={\el{Ορισμός δομής } particle} , frame=tlrb]{Name}
	struct particle
{
        double val;
};
`

\end{lstlisting}

\selectlanguage{greek}

\begin{table}[htbp]
\centering
\captionsetup{justification=raggedright,
singlelinecheck=false
}
\caption{ \emph{Καταγραφή χρόνων εκτέλεσης παραδειγμάτων}}
\def\arraystretch{1.5}
\begin{tabular}{| p{0.35\textwidth} | p{0.35\textwidth}|}
 \textbf{Αριθμός Βημάτων\cellcolor[HTML]{D0D0D0}} & \textbf{Χρόνος Υπολογισμού (\emph{\en{sec}}) }\cellcolor[HTML]{D0D0D0} \\
\hline
 100000000 & 2.18\\
\hline
 200000000 &   4.24\\
\hline
 300000000 &  6.44\\
\hline
 400000000 &  8.5  \\
 \hline
\end{tabular}
\end{table}
\subparagraph{}

\paragraph{Σχόλιο:}
\subparagraph{}
Παρατηρείται κακή επίδοση του αλγόριθμου οφείλεται στο φαινόμενο που ονομάζεται \textbf{\emph{\en{false sharing}}} και αναφέρθηκε στα προηγούμενα κεφάλαιο. Για την αποτροπή του, θα πρέπει η δομή \emph{\en{particle}} να είναι μεγαλύτερου μεγέθους από 64 \en{byte}, όσο δηλαδή είναι και το μέγεθος της μνήμης \emph{\en{cache}}.

\clearpage
\subsubsection{Παραλλαγή 2\textsuperscript{η}}
\subparagraph{}
Σε συνέχεια της προηγούμενης παραλλαγής, χρησιμοποιείται ένα τεχνητό κενό ανάμεσα στα στοιχεία \emph{\en{val}} του διανύσματος που αποθηκεύουν τους υπολογισμούς του κάθε νήματος, για την αποφυγή του φαινομένου \emph{\en{false sharing}}. Ο κώδικας υπολογισμού του $\pi$ παραμένει ο ίδιος, όμως η δομή που θα χρησιμοποιηθεί είναι η παρακάτω.
\selectlanguage{english}
\begin{lstlisting}[language=C++, caption={\el{Δομή} particle \el{με τεχνητό κενό}} , frame=tlrb]{Name}
struct particle
{
        double val;
        double pad1;
        double pad2;
        double pad3;
        double pad4;
        double pad5;
        double pad6;
        double pad7;
};
\end{lstlisting}
\selectlanguage{greek}
\selectlanguage{greek}

\begin{table}[htbp]
\centering
\captionsetup{justification=raggedright,
singlelinecheck=false
}
\caption{ \emph{Καταγραφή χρόνων εκτέλεσης παραδειγμάτων}}
\def\arraystretch{1.5}
\begin{tabular}{| p{0.35\textwidth} | p{0.35\textwidth}|}
 \textbf{Αριθμός Βημάτων\cellcolor[HTML]{D0D0D0}} & \textbf{Χρόνος Υπολογισμού (\emph{\en{sec}}) }\cellcolor[HTML]{D0D0D0} \\
\hline
 100000000 & 0.19\\
\hline
 200000000 &  0.36\\
\hline
 300000000 & 0.53 \\
\hline
 400000000 &  0.7\\
 \hline
\end{tabular}
\end{table}
\paragraph{Σχόλιο:}
\subparagraph{}
Παρατηρείται σημαντική βελτίωση σε σύγκριση με την παραλλαγή 1. Το μέγεθος της δομής \emph{\en{particle}} είναι ίσο με 64 \en{bytes}, όσο δηλαδή είναι και το μέγεθος της μνήμης \emph{\en{cache}} στη συγκεκριμένη αρχιτεκτονική.

\clearpage
\subsubsection{Παραλλαγή 3\textsuperscript{η}}
\subparagraph{}
Στη συγκεκριμένη παραλλαγή γίνεται χρήση της οδηγίας \textbf{\en{\emph{pragma omp atomic}}} που εξασφαλίζει την αποφυγή του φαινομένου \emph{\en{race condition}}, που προκαλείται όταν πολλά νήματα τροποποιούν την ίδια θέση διεύθυνση μνήμης ταυτόχρονα.

\selectlanguage{english}
\begin{lstlisting}[language=C++, caption={\el{Υλοποίηση παραλλαγής 3}} , frame=tlrb]{Name}
double pi(long num_steps) {
    int nthreads = 0;
    double pi = .0;
    double step= 1.0/(double)num_steps;
#pragma omp parallel
    {
        int id = omp_get_thread_num();
        int nthrds = omp_get_num_threads();
        double sum = 0.0, x = 0.0;

        if (id == 0) nthreads = nthrds;    

        for (int i = id; i < num_steps; i += nthreads) {
            x = (i + 0.5)*step;
            sum += 4.0/(1.0 + x*x);
        }
        sum *= step;
#pragma omp atomic
        pi += sum;
    }

    return pi;
}      
\end{lstlisting}
\selectlanguage{greek}

\paragraph{Σχόλιο:}
\subparagraph{}
Η χρήση της οδηγίας \emph{\en{atomic}} διευκολύνει την υλοποίηση του αλγορίθμου καθώς δεν απαιτείται η δημιουργία διανύσματος μεταβλητών, όπου κάθε νήμα θα αποθηκεύει τα αποτελέσματα των υπολογισμών σε συγκεκριμένη θέση μνήμης.

\selectlanguage{greek}

\begin{table}[htbp]
\centering
\captionsetup{justification=raggedright,
singlelinecheck=false
}
\caption{ \emph{Καταγραφή χρόνων εκτέλεσης παραδειγμάτων}}
\def\arraystretch{1.5}
\begin{tabular}{| p{0.35\textwidth} | p{0.35\textwidth}|}
 \textbf{Αριθμός Βημάτων\cellcolor[HTML]{D0D0D0}} & \textbf{Χρόνος Υπολογισμού (\emph{\en{sec}}) }\cellcolor[HTML]{D0D0D0} \\
\hline
 100000000 & 0.207 \\
\hline
 200000000 & 0.387 \\
\hline
 300000000 & 0.579 \\
\hline
 400000000 & 0.761 \\
 \hline
\end{tabular}
\end{table}

\clearpage
\subsubsection{Παραλλαγή 4\textsuperscript{η}}
\subparagraph{}
Η παραλλαγή 4 υλοποιήθηκε με τη χρήση της φράσης \textbf{\emph{\en{reduction}}}.

\selectlanguage{english}
\begin{lstlisting}[language=C++, caption={\el{Υλοποίηση παραλλαγής 4}} , frame=tlrb]{Name}
    double pi(long num_steps, int num_threads) {
    double pi = .0;
    double step= 1.0/(double)num_steps;
    double sum = 0.0;
    omp_set_num_threads(num_threads);

#pragma omp parallel
    {
        double x = 0.0;

#pragma omp for reduction(+:sum)
        for (int i = 0; i < num_steps; i++) {
            x = (i + 0.5)*step;
            sum += 4.0/(1.0 + x*x);
        }
    }
    pi = step * sum;

    return pi;
}

\end{lstlisting}
\selectlanguage{greek}

\selectlanguage{greek}

\begin{table}[htbp]
\centering
\captionsetup{justification=raggedright,
singlelinecheck=false
}
\caption{ \emph{Καταγραφή χρόνων εκτέλεσης παραδειγμάτων}}
\def\arraystretch{1.5}
\begin{tabular}{| p{0.35\textwidth} | p{0.35\textwidth}|}
 \textbf{Αριθμός Βημάτων\cellcolor[HTML]{D0D0D0}} & \textbf{Χρόνος Υπολογισμού (\emph{\en{sec}}) }\cellcolor[HTML]{D0D0D0} \\
\hline
 100000000 &  0.027 \\
\hline
 200000000 &  0.052 \\
\hline
 300000000 & 0.071 \\
\hline
 400000000 & 0.0901 \\
 \hline
\end{tabular}
\end{table}

\clearpage
\subsubsection{Παραλλαγή 5\textsuperscript{η}}
\subparagraph{}

Η παραλλαγή 5 υλοποιήθηκε με τη χρήση διεργασιών. Οπως αναφέρθηκε στα προηγούμενα κεφάλαια, η κατασκευή των διεργασιών γίνεται από ένα μοναδικό νήμα, ενώ η εκτέλεση τους από πολλαπλά νήματα ταυτόχρονα.
\selectlanguage{english}
\begin{lstlisting}[language=C++, caption={\el{Δημιουργία διεργασιών}} , frame=tlrb]{Name}
int main(int argc, char **argv) {
    Opts o;
    parseArgs(argc, argv, o);
    auto seconds = omp_get_wtime();
    double step = 1.0/(double)o.num_steps;
    double sum = 0.0;
    double p = 0.0;
#pragma omp parallel
    {
#pragma omp single
        sum = pi_comp(0, o.num_steps, step);
    }
    p = step * sum;
	double time_elapsed = omp_get_wtime() - seconds;
    std::cout << "Elapsed Time: " << time_elapsed << std::endl;
    std::cout << "pi Value: " << p << std::endl;
    return 0;
}

\end{lstlisting}
\selectlanguage{greek}

\begin{table}[htbp]
\centering
\captionsetup{justification=raggedright,
singlelinecheck=false
}
\caption{ \emph{Καταγραφή χρόνων εκτέλεσης παραδειγμάτων}}
\def\arraystretch{1.5}
\begin{tabular}{| p{0.35\textwidth} | p{0.35\textwidth}|}
 \textbf{Αριθμός Βημάτων\cellcolor[HTML]{D0D0D0}} & \textbf{Χρόνος Υπολογισμού (\emph{\en{sec}}) (\emph{\en{ MINBLK: 10000000}}) }\cellcolor[HTML]{D0D0D0} \\
\hline
 100000000 &  0.372117\\
\hline
 200000000 &   0.736442 \\
\hline
 300000000 &  1.09589\\
\hline
 400000000 &  1.46092\\
 \hline
 \end{tabular}
\end{table}

\clearpage
\selectlanguage{english}
\begin{lstlisting}[language=C++, caption={\el{Υλοποίηση υπολογισμού $\pi$ με διεργασίες}} , frame=tlrb]{Name}
#define MIN_BLK 1000000

double pi_comp(int Nstart, int Nfinish, double step) {
    double x = 0.0;
    double sum = 0.0, sum1 = 0.0, sum2 = 0.0;

    if (Nfinish - Nstart < MIN_BLK) {
        for (int i = Nstart; i < Nfinish; ++i) {
            x = (i + 0.5) * step;
            sum += 4.0/(1.0 + x*x);
        }
    } else {
        int iblk = Nfinish-Nstart;
#pragma omp task shared(sum1)
        sum1 = pi_comp(Nstart, Nfinish-iblk/2, step);
#pragma omp task shared(sum2)
        sum2 = pi_comp(Nfinish-iblk/2, Nfinish, step);
#pragma omp taskwait
        sum = sum1 + sum2;
    }

    return sum;
}
\end{lstlisting}
\selectlanguage{greek}

\begin{table}[htbp]
\centering
\captionsetup{justification=raggedright,
singlelinecheck=false
}
\caption{ \emph{Καταγραφή χρόνων εκτέλεσης παραδειγμάτων}}
\def\arraystretch{1.5}
\begin{tabular}{| p{0.35\textwidth} | p{0.35\textwidth}|}
 \textbf{Αριθμός Βημάτων\cellcolor[HTML]{D0D0D0}} & \textbf{Χρόνος Υπολογισμού (\emph{\en{sec}}) 
 (\en{\emph{MINBLK: 50000000}}) }\cellcolor[HTML]{D0D0D0} \\
\hline
 100000000 &  0.278347\\
\hline
 200000000 &   0.65436\\
\hline
 300000000 &   0.973535 \\
\hline
 400000000 &   1.45033 \\
  \hline
   \end{tabular}
\end{table}


\clearpage
\subsubsection{Παραλλαγή 6\textsuperscript{η}}
\subparagraph{}

\selectlanguage{english}
\begin{lstlisting}[language=C++, caption={\el{Υλοποίηση παραλλαγής 6}} , frame=tlrb]{Name}
double pi(long num_steps) {
    double dH = 1.0/(double)num_steps;
    double dX, dSum = 0.0;
    
#pragma omp parallel for simd private(dX) \
    reduction(+:dSum) schedule(simd:static)
    for (int i = 0; i < num_steps; i++) { 
        dX = dH * ((double) i  + 0.5);
        dSum += (4.0 / (1.0 + dX * dX));
    } // End parallel for simd region
    
    return dH * dSum;   
}

\end{lstlisting}
\selectlanguage{greek}

\begin{table}[htbp]
\centering
\captionsetup{justification=raggedright,
singlelinecheck=false
}
\caption{ \emph{Καταγραφή χρόνων εκτέλεσης παραδειγμάτων}}
\def\arraystretch{1.5}
\begin{tabular}{| p{0.35\textwidth} | p{0.35\textwidth}|}
 \textbf{Αριθμός Βημάτων\cellcolor[HTML]{D0D0D0}} & \textbf{Χρόνος Υπολογισμού (\emph{\en{sec}}) 
 (\en{\emph{MINBLK: 50000000}}) }\cellcolor[HTML]{D0D0D0} \\
\hline
 100000000 &  0.207\\
\hline
 200000000 &   0.371\\
\hline
 300000000 &   0.546 \\
\hline
 400000000 &   1.722 \\
  \hline
   \end{tabular}
\end{table}


\clearpage
\subsubsection{Παραλλαγή 7\textsuperscript{η}}
\subparagraph{}
\selectlanguage{english}
\begin{lstlisting}[language=C++, caption={\el{Υλοποίηση παραλλαγής 7}} , frame=tlrb]{Name}
double pi(long num_steps) {
    double dH = 1.0/(double)num_steps;
    double dX = 0.0, dSum = 0.0;

#pragma omp target teams distribute map(tofrom: dSum),\
                                    map(to:dX,  dH, num_steps)\
                                    reduction(+:dSum)
    for (int i = 0; i < num_steps; i++) {
        dX = dH * ((double) i  + 0.5);
        dSum += (4.0 / (1.0 + dX * dX));
    } // End parallel for simd region

    return dH * dSum;
}

\end{lstlisting}
\selectlanguage{greek}

\begin{table}[htbp]
\centering
\captionsetup{justification=raggedright,
singlelinecheck=false
}
\caption{ \emph{Καταγραφή χρόνων εκτέλεσης παραδειγμάτων}}
\def\arraystretch{1.5}
\begin{tabular}{| p{0.3\textwidth} | p{0.3\textwidth}|p{0.3\textwidth}| }
 \textbf{Αριθμός Βημάτων\cellcolor[HTML]{D0D0D0}} & \textbf{Χρόνος Υπολογισμού (\emph{\en{sec}}) }\cellcolor[HTML]{D0D0D0} & \textbf{Αποτέλεσμα Υπολογισμού (\emph{\en{sec}}) }\cellcolor[HTML]{D0D0D0}  \\
\hline
 100000000 &  5.540  & 3.100  \\
\hline
 200000000 &  10.178 & 3.099 \\
\hline
 300000000 &  14.757 & 3.098 \\
\hline
 400000000 &  19.432 & 3.098 \\
 \hline
\end{tabular}
\end{table}

