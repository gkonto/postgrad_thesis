\begin{center}
\selectlanguage{greek}

\textsc{ ΠΑΝΕΠΙΣΤΗΜΙΟ ΜΑΚΕΔΟΝΙΑΣ\\[0.3 cm]
ΠΡΟΓΡΑΜΜΑ ΜΕΤΑΠΤΥΧΙΑΚΩΝ ΣΠΟΥΔΩΝ\\[0.3 cm]
ΤΜΗΜΑΤΟΣ ΕΦΑΡΜΟΣΜΕΝΗΣ ΠΛΗΡΟΦΟΡΙΚΗΣ}\\[2.5 cm]
{ \large ΠΑΡΑΛΛΗΛΟΣ ΠΡΟΓΡΑΜΜΑΤΙΣΜΟΣ ΜΕ ΧΡΗΣΗ \en{OpenMP}\\[0.4 cm] } Διπλωματική Εργασία\\[1 cm]
του\\[0.5 cm]
\large
Κοντογιάννη Γεώργιου
\begin{minipage}{0.4\textwidth}
\end{minipage}
\vfill
{\large Θεσσαλονίκη, Φεβρουάριος 2021}

 \end{center}
 
\pagenumbering{gobble}
\newpage
\mbox{}


\newpage
\pagenumbering{roman}
\setcounter{page}{3} 

 \begin{center}
{\large {ΠΑΡΑΛΛΗΛΟΣ ΠΡΟΓΡΑΜΜΑΤΙΣΜΟΣ ΜΕ ΧΡΗΣΗ \en{OpenMP}}}\\[2 cm]
Κοντογιάννης Γεώργιος\\[0.5 cm]
Δίπλωμα Πολιτικού Μηχανικού, ΑΠΘ, 2016\\[2 cm]
Διπλωματική Εργασία\\[0.5 cm]
υποβαλλόμενη για τη μερική εκπλήρωση των απαιτήσεων του\\[0.5 cm]
ΜΕΤΑΠΤΥΧΙΑΚΟΥ ΤΙΤΛΟΥ ΣΠΟΥΔΩΝ ΣΤΗΝ ΕΦΑΡΜΟΣΜΕΝΗ ΠΛΗΡΟΦΟΡΙΚΗ\\[2 cm]
\begin{flushleft}
Επιβλέπων Καθηγητής\\
Μαργαρίτης Κωνσταντίνος
\vfill
Εγκρίθηκε από την τριμελή εξεταστική επιτροπή την ηη/μμ/εεεε\\[0.5 cm]
\begin{tabular}{  p{\dimexpr 0.3333\linewidth-2\tabcolsep} 
                   p{\dimexpr 0.3333\linewidth-2\tabcolsep} p{\dimexpr 0.3333\linewidth-2\tabcolsep}
                   } Ονοματεπώνυμο 1 & Ονοματεπώνυμο 2  & Ονοματεπώνυμο 3 \\[1 cm]
\dotfill & \dotfill  & \dotfill \\
\end{tabular}\\[2 cm]
Κοντογιάννης Γεώργιος \\[0.5 cm]
\begin{tabular}{  p{\dimexpr 0.3333\linewidth-2\tabcolsep}   }
\dotfill
\end{tabular}\\[1 cm]
\end{flushleft}
\end{center}
  
%\setstretch{1.5}
%\setstretch{0.5}

\clearpage
\begin{small}
\begin{flushleft}
\
\vfil
\emph{Η σύνταξη της παρούσας εργασίας έγινε στο   \begin{LARGE}\en{\LaTeX}\end{LARGE}}
\end{flushleft}
\vfil
\end{small}



\clearpage
\begin{flushleft}
{\large \textbf{Περίληψη}}\\[0.5 cm]
\end{flushleft}

\subparagraph{}
Αντικείμενο της διπλωματικής εργασίας είναι η μελέτη του \en{OpenMP}, ενός προτύπου
παράλληλου προγραμματισμού, που δίνει στο χρήστη τη δυνατότητα αναπτύξης παράλληλων προγραμμάτων για
συστήματα μοιραζόμενης μνήμης, τα οποία  είναι ανεξάρτητα από τη αρχιτεκτονική του συστήματος και
έχουν μεγάλη ικανότητα κλιμάκωσης\cite{pdplab}.

Σκοπός της εργασίας είναι η συνοπτική ανακεφαλαίωση των βασικών χαρακτηριστικών των παλαιών εκδόσεων
(\en{OpenMP 2.5}), η μελέτη και περιγραφή των κύριων χαρακτηριστικών των νεότερων (3.0 και 4.5)
καθώς και η υλοποίηση αλγορίθμων σειριακά και παράλληλα εκτελέσιμων, με σκοπό τη συγκριτική μελέτη
των παραλλαγών του κάθε προβλήματος για την εξαγωγή συμπερασμάτων. Για την παράλληλη υλοποίηση θα
γίνει χρήση της Διεπαφής Προγραμματισμού Εφαρμογών \en{(Application Programming Interface {-} API)
OpenMP}, με χαρακτηριστικά που εισήχθησαν στις εκδόσεις \en{OpenMP} 3.0 που δημοσιεύθηκε το 2008 και
\en{OpenMP} 4.5 που δημοσιεύθηκε 2015 καθώς και χαρακτηριστικά παλαιότερων
εκδόσεων\cite{thenextstep59}.

Τον Μάιο του 2008 κυκλοφόρησε η έκδοση του \en{OpenMP} 3.0. Στην κυκλοφορία συμπεριλήφθηκε για
πρώτη φορά η έννοια των διεργασιών \en{(Tasking)} αλλά και βελτιώσεις στην υποστήριξη της διεπαφής
μέσω της \en{C++}. Αποτελεί την πρώτη ενημέρωση μετά την έκδοση 2.5 με σημαντικές βελτιώσεις. Το
2011 κυκλοφόρησε η \en{OpenMP} 3.1 χωρίς αξιοσημείωτες νέες προσθήκες. Νέα χαρακτηριστικά ωστόσο,
εισήχθησαν στο \en{OpenMP} 4.0 που κυκλοφόρησε τον Ιούλιο του 2013, όπου έγινε υποστήριξη της
αρχιτεκτονικής \en{cc-NUMA}, του ετερογενούς προγραμματισμού, της διαχείρισης σφαλμάτων στην περιοχή
παράλληλου κώδικα και της διανυσματικοποίησης μέσω \en{SIMD}. Τον Ιούλιο του 2015 σημαντική βελτίωση
έγινε στα παραπάνω χαρακτηριστικά με την έκδοση \en{OpenMP} 4.5\cite{thenextstep20}.

Τα προαναφερθέντα χαρακτηριστικά χρησιμοποιήθηκαν για την υλοποίηση αλγορίθμων σε διάφορες
παραλλαγές, με σκοπό τη συγκριτική μελέτη τους για την εξαγωγή συμπερασμάτων αναφορικά με τη
βελτίωση της απόδοσης σε σχέση με τη σειριακή υλοποίηση, τη μεταξύ τους σύγκριση καθώς επίσης, την
αξιολόγηση της ευχρηστίας της υλοποίησής τους. Στόχος της έρευνας είναι η συλλογή και καταγραφή
παρατηρήσεων που προκύπτουν σε κάθε υλοποίηση, για την καλύτερη κατανόηση των εννοιών του παράλληλου
προγραμματισμου.

Για την ικανότητα παραλληλοποίησης του κώδικα, απαιτούνται αλγόριθμοι που αποτελούνται απο διεργασίες
ανεξάρτητες μεταξύ τους και ικανές να εκτελεστούν ταυτόχρονα, σε διαφορετικούς επεξεργαστές. Τέτοιοι
αλγόριθμοι είναι οι εξής:

\begin{itemize}
    \item μετασχηματισμός \textbf{\en{Fourier - (Discrete Fourier Transform)}}, 
    \item ταξινόμηση \textbf{\en{Mergesort}},
    \item ταξινόμηση \textbf{\en{Quicksort}},
    \item διεργασία \textbf{\en{Producer-Consumer}},
    \item υπολογισμός $\pi$, 
    \item υπολογισμός πρώτων αριθμών,
    \item υπολογισμός εσωτερικού γινομένου,
    \item πολλαπλασιασμός πινάκων,
    \item \en{Single precision A X plus Y - \textbf{SAXPY}},
    \item \en{\textbf{Linked list} traversal}
\end{itemize}

Υλοποιήσεις των παραπάνω προβλημάτων χρησιμοποιούνται στα πλαίσια της παρούσας εργασίας.
Δημιουργήθηκαν παραλλαγές του κάθε προβλήματος που χρησιμοποιούν την κεντρική μονάδα επεξεργασίας
(\textbf{\en{CPU}}) για σειριακή και παράλληλη εκτέλεση. 'Οπου είναι εφικτό υλοποιείται παραλλαγή
για εκτελεσή μέσω της μονάδας επεξεργασίας της κάρτας γραφικών (\textbf{\en{GPU}}). Οι χρονικές
καταγραφές συγκρίνονται μεταξύ τους. Ακόμη, γίνεται αξιολόγηση της ευχρηστίας για την υλοποίηση της
κάθε παραλλαγής καθώς και καταγραφή παρατηρήσεων που τυχόν προέκυψαν.\\[1 cm]

\indent \textbf{Λέξεις Κλειδιά:} Παράλληλος Προγραμματισμός, Παραλληλοποίηση, \en{OpenMP,
accelerators, offloading, vectorization, SIMD, OpenMP4.5, UDRs}

\clearpage
\selectlanguage{english}
\begin{flushleft}

{\large \textbf{\en{Abstract}}}\\[0.5 cm]
\end{flushleft}
\indent \textbf{\en{Keywords:}}

\clearpage
\selectlanguage{greek}
\begin{flushleft}
{\large \textbf{Ευχαριστίες}}\\[0.5 cm]
\end{flushleft}
\subparagraph{}
Εκφράζω τις θερμές μου ευχαριστίες στον επιβλέποντα καθηγητή κ. Κωνσταντίνο Μαργαρίτη, για την
ουσιαστική του συνεισφορά στην εκπόνηση της παρούσας εργασίας.

\clearpage
\singlespacing
\tableofcontents
%\setstretch{1.5}

\selectlanguage{greek}
\renewcommand{\listfigurename}{Κατάλογος Εικόνων (αν υπάρχουν)}
\clearpage
\listoffigures

\selectlanguage{greek}
\renewcommand{\listtablename}{Κατάλογος Πινάκων (αν υπάρχουν)}
\clearpage
\listoftables

\clearpage
\begin{flushleft}
\lstlistoflistings
\end{flushleft}

\clearpage
\setcounter{page}{1}
\pagenumbering{arabic}