\begin{center}
\selectlanguage{greek}

\textsc{ ΠΑΝΕΠΙΣΤΗΜΙΟ ΜΑΚΕΔΟΝΙΑΣ\\[0.3 cm]
ΠΡΟΓΡΑΜΜΑ ΜΕΤΑΠΤΥΧΙΑΚΩΝ ΣΠΟΥΔΩΝ\\[0.3 cm]
ΤΜΗΜΑΤΟΣ ΕΦΑΡΜΟΣΜΕΝΗΣ ΠΛΗΡΟΦΟΡΙΚΗΣ}\\[2.5 cm]
{ \large 
ΠΑΡΑΛΛΗΛΟΣ ΠΡΟΓΡΑΜΜΑΤΙΣΜΟΣ ΜΕ ΧΡΗΣΗ \en{OpenMP}\\[0.4 cm] }
Διπλωματική Εργασία\\[1 cm]
του\\[0.5 cm]
\large
Κοντογιάννη Γεώργιου
\begin{minipage}{0.4\textwidth}
\end{minipage}
\vfill
{\large Θεσσαλονίκη, Οκτώβριος 2020}

 \end{center}
 
\pagenumbering{gobble}
\newpage
\mbox{}


\newpage
\pagenumbering{roman}
\setcounter{page}{3} 

 \begin{center}
{\large {ΠΑΡΑΛΛΗΛΟΣ ΠΡΟΓΡΑΜΜΑΤΙΣΜΟΣ ΜΕ ΧΡΗΣΗ \en{OpenMP}}}\\[2 cm]
Κοντογιάννης Γεώργιος\\[0.5 cm]
Δίπλωμα Πολιτικού Μηχανικού, ΑΠΘ, 2016\\[2 cm]
Διπλωματική Εργασία\\[0.5 cm]
υποβαλλόμενη για τη μερική εκπλήρωση των απαιτήσεων του\\[0.5 cm]
ΜΕΤΑΠΤΥΧΙΑΚΟΥ ΤΙΤΛΟΥ ΣΠΟΥΔΩΝ ΣΤΗΝ ΕΦΑΡΜΟΣΜΕΝΗ ΠΛΗΡΟΦΟΡΙΚΗ\\[2 cm]
\begin{flushleft}
Επιβλέπων Καθηγητής\\
Μαργαρίτης Κωνσταντίνος
\vfill
Εγκρίθηκε από την τριμελή εξεταστική επιτροπή την ηη/μμ/εεεε\\[0.5 cm]
\begin{tabular}{  p{\dimexpr 0.3333\linewidth-2\tabcolsep} 
                   p{\dimexpr 0.3333\linewidth-2\tabcolsep} 
                   p{\dimexpr 0.3333\linewidth-2\tabcolsep}  }
Ονοματεπώνυμο 1 & Ονοματεπώνυμο 2  & Ονοματεπώνυμο 3 \\[1 cm]
\dotfill & \dotfill  & \dotfill \\
\end{tabular}\\[2 cm]
Κοντογιάννης Γεώργιος \\[0.5 cm]
\begin{tabular}{  p{\dimexpr 0.3333\linewidth-2\tabcolsep}   }
\dotfill
\end{tabular}\\[1 cm]
\end{flushleft}
\end{center}
  
\setstretch{1.5}

\clearpage
\begin{small}
\begin{flushleft}
\
\vfil
\emph{Η σύνταξη της παρούσας εργασίας έγινε στο   \begin{LARGE}\en{\LaTeX}\end{LARGE}}
\end{flushleft}
\vfil
\end{small}



\clearpage
\begin{flushleft}
{\large \textbf{Περίληψη}}\\[0.5 cm]
\end{flushleft}

\subparagraph{}
Αντικείμενο της παρούσας διπλωματικής εργασίας είναι η μελέτη του \en{OpenMP}, ενός πρότυπου παράλληλου προγραμματισμού, που δίνει στο χρήστη τη δυνατότητα αναπτύξης παράλληλων προγραμμάτων για συστήματα μοιραζόμενης μνήμης, τα οποία  είναι ανεξάρτητα από τη συγκεκριμένη αρχιτεκτονική και έχουν μεγάλη ικανότητα κλιμάκωσης\cite{pdplab}.

Σκοπός της εργασίας είναι η μελέτη και συνοπτική περιγραφή των κύριων χαρακτηριστικών του \en{OpenMP 2.5} αλλά και των νεότερων εκδόσεων 3.0 και 4.5 και η υλοποίηση αλγορίθμων σειριακά και παράλληλα εκτελέσιμων, με σκοπό τη συγκριτική μελέτη της απόδοσής τους. Για την παράλληλη υλοποίηση θα γίνει χρήση της Διεπαφής Προγραμματισμού Εφαρμογών \en{(Application Programming Interface {-} API) OpenMP}, με χαρακτηριστικά που εισήχθησαν στις εκδόσεις \en{OpenMP} 3.0 που δημοσιεύθηκε το 2008 και \en{OpenMP} 4.5 που δημοσιεύθηκε 2015. Χρησιμοποιήθηκαν επίσης χαρακτηριστικά παλαιότερων εκδόσεων\cite{thenextstep59}.

Τον Μαιο του 2008 κυκλοφόρησαν οι προδιαγραφές του \en{OpenMP} 3.0 με την εισαγωγή των διεργασιών \en{(Tasking)} αλλά και βελτιώσεις στη \en{C++}. Αυτή ήταν η πρώτη ενημέρωση από την έκδοση 2.5 με σημαντικές βελτιώσεις. Το 2011 κυκλοφόρησε το \en{OpenMP} 3.1 χωρίς καινούργιο χαρακτηριστικά. Νέα λειτουργικότητα υλοποιήθηκε στο \en{OpenMP} 4.0 που κυκλοφόρησε τον Ιούλιο του 2013, όπου έγινε υποστήριξη της αρχιτεκτονικής \en{cc-NUMA}, του ετερογενούς προγραμματισμού, της διαχείρισης σφαλμάτων στο μπλοκ παράλληλου κώδικα και της διανυσματικοποίησης μέσω \en{SIMD}. Τον Ιούλιο του 2015 σημαντική βελτίωση έγινε στα παραπάνω χαρακτηριστικά με την έκδοση \en{OpenMP} 4.5\cite{thenextstep20}.

Τα προαναφερθέντα χαρακτηριστικά χρησιμοποιήθηκαν για την υλοποίηση των αλγορίθμων 
με διαφορετικές εναλλακτικές μεθόδους, με στόχο τη συγκριτική μελέτη τους για την εξαγωγή συμπερασμάτων αναφορικά με τη βελτίωση της απόδοσης σε σχέση με τη σειριακή υλοποίηση αλλά και τη μεταξύ τους σύγκριση καθώς επίσης, και αξιολόγηση της ευχρηστίας της υλοποίησής τους. Στόχος της έρευνας είναι να βρεθούν οι καλύτερες υλοποιήσεις των αλγορίθμων με την επίτευξη της μέγιστης αξιοποίησης της χρήσης \en{CPU} και/ή \en{GPU}. Ακόμη, γίνεται καταγραφή και αναφορά των προβλημάτων που μπορεί να προκύψουν για κάποια υλοποίηση.

Για την παραλληλοποίηση κώδικα, απαιτείται η σχεδίαση με τέτοιο τρόπο ώστε να παράγεται ένας μεγάλος αριθμός παράλληλων λειτουργιών που εκτελούνται από διαφορετικούς επεξεργαστές. Οι  αλγόριθμοι που χρησιμοποιήθηκαν στην παρούσα εργασία περιέχουν ένα μεγάλο αριθμών λειτουργιών, ικανών να εκτελεστούν παράλληλα. 

Τα βασικότερα παραδείγματα που χρησιμοποιήθηκαν είναι:
\begin{itemize}
    \item μετασχηματισμός \en{Fourier}, 
    \item \en{mergesort},
    \item υπολογισμός $\pi$, 
    \item πολλαπλασιασμός πινάκων,
    \item απλή εξίσωση διάδοσης θερμότητας,
    \item παραγοντοποίηση \en{cholensky}.
\end{itemize}

     
Για να υπάρχει άμεση σύγκριση των αποτελεσμάτων ο βασικός κορμός υλοποίησης είναι ο ίδιος για κάθε εφαρμογή, και οι χρονικές καταγραφές έγιναν σε συγκεκριμένα τμήματα του κώδικα. Οι παραλλαγές του κάθε αλγόριθμοι χρησιμοποιούν την \en{CPU} με απλή εκτέλεση χωρίς παραλληλοποίηση και με παράλληλη εκτέλεση. Οπου είναι εφικτό ο αλγόριθμους υλοποιείται  για εκτέλεση στην \en{GPU} για την επίλυση του. Οι χρονικές καταγραφές συγκρίνονται μεταξύ τους για την εξαγωγή συμπερασμάτων. Ακόμη, γίνεται αξιολόγηση της ευχρηστίας για την υλοποίηση της κάθε παραλλαγής αλλά και προβλημάτων που προέκυψαν.\\[1 cm]

\indent \textbf{Λέξεις Κλειδιά:}
Παράλληλος Προγραμματισμός, Παραλληλοποίηση, \en{OpenMP, accelerators, offloading, vectorization, SIMD, OpenMP4.5, UDRs}

\clearpage
\selectlanguage{english}
\begin{flushleft}

{\large \textbf{Abstract}}\\[0.5 cm]
\end{flushleft}
[Enter abstract here.]\\[1 cm]
\indent \textbf{Keywords:}

\clearpage
\selectlanguage{greek}
\begin{flushleft}
{\large \textbf{Ευχαριστίες}}\\[0.5 cm]
\end{flushleft}
\subparagraph{}
Εκφράζω τις θερμές μου ευχαριστίες στον επιβλέποντα καθηγητή κ. Κωνσταντίνο Μαργαρίτη, για την ουσιαστική του συνεισφορά στην εκπόνηση της παρούσας εργασίας.

\clearpage
\singlespacing
\tableofcontents
\setstretch{1.5}

\selectlanguage{greek}
\renewcommand{\listfigurename}{Κατάλογος Εικόνων (αν υπάρχουν)}
\clearpage
\listoffigures

\selectlanguage{greek}
\renewcommand{\listtablename}{Κατάλογος Πινάκων (αν υπάρχουν)}
\clearpage
\listoftables

\clearpage
\begin{flushleft}
\lstlistoflistings
\end{flushleft}

\clearpage
\setcounter{page}{1}
\pagenumbering{arabic}