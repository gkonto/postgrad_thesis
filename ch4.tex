\subsection{Υπολογισμός εσωτερικού γινόμενου}
\subparagraph{}

Δες \emph{\en{pdf}} σελ 304

\subsection{Πρόβλημα προσπέλασης συνδεδεμένης λίστας}
\subparagraph{}
Το παρακάτω παράδειγμα διατρέχει μια συνδεδεμένη λίστα, όπου κάθε κόμβος της θα χρησιμοποιεί τα δεδομένα του για την εκτέλεση μια μεμονομένης διεργασίας, ξεχωριστής από τους υπόλοιπους κόμβους. Στην προκειμένη περίπτωση τα δεδομένα των κόμβων είναι ένας ακέραιος αριθμός.


Κάθε κόμβος της λίστας αρχικοποιείται με τον αριθμό \textbf{33}. Η κάθε εργασία θα έχει ως στόχο τον υπολογισμό του αριθμού της ακολουθίας \emph{\en{fibonacci}}.
Το μέγεθος του προβλήματος επηρεάζεται από τον αριθμό των συνολικών κόμβων της λίστας και όχι απο την τιμή που πρέπει να υπολογιστεί για την ακολουθία \emph{fibonacci}η οποία είναι πάντα σταθερή και ίση με \textbf{33}.
Η ανεξαρτησία των κόμβων διευκολύνει τον παράλληλο υπολογισμό. Ωστόσο, οι συνδεδεμένες λίστες διετρέχονται σειριακά, καταστώντας τον παραλληλισμό μή εφικτό.
\ \\
\begin{center}

\selectlanguage{english}
\begin{lstlisting}[ tabsize = 2, basicstyle=\small, language=C++, caption={\el{Συνάρτηση αρχικοποίησης κόμβων}}, frame = tb]{Name}
		
Node *init_nodes(int num, int value) {
     Node *head = new Node(value);
     Node *temp = nullptr;
 
     Node *p = head;
     for (int i = 0; i < num; ++i) {
         temp =  new Node(value);
         p->next_ = temp;
         p = temp;
         p->data_ = value;
     }
     p->next_ = nullptr;
     return head;
 }
\end{lstlisting}
\end{center}
\clearpage
\selectlanguage{english}
\begin{lstlisting}[ tabsize = 2, basicstyle=\small, language=C++, caption={\el{Δομή κόμβου συνδεδεμένης λίστας}}, frame = tb]{Name}
 struct Node {
     Node *next_ = nullptr; // pointer to the next node
     int data_ = 0;         // the stored data is an integer in this example
 };
\end{lstlisting}
\selectlanguage{greek}

Λύση στο πρόβλημα, αποτελεί η αρχική προσπέλαση της λίστας με σκοπό την εισαγωγή των κόμβων της σε διάνυσμα και στη συνέχεια η παραλληλη εκτέλεση μέσω οδηγίας διαμοιρασμού εργασίας βρόγχου - \emph{\en{for}}.
Αυτή η λύση ωστόσο προκαλεί τα παρακάτω προβλήματα:
\begin{enumerate}
\item Γίνεται αρχικά μία προσπέλαση σε όλους τους κόμβους της λιστας, με σκοπό την αποθήκευση τους σε ένα σειριακό μέσο αποθήκευσης.
\item Σε περίπτωση που ο αριθμός των κόμβων της λίστας δεν ειναι γνωστός, θα πρέπει να γίνουν περισσότερες απο μία δεσμεύσεις μνήμης \emph{\en{malloc}}, διαδικασίας χρονοβόρας.
\item Γίνεται άσκοπη δέσμευση μνήμης για το διάνυσμα που θα αποθηκευτούν οι κόμβοι. Σε περίπτωση μεγάλου αριθμού κόμβων λίστας, αυτό μπορει να αποτελέσει πρόβλημα.

\end{enumerate}
\clearpage
\selectlanguage{english}
\begin{lstlisting}[ tabsize = 2, basicstyle=\small, language=C++, caption={\el{Λύση προβλήματος συνδεδεμένης λίστας με χρηση οδηγίας διαμοιρασμού εργασίας} }, frame = tb]{Name}
	 std::vector<Node *> nodes_(o.num_nodes_);  
     for (int i = 0; i < o.num_nodes_; ++i) {
         nodes_[i] = p;
         p = p->next_;
     }
	 #pragma omp parallel
     {
 		#pragma omp for schedule(static, 1) 
         for (int i = 0; i < o.num_nodes_; ++i) {
             fib(p->data_);
         }
     }
\end{lstlisting}
\selectlanguage{greek}

Τα παραπάνω προβλήματα επιλύονται με την χρήση διεργασιών. Ο κώδικας διέρχεται μια φορά από την λίστα με σκοπό την δημουργία μιας διεργασίας για κάθε κόμβο. Οι διεργασίας ξεκινούν να εκτελούνται αυτόματα, μόλις βρεθεί μή ενεργό νήμα, δηλαδή μόλις ένα νήμα φτάσει στο φράγμα κώδικα.

\ \\
\selectlanguage{english}
\begin{lstlisting}[ tabsize = 4, basicstyle=\small, language=C++, caption={\el{Λύση προβλήματος συνδεδεμένης λίστας με χρηση οδηγίας διαμοιρασμού εργασίας} }, frame = tb]{Name}
   #pragma omp parallel
   {
   	#pragma omp single
   	{
   		Node *p = head;
   		while (p) {
   			#pragma omp task firstprivate(p)
   				fib(p->data_);
   			p = p->next_;
   		}
   	}
   }

\end{lstlisting}

\clearpage
\selectlanguage{greek}

\subsubsection{ Αποτελέσματα εκτέλεσης}
\subparagraph{}
Στα παρακάτω διαγράμματα γίνεται σύγκριση με αποτελέσματα απο του αλγόριθμου με διαφορετικές μεθόδους.

\begin{tikzpicture}
\begin{axis}[
    title={Τίτλος},
    xlabel={Αριθμός κόμβων},
    legend cell align = {left},
    ylabel={Χρόνος εκτέλεσης \emph{\en{sec}}},
    xmin=30, xmax=90,
    ymin=0, ymax=7,
    xtick={0, 30, 40, 50, 60, 70, 80, 90},
    ytick={0,1,2,3,4,5,6, 7},
    legend pos= outer north east,
    ymajorgrids=true,
    width = 0.65\textwidth,
    grid style=dashed,
]

\addplot[
    color=blue,
    mark=triangle,
    ]
    coordinates {
    (30,1.92)(40,2.55)(50,3.16)(60,3.8)(70,4.41)(80,5.03)(90,6.65)
    };
    \addlegendentry{Σειριακή}


    \addplot[
    color=red,
    mark=*,
    ]
    coordinates {
    (30,1.92)(40,2.46)(50,3.06)(60,3.73)(70,4.59)(80,4.91)(90,5.51)
    };
	\addlegendentry{\en{c1 - Threads} 1}

        \addplot[
    color=green,
    mark=square,
    ]
    coordinates {
    (30,0.928)(40,1.26)(50,1.60)(60,1.85)(70,2.17)(80,2.66)(90,2.76)
    };
	\addlegendentry{\en{c1 - Threads} 2}
            \addplot[
    color=black,
    mark=circle,
    ]
    coordinates {
    (30,0.62)(40,0.91)(50,1.05)(60,1.25)(70,1.49)(80,1.69)(90,2.01)
    };
	\addlegendentry{\en{c1 - Threads} 3}
	
            \addplot[
    color=black,
    mark=x,
	mark options={fill=white},
    ]
    coordinates {
    (30,0.49)(40,0.64)(50,0.81)(60,0.94)(70,1.11)(80,1.32)(90,1.42)
    };
	\addlegendentry{\en{c1 - Threads} 4}
\end{axis}
\end{tikzpicture}



\begin{tikzpicture}
\begin{axis}[
    title={Τίτλος},
    xlabel={Αριθμός κόμβων},
    ylabel={Χρόνος εκτέλεσης \emph{\en{sec}}},
    xmin=30, xmax=90,
    ymin=0, ymax=7,
    legend cell align = {left},
    xtick={0, 30, 40, 50, 60, 70, 80, 90},
    ytick={0,1,2,3,4,5,6, 7},
    legend pos= outer north east,
    ymajorgrids=true,
    width = 0.65\textwidth,
    grid style=dashed,
]

\addplot[
    color=blue,
    mark=triangle,
    ]
    coordinates {
    (30,1.92)(40,2.55)(50,3.16)(60,3.8)(70,4.41)(80,5.03)(90,6.65)
    };
    \addlegendentry{Σειριακή}


    \addplot[
    color=red,
    mark=*,
    ]
    coordinates {
    (30,1.88436)(40,2.57883)(50,3.17825)(60,3.79555)(70,4.31703)(80,4.92457)(90,5.71748)
    };
	\addlegendentry{\en{tasks - Threads} 1}

        \addplot[
    color=green,
    mark=square,
    ]
    coordinates {
    (30,1.00991)(40,1.32349)(50,1.58554)(60,1.91951)(70,2.26585)(80,2.57923)(90,2.80001)
    };
	\addlegendentry{\en{tasks - Threads} 2}
            \addplot[
    color=black,
    mark=circle,
    ]
    coordinates {
    (30,0.670711)(40,0.886081)(50,1.07452)(60,1.28165)(70,1.51305)(80,1.70403)(90,1.88726)
    };
	\addlegendentry{\en{tasks - Threads} 3}
	
            \addplot[
    color=black,
    mark=x,
	mark options={fill=white},
    ]
    coordinates {
    (30,0.511133)(40,0.693767)(50,0.824834)(60,0.978441)(70,1.13911)(80,1.32485)(90,1.44752)
    };
	\addlegendentry{\en{tasks - Threads} 4}
\end{axis}
\end{tikzpicture}
	
\clearpage

\selectlanguage{greek}
\subsubsection{Συμπεράσματα και παρατηρήσεις}
\subparagraph{}
\clearpage
\subsection{Πρόβλημα πολλαπλασιασμού πινάκων}
\subparagraph{}
Σε αυτό το πρόβλημα γίνεται προσπάθεια επίλυσης πολλαπλασιασμού διδιάστατων πινάκων, αποτελούμενων απο τυχαίους ακεραίους:

$$C[K][M] = A[K][N] * B[N][M]$$

Για απλούστευση κώδικα, οι πίνακες που χρησιμοποιήθηκαν είναι τετραγωνικοί.

\selectlanguage{english}
\begin{lstlisting}[ tabsize = 2, basicstyle=\small, language=C++, caption={\el{Σειριακή μέθοδος πολλαπλασιασμού πινάκων}}, frame = tb]{Name}
    for (int i = 0; i < K; ++i) {
        for (int j = 0; j < M; ++j) {
            int tmp = 0;
            for (int k = 0; k < N; ++k) {
                tmp += A[i][k] * B[k][j];
            }
            C[i][j] = tmp;
        }
    }
\end{lstlisting}

\begin{lstlisting}[ tabsize = 2, basicstyle=\small, language=C++, caption={\el{Παράλληλος πολλαπλασιασμός με χρήση οδηγίας διαμοιρασμού βρόγχου}}, frame = tb]{Name}

  #pragma omp parallel for
     for (int i = 0; i < r1; ++i) {
         for (int j = 0; j < c2; ++j) {
             int tmp = 0;
             for (int k = 0; k < c1; ++k) {
                 tmp += A[i][k] * B[k][j];
             }
             C[i][j] = tmp;
         }
     }
\end{lstlisting}
\selectlanguage{greek}
Αλλη λύση του προβλήματος, είναι με διαχωρισμό των πινάκων σε υποπίνακες, και πολλαπλασιασμού των υποπινάκων ως ξεχωριστές διεργασίες\cite{examplesopm45}.

\selectlanguage{english}
\begin{lstlisting}[ tabsize = 2, basicstyle=\small, language=C++, caption={\el{Πολλαπλασιασμός πινάκων με χρήση διεργασιών (Πολλαπλασιασμός με χρήση υποπινάκων)}}, frame = tb]{Name}
void matmul(int N, int BS, int **A, int **B, int **C)
{
	int i, j, k, ii, jj, kk;
	for (i = 0; i < N; i+=BS) {
    	for (j = 0; j < N; j+=BS) {
        	for (k = 0; k < N; k+=BS) {
			// Note 1: i, j, k, A, B, C are firstprivate by default
			// Note 2: A, B and C are just pointers
			#pragma omp task private(ii, jj, kk) \
            	depend (in: A[i:BS][k:BS], B[k:BS][j:BS] ) \
            	depend (inout: C[i:BS][j:BS] )
			for (ii = i; ii < i+BS; ii++ )
				for (jj = j; jj < j+BS; jj++ )
					for (kk = k; kk < k+BS; kk++ )
						C[ii][jj] = C[ii][jj] + A[ii][kk] * B[kk][jj];
            }
        
        }
    }
}


\end{lstlisting}
\selectlanguage{greek}

\clearpage

\subsubsection{ Αποτελέσματα εκτέλεσης}
\subparagraph{}
Συγκριτικά διαγράματα εκτελέσης των τριών μεθόδων με διαφορετικά μεγέθη πινάκων αναφέρονται παρακάτω.
\selectlanguage{greek}

\begin{tikzpicture}
\begin{axis}[
    title={Τίτλος},
    xlabel={Μέγεθος τετραγωνικού πίνακα},
    ylabel={Χρόνος εκτέλεσης \emph{\en{sec}}},
    xmin=400, xmax=1500,
    ymin=0, ymax=100,
    legend cell align = {left},
    xtick={400, 500, 600, 700, 800, 900. 1000, 1100, 1200, 1300, 1400, 1500},
    ytick={0,10,20,30,40,50,60, 70, 80, 90, 100},
    legend pos= outer north east,
    ymajorgrids=true,
    width = 0.65\textwidth,
                       % label style={font=\small},
                    tick label style={font=\tiny} ,
    grid style=dashed,
]

\addplot[
    color=blue,
    mark=triangle,
    ]
    coordinates {
    (400,0.887962)(800,7.822277)(900,12.390038)(1000,19.337082)(1100,26.733613)(1200,37.184982)
    (1400,63.753732 )(1500,76.393354 )
    };
    \addlegendentry{Σειριακή}
    
    \addplot[
    color=red,
    mark=*,
    ]
    coordinates {
    (400,0.865326)(800,7.71938)(900,12.2757)(1000,18.947)(1100,26.1026)(1200,35.3035)
    (1400,60.8442 )(1500, 72.6156)
    };
    \addlegendentry{\en{omp for - Threads 1}}
    
        \addplot[
    color=green,
    mark=*,
    ]
    coordinates {
    (400,0.442599)(800,3.86485)(900,6.1835)(1000,9.51565)(1100,13.0426)(1200,18.5994)
    (1400,32.9676)(1500,37.6549)
    };
    \addlegendentry{\en{omp for - Threads 2}}
    
        \addplot[
    color=black,
    mark=*,
    ]
    coordinates {
    (400,0.298103)(800,2.59139)(900,4.11546)(1000,6.34454)(1100,8.81577)(1200,12.4621)
    (1400,22.4629)(1500,25.3575)
    };
    \addlegendentry{\en{omp for - Threads 3}}
            \addplot[
    color=black,
    mark=square,
    ]
    coordinates {
    (400,0.237109)(800,1.9512)(900,3.11276)(1000,4.80297)(1100,6.61739)(1200,9.37975)
    (1400,16.9502)(1500,19.0139)
    };
    \addlegendentry{\en{omp for - Threads 4}}
\end{axis}
\end{tikzpicture}

\begin{tikzpicture}
\begin{axis}[
    title={Τίτλος},
    xlabel={Μέγεθος τετραγωνικού πίνακα},
    ylabel={Χρόνος εκτέλεσης \emph{\en{sec}}},
    xmin=400, xmax=1500,
    ymin=0, ymax=100,
    legend cell align = {left},
    xtick={400, 500, 600, 700, 800, 900. 1000, 1100, 1200, 1300, 1400, 1500},
    ytick={0,10,20,30,40,50,60, 70, 80, 90, 100},
    legend pos= outer north east,
    ymajorgrids=true,
    width = 0.65\textwidth,
                       % label style={font=\small},
                    tick label style={font=\tiny} ,
    grid style=dashed,
]

\addplot[
    color=blue,
    mark=triangle,
    ]
    coordinates {
    (400,0.887962)(800,7.822277)(900,12.390038)(1000,19.337082)(1100,26.733613)(1200,37.184982)
    (1400,63.753732 )(1500,76.393354 )
    };
    \addlegendentry{Σειριακή}
    
        \addplot[
    color=red,
    mark=*,
    ]
    coordinates {
    (400,1.01848)(800,8.38982)(900,12.032)(1000,16.6239)(1100,22.1572)(1200,28.7798)
    (1400, 45.724)(1500, 56.1838)
    };
    \addlegendentry{\en{tasks - Threads 1}}
    
        \addplot[
    color=green,
    mark=*,
    ]
    coordinates {
    (400,0.520894)(800,4.2116)(900,6.03741)(1000,8.37473)(1100,11.1121)(1200,14.4275)
    (1400,22.9955)(1500,28.1515)
    };
    \addlegendentry{\en{tasks - Threads 2}}
    
        \addplot[
    color=black,
    mark=*,
    ]
    coordinates {
    (400,0.351305)(800,2.82444)(900,4.03187)(1000,5.57483)(1100,7.42902)(1200,9.6402)
    (1400,15.2952)(1500,18.8127)
    };
    \addlegendentry{\en{tasks - Threads 3}}
            \addplot[
    color=black,
    mark=square,
    ]
    coordinates {
    (400,0.272871)(800,2.13485)(900,3.04206)(1000,4.18663)(1100,5.58043)(1200,7.24081)
    (1400,11.4919)(1500,14.1197)
    };
    \addlegendentry{\en{tasks - Threads 4}}
\end{axis}
\end{tikzpicture}
\selectlanguage{greek}


\subsection{Υπολογισμού π}
\subparagraph{}

Να γράψω τα βήματα του ιντεγκρατιον και τι εινα ιτο ιντεγκρασιον.


\selectlanguage{greek}
\begin{tikzpicture}
\begin{axis}[
    title={Υπολογισμός π},
    xlabel={Βήματα ολοκλήρωσης},
    ylabel={Χρόνος εκτέλεσης \emph{\en{sec}}},
    xmin=1000000, xmax=1024000000,
    ymin=0, ymax=50,
    legend cell align = {left},
    xtick={1000000, 2000000, 4000000, 8000000, 16000000, 32000000 , 64000000, 128000000, 256000000, 512000000,1024000000 },
    ytick={0,10,20,30,40,50},
    legend pos= outer north east,
    ymajorgrids=true,
    width = 0.65\textwidth,
                       % label style={font=\small},
                    tick label style={font=\tiny, rotate=90} ,
    grid style=dashed,
]
\addplot[
    color=black,
    mark=square,
    ]
    coordinates {
    (1000000,0.0338001)(2000000,0.0572331)(4000000,0.111447)(8000000,0.221375)(16000000,0.426203)(32000000,0.870774)(64000000,1.73383)(128000000,3.4648)(256000000,6.92544)(512000000,13.8453)(1024000000,27.5772)
    };
    \addlegendentry{\en{c1 - Threads 1}}
    
    \addplot[
    color=blue,
    mark=*,
    ]
    coordinates {
    (1000000,0.0405066)(2000000,0.0819236)(4000000,0.210316)(8000000,0.300327)(16000000,0.579097)(32000000,1.24199)(64000000,3.20186)(128000000,4.52216)(256000000,10.8937)(512000000,18.8005)(1024000000,37.8296)
    };
    \addlegendentry{\en{c1 - Threads 2}}
    
\addplot[
    color=red,
    mark=triangle,
    ]
    coordinates {
    (1000000,0.0436258)(2000000,0.080974)(4000000,0.175825)(8000000,0.350371)(16000000,0.664527)(32000000,1.23263)(64000000,2.80946)(128000000,4.75729)(256000000,11.0764)(512000000,19.2894)(1024000000,44.2821)
    };
    \addlegendentry{\en{c1 - Threads 3}}
    
    \addplot[
    color=purple,
    mark=x,
    ]
    coordinates {
    (1000000,0.0532062)(2000000,0.101833)(4000000,0.17677)(8000000,0.325933)(16000000,0.677992)(32000000,1.31605)(64000000,2.74918)(128000000,5.08923)(256000000,11.1267)(512000000,21.8488)(1024000000,44.0514)
    };
    \addlegendentry{\en{c1 - Threads 4}}
\end{axis}
\end{tikzpicture}


\begin{tikzpicture}
\begin{axis}[
    title={Υπολογισμός π},
    xlabel={Βήματα ολοκλήρωσης},
    ylabel={Χρόνος εκτέλεσης \emph{\en{sec}}},
    xmin=1000000, xmax=1024000000,
    ymin=0, ymax=50,
    legend cell align = {left},
    xtick={1000000, 2000000, 4000000, 8000000, 16000000, 32000000 , 64000000, 128000000, 256000000, 512000000,1024000000 },
    ytick={0,10,20,30,40,50},
    legend pos= outer north east,
    ymajorgrids=true,
    width = 0.65\textwidth,
                       % label style={font=\small},
                    tick label style={font=\tiny, rotate=90} ,
    grid style=dashed,
]
\addplot[
    color=black,
    mark=square,
    ]
    coordinates {
    (1000000,0.0308711)(2000000,0.0573471)(4000000,0.113584)(8000000,0.218868)(16000000,0.435251)(32000000,0.843545)(64000000,1.68444)(128000000,3.40944)(256000000,6.82742)(512000000,13.6523)(1024000000,27.4227)
    };
    \addlegendentry{\en{c2 - Threads 1}}
    
    \addplot[
    color=blue,
    mark=*,
    ]
    coordinates {
    (1000000,0.0317523)(2000000,0.0577772)(4000000,0.107732)(8000000,0.218209)(16000000,0.45667)(32000000,0.844152)(64000000,1.89964)(128000000,4.94725)(256000000,7.2048)(512000000,15.6844)(1024000000,29.1235)
    };
    \addlegendentry{\en{c2 - Threads 2}}
    
\addplot[
    color=red,
    mark=triangle,
    ]
    coordinates {
    (1000000,0.0310962)(2000000,0.0537551)(4000000,0.101099)(8000000,0.192268)(16000000,0.38278)(32000000,0.757889)(64000000,1.51423)(128000000,3.55208)(256000000,6.08762)(512000000,13.4538)(1024000000,29.8761)
    };
    \addlegendentry{\en{c2 - Threads 3}}
    
    \addplot[
    color=purple,
    mark=x,
    ]
    coordinates {
    (1000000,0.0159759)(2000000,0.0215767)(4000000,0.0348523)(8000000,0.0612981)(16000000,0.115281)(32000000,0.232632)(64000000,0.444231)(128000000,0.86781)(256000000,1.73724)(512000000,3.44095)(1024000000,6.86932)
    };
    \addlegendentry{\en{c2 - Threads 4}}
\end{axis}
\end{tikzpicture}

\begin{tikzpicture}
\begin{axis}[
    title={Υπολογισμός π},
    xlabel={Βήματα ολοκλήρωσης},
    ylabel={Χρόνος εκτέλεσης \emph{\en{sec}}},
    xmin=1000000, xmax=1024000000,
    ymin=0, ymax=50,
    legend cell align = {left},
    xtick={1000000, 2000000, 4000000, 8000000, 16000000, 32000000 , 64000000, 128000000, 256000000, 512000000,1024000000 },
    ytick={0,10,20,30,40,50},
    legend pos= outer north east,
    ymajorgrids=true,
    width = 0.65\textwidth,
                       % label style={font=\small},
                    tick label style={font=\tiny, rotate=90} ,
    grid style=dashed,
]
\addplot[
    color=black,
    mark=square,
    ]
    coordinates {
    (1000000,0.0341946)(2000000,0.059996)(4000000,0.120679)(8000000,0.236262)(16000000,0.477396)(32000000,0.924997)(64000000,1.84543)(128000000,3.7328)(256000000,7.49432)(512000000,14.7111)(1024000000,29.4277)
    };
    \addlegendentry{\en{c3 - Threads 1}}
    
    \addplot[
    color=blue,
    mark=*,
    ]
    coordinates {
    (1000000,0.0238432)(2000000,0.035713)(4000000,0.0669172)(8000000,0.12263)(16000000,0.238332)(32000000,0.474312)(64000000,0.925321)(128000000,1.85284)(256000000,3.6848)(512000000,7.40389)(1024000000,14.8101)
    };
    \addlegendentry{\en{c3 - Threads 2}}
    
\addplot[
    color=red,
    mark=triangle,
    ]
    coordinates {
    (1000000,0.0176296)(2000000,0.0273897)(4000000,0.0485577)(8000000,0.0817514)(16000000,0.160832)(32000000,0.315039)(64000000,0.621319)(128000000,1.2484)(256000000,2.4588)(512000000,4.92288)(1024000000,10.001)
    };
    \addlegendentry{\en{c3 - Threads 3}}
    
    \addplot[
    color=purple,
    mark=x,
    ]
    coordinates {
    (1000000,0.0155331)(2000000,0.0244363)(4000000,0.0377688)(8000000,0.0670685)(16000000,0.124106)(32000000,0.247325)(64000000,0.474941)(128000000,0.93633)(256000000,1.8583)(512000000,3.69461)(1024000000,7.45551)
    };
    \addlegendentry{\en{c3 - Threads 4}}
\end{axis}
\end{tikzpicture}


\begin{tikzpicture}
\begin{axis}[
    title={Υπολογισμός π},
    xlabel={Βήματα ολοκλήρωσης},
    ylabel={Χρόνος εκτέλεσης \emph{\en{sec}}},
    xmin=1000000, xmax=1024000000,
    ymin=0, ymax=50,
    legend cell align = {left},
    xtick={1000000, 2000000, 4000000, 8000000, 16000000, 32000000 , 64000000, 128000000, 256000000, 512000000,1024000000 },
    ytick={0,10,20,30,40,50},
    legend pos= outer north east,
    ymajorgrids=true,
    width = 0.65\textwidth,
                       % label style={font=\small},
                    tick label style={font=\tiny, rotate=90} ,
    grid style=dashed,
]
\addplot[
    color=black,
    mark=square,
    ]
    coordinates {
    (1000000,0.0326437)(2000000,0.0627669)(4000000,0.119978)(8000000,0.237184)(16000000,0.467504)(32000000,0.919998)(64000000,1.86394)(128000000,3.71113)(256000000,7.35653)(512000000,14.819)(1024000000,29.1752)
    };
    \addlegendentry{\en{c4 - Threads 1}}
    
    \addplot[
    color=blue,
    mark=*,
    ]
    coordinates {
    (1000000,0.0243332)(2000000,0.0373347)(4000000,0.0671596)(8000000,0.123871)(16000000,0.238785)(32000000,0.468718)(64000000,0.934431)(128000000,1.8579)(256000000,3.70638)(512000000,7.41984)(1024000000,14.823)
    };
    \addlegendentry{\en{c4 - Threads 2}}
    
\addplot[
    color=red,
    mark=triangle,
    ]
    coordinates {
    (1000000,0.0178636)(2000000,0.0274041)(4000000,0.0469626)(8000000,0.086801)(16000000,0.160611)(32000000,0.316492)(64000000,0.62383)(128000000,1.24284)(256000000,2.48271)(512000000,4.94991)(1024000000,9.88145)
    };
    \addlegendentry{\en{c4 - Threads 3}}
    
    \addplot[
    color=purple,
    mark=x,
    ]
    coordinates {
    (1000000,0.0157718)(2000000,0.0228653)(4000000,0.0386606)(8000000,0.0663903)(16000000,0.124964)(32000000,0.247752)(64000000,0.481567)(128000000,0.944467)(256000000,1.859)(512000000,3.71203)(1024000000,7.42446)
    };
    \addlegendentry{\en{c4 - Threads 4}}
\end{axis}
\end{tikzpicture}

\begin{tikzpicture}
\begin{axis}[
    title={Υπολογισμός π},
    xlabel={Βήματα ολοκλήρωσης},
    ylabel={Χρόνος εκτέλεσης \emph{\en{sec}}},
    xmin=1000000, xmax=1024000000,
    ymin=0, ymax=50,
    legend cell align = {left},
    xtick={1000000, 2000000, 4000000, 8000000, 16000000, 32000000 , 64000000, 128000000, 256000000, 512000000,1024000000 },
    ytick={0,10,20,30,40,50},
    legend pos= outer north east,
    ymajorgrids=true,
    width = 0.65\textwidth,
                       % label style={font=\small},
                    tick label style={font=\tiny, rotate=90} ,
    grid style=dashed,
]
\addplot[
    color=black,
    mark=square,
    ]
    coordinates {
    (1000000,0.0339535)(2000000,0.0632102)(4000000,0.120177)(8000000,0.238885)(16000000,0.467136)(32000000,0.924733)(64000000,1.849)(128000000,3.6898)(256000000,7.37323)(512000000,14.7417)(1024000000,29.6053)
    };
    \addlegendentry{\en{c5 - Threads 1}}
    
    \addplot[
    color=blue,
    mark=*,
    ]
    coordinates {
    (1000000,0.024335)(2000000,0.0344515)(4000000,0.0675575)(8000000,0.125115)(16000000,0.238339)(32000000,0.468582)(64000000,0.928458)(128000000,1.85658)(256000000,3.69315)(512000000,7.37672)(1024000000,14.7462)
    };
    \addlegendentry{\en{c5 - Threads 2}}
    
\addplot[
    color=red,
    mark=triangle,
    ]
    coordinates {
    (1000000,0.0164176)(2000000,0.0281208)(4000000,0.0487261)(8000000,0.0858676)(16000000,0.161913)(32000000,0.315858)(64000000,0.622041)(128000000,1.23661)(256000000,2.46552)(512000000,4.92953)(1024000000,9.83403)
    };
    \addlegendentry{\en{c5 - Threads 3}}
    
    \addplot[
    color=purple,
    mark=x,
    ]
    coordinates {
    (1000000,0.0157494)(2000000,0.0247767)(4000000,0.0356085)(8000000,0.0663202)(16000000,0.131022)(32000000,0.247863)(64000000,0.478779)(128000000,0.938146)(256000000,1.85777)(512000000,3.69327)(1024000000,7.3935)
    };
    \addlegendentry{\en{c5 - Threads 4}}
\end{axis}
\end{tikzpicture}

\begin{tikzpicture}
\begin{axis}[
    title={Υπολογισμός π},
    xlabel={Βήματα ολοκλήρωσης},
    ylabel={Χρόνος εκτέλεσης \emph{\en{sec}}},
    xmin=1000000, xmax=1024000000,
    ymin=0, ymax=50,
    legend cell align = {left},
    xtick={1000000, 2000000, 4000000, 8000000, 16000000, 32000000 , 64000000, 128000000, 256000000, 512000000,1024000000 },
    ytick={0,10,20,30,40,50},
    legend pos= outer north east,
    ymajorgrids=true,
    width = 0.65\textwidth,
                       % label style={font=\small},
                    tick label style={font=\tiny, rotate=90} ,
    grid style=dashed,
]
\addplot[
    color=black,
    mark=square,
    ]
    coordinates {
    (1000000,0.0336593)(2000000,0.0600598)(4000000,0.118257)(8000000,0.23635)(16000000,0.463466)(32000000,0.918388)(64000000,1.83331)(128000000,3.65506)(256000000,7.30093)(512000000,14.6071)(1024000000,29.2407)
    };
    \addlegendentry{\en{c6 - Threads 1}}
    
    \addplot[
    color=blue,
    mark=*,
    ]
    coordinates {
    (1000000,0.0355824)(2000000,0.0627649)(4000000,0.119635)(8000000,0.233565)(16000000,0.237902)(32000000,0.464477)(64000000,0.919029)(128000000,1.83531)(256000000,3.66849)(512000000,7.30996)(1024000000,14.6068)
    };
    \addlegendentry{\en{c6 - Threads 2}}
    
\addplot[
    color=red,
    mark=triangle,
    ]
    coordinates {
    (1000000,0.0346595)(2000000,0.0630393)(4000000,0.121526)(8000000,0.233586)(16000000,0.236518)(32000000,0.465431)(64000000,0.922043)(128000000,1.37825)(256000000,3.67671)(512000000,7.31318)(1024000000,14.5934)
    };
    \addlegendentry{\en{c6 - Threads 3}}
    
    \addplot[
    color=purple,
    mark=x,
    ]
    coordinates {
    (1000000,0.0349871)(2000000,0.0628435)(4000000,0.119083)(8000000,0.235015)(16000000,0.236679)(32000000,0.466)(64000000,0.694128)(128000000,1.37762)(256000000,2.75137)(512000000,5.03613)(1024000000,10.9699)
    };
    \addlegendentry{\en{c6 - Threads 4}}
\end{axis}
\end{tikzpicture}
\selectlanguage{greek}

\clearpage
\subsection{Πρόβλημα ακολουθίας \en{Fibonacci}}
\subparagraph{}
\selectlanguage{greek}


Στο παρακάτω παράδειγμα, γίνεται ο υπολογισμός αριθμών της ακολουθίας Fibonacci.

\begin{tikzpicture}
\begin{axis}[
    title={Υπολογισμός ακολουθίας \en{fibonacci}},
    xlabel={Ακέραιο όρισμα υπολιγσμού ακολουθίας},
    ylabel={Χρόνος εκτέλεσης \emph{\en{sec}}},
    xmin=20, xmax=33,
    ymin=0, ymax=14,
    legend cell align = {left},
    xtick={20, 21, 22, 23, 24, 25, 26, 27, 28, 29, 30, 31, 32, 33},
    ytick={0, 1, 2, 3, 4, 5, 6, 7, 8, 9, 10, 11, 12, 13, 14},
    legend pos= outer north east,
    ymajorgrids=true,
    width = 0.65\textwidth,
                       % label style={font=\small},
                    tick label style={font=\tiny, rotate=90} ,
    grid style=dashed,
]

    
    \addplot[
    color=green,
    mark=square,
    ]
    coordinates {
    (20,0.000122097)(21,0.000272111)(22,0.000430536)(23,0.000409333)(24,0.00113057)(25,0.00141876)(26,0.00298631)(27,0.00495223)(28,0.00562632)(29,0.00936043)(30,0.0116785)
    (31,0.0184183)(32,0.0275102)(33,0.0401246)
    };
    \addlegendentry{Σειριακα}
    
        \addplot[
    color=purple,
    mark=*,
    ]
    coordinates {
    (20,0.00890836)(21,0.0108529)(22,0.0133473)(23,0.0193422)(24,0.0294923)(25,0.0440121)(26,0.0698217)(27,0.108529)(28,0.167679)(29,0.272621)(30,0.441719)
    (31,0.706872)(32,1.1736)(33,1.82593)
    };
    \addlegendentry{\en{Tasks - Threads 1}}
    
            \addplot[
    color=blue,
    mark=x,
    ]
    coordinates {
    (20,0.0208804)(21,0.0307545)(22,0.0459824)(23,0.0724208)(24,0.112487)(25,0.18253)(26,0.285626)(27,0.461611)(28,0.739798)(29,1.20895)(30,1.84069)
    (31,2.83749)(32,5.00284)(33,7.07542)
    };
    \addlegendentry{\en{Tasks - Threads 2}}
    
                \addplot[
    color=purple,
    mark=triangle,
    ]
    coordinates {
    (20,0.0301488)(21,0.0389012)(22,0.0590625)(23,0.0946988)(24,0.149659)(25,0.235231)(26,0.394644)(27,0.609937)(28,1.01815)(29,1.79491)(30,2.60338)
    (31,4.43093)(32,7.3724)(33,11.6082)
    };
    \addlegendentry{\en{Tasks - Threads 3}}
                    \addplot[
    color=black,
    mark=x,
    ]
    coordinates {
    (20,0.0363369)(21,0.055894)(22,0.0776081)(23,0.140359)(24,0.206626)(25,0.327944)(26,0.548839)(27,0.785649)(28,1.39474)(29,2.26885)(30,3.63612)
    (31,5.80738)(32,10.5055)(33,15.6925)
    };
    \addlegendentry{\en{Tasks - Threads 4}}
\end{axis}
\end{tikzpicture}
\selectlanguage{greek}
