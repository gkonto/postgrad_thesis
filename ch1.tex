\section{Εισαγωγή}
\subsection{Συνοπτικά για το \en{OpenMP}}
\subparagraph{}
Το \en{OpenMP} είναι μια Διεπαφή Προγραμματισμού Εφαρμογών \en{(API)} που χρησιμοποιείται για παραλληλοποίηση συστημάτων μοιραζόμενης μνήμης, από λογισμικά γραμμένα σε γλώσσες \en{C/C++} και \en{Fortran}. Η διεπαφή αποτελείται από τα παρακάτω σύνολα\cite{thenextstep20}:
\begin{itemize}
    \item οδηγιών\en{(directives)} για τον μεταγλωττιστή που έχουν ως στόχο τον καθορισμό και τον έλεγχο της παραλληλοποίησης.
    \item ενσωματωμένων συναρτήσεων της βιβλιοθήκης \en{OpenMP}.
    \item μεταβλητών περιβάλλοντος.
\end{itemize}

Οι εντολές παραλληλοποίησης εφαρμόζονται στο τμήμα του κώδικα που ακολουθεί της οδηγίας. Κάθε κατασκευή ξεκινάει με \emph{\en{\#pragma omp}} και ακολουθούν οι οδηγίες για το μεταγλωττιστή. Το δομημένο τμήμα κώδικα μπορεί να αποτελείται από μια απλή εντολή ή ένα σύνολο απλών εντολών\cite{ompsyntaxrefguide}. Οι εντολές που βρίσκονται εντός της περιοχής παράλληλου κώδικα, εκτελούνται από όλα τα νηματα που δημιουργούνται κατά τη διάρκεια της παραλληλοποίησης. Η παραλληλοποίηση ολοκληρώνεται με το πέρας της εκτέλεσης των εντολών εντός αυτής της περιοχής.
\ \\
\selectlanguage{english}
\begin{lstlisting}[language=C++, caption={\el{Γραμματική σύνταξης οδηγίας} OpenMP}, frame = single, xleftmargin=.1\textwidth]
#pragma omp (directive) [clause[[,] clause]...] new-line
\end{lstlisting} 
\selectlanguage{greek}
\ \\
\par
Με τη χρήση του \emph{\en{OpenMP}} οι εφαρμογές εκμεταλλεύονται την ύπαρξη πολλαπλών επεξεργαστικών μονάδων, με σκοπό την επίτευξη αύξησης των υπολογιστικών επιδόσεων και μείωση του απαιτούμενου χρόνου εκτέλεσης της εφαρμογής. Ο παράλληλος προγραμματισμός μπορεί να ιδωθεί ως ειδική περίπτωση ταυτόχρονου προγραμματισμού, όπου η εκτέλεση γίνεται πραγματικά παράλληλα και όχι ψευδοπαράλληλα\cite{googleparallelprog}.
\ \\
\selectlanguage{english}
\begin{lstlisting}[language=C++, caption={\el{Παράδειγμα παράλληλου κώδικα} OpenMP}, frame=tb]
#include <omp.h>    // OpenMP include file
#include <stdio.h>  // Include input-output library

int main(void) {
	#pragma omp parallel	
	{
		int id = omp_get_thread_num();
		std::cout << "Hello " << id;
		std::cout << "world " << std::endl;
	}
}
\end{lstlisting}
\selectlanguage{greek}

\newpage
\subsection{Σκοπός – Στόχοι}
\subparagraph{}
{\LARGE \en{TODO}}\\
Σκόπος είναι η ανάλυση του OpenMP και των νεων χαρακτηρίστικών του. Να δουμε τα οφέλει του API και στόχος 
ειναι η σύγκριση των νεων εργαλείων και των παλιών και του σειριακου κώδικα για να δούμε τι κερδίζουμε και τι όχι.

\subsection{Διάρθρωση της μελέτης}
\subparagraph{}
{\LARGE \en{TODO}}\\
Εδώ περιγράφουμε τα κεφάλαια της διπλωματικής. Συνήθως η ενότητα αυτή έχει την παρακάτω μορφή (δεν θα σας πάρει πάνω από 1 μεγάλη παράγραφο): Εργασίες σχετικές με το αντικείμενο της διπλωματικής παρουσιάζονται στο Κεφάλαιο 2. Το Κεφάλαιο 3 συζητά…. Στο Κεφάλαιο 4 αναπτύσσουμε …κλπ.

