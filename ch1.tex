\section{Εισαγωγή}
\subsection{Συνοπτικά για το \en{OpenMP}}
\subparagraph{}
Το \en{OpenMP} είναι μια Διεπαφή Προγραμματισμού Εφαρμογών \en{(API)} που χρησιμοποιείται για παραλληλοποίηση συστημάτων μοιραζόμενης μνήμης, από λογισμικά γραμμένα σε γλώσσες \en{C/C++} και \en{Fortran}. Η διεπαφή αποτελείται από τα παρακάτω σύνολα\cite{thenextstep20}:
\begin{itemize}
    \item οδηγιών\en{(directives)} για τον μεταγλωττιστή που έχουν ως στόχο τον καθορισμό και τον έλεγχο της παραλληλοποίησης.
    \item ενσωματωμένων ρουτίνων της βιβλιοθήκης \en{OpenMP}.
    \item μεταβλητών περιβάλλοντος.
\end{itemize}

Οι εντολές παραλληλοποίησης εφαρμόζονται στο τμήμα του κώδικα που ακολουθεί της οδηγίας. Κάθε κατασκευή ξεκινάει με \emph{\en{\#pragma omp}} και ακολουθούν οι οδηγίες για το μεταγλωττιστή. Το δομημένο τμήμα κώδικα μπορεί να αποτελείται από μια απλή εντολή ή ένα σύνολο απλών εντολών\cite{ompsyntaxrefguide}. Οι εντολές που βρίσκονται εντός της περιοχής παράλληλου κώδικα, εκτελούνται από όλα τα νηματα που δημιουργούνται κατά τη διάρκεια της παραλληλοποίησης. Η παραλληλοποίηση ολοκληρώνεται με το πέρας της εκτέλεσης των εντολών εντός αυτής της περιοχής.
\ \\
\selectlanguage{english}
\begin{lstlisting}[language=C++, caption={\el{Γραμματική σύνταξης οδηγίας} OpenMP}, frame = single, xleftmargin=.1\textwidth]
#pragma omp (directive) [clause[[,] clause]...] new-line
\end{lstlisting} 
\selectlanguage{greek}
\ \\
\par
Με τη χρήση του \emph{\en{OpenMP}} οι εφαρμογές εκμεταλλεύονται την ύπαρξη πολλαπλών επεξεργαστικών μονάδων, με σκοπό την επίτευξη αύξησης των υπολογιστικών επιδόσεων και μείωση του απαιτούμενου χρόνου εκτέλεσης της εφαρμογής. Ο παράλληλος προγραμματισμός μπορεί να ιδωθεί ως ειδική περίπτωση ταυτόχρονου προγραμματισμού, όπου η εκτέλεση γίνεται πραγματικά παράλληλα και όχι ψευδοπαράλληλα\cite{googleparallelprog}.
\ \\
\selectlanguage{english}
\begin{lstlisting}[language=C++, caption={\el{Παράδειγμα παράλληλου κώδικα} OpenMP}, frame=tb]
#include <omp.h>    // OpenMP include file
#include <stdio.h>  // Include input-output library

int main(void) {
	#pragma omp parallel	
	{
		int id = omp_get_thread_num();
		std::cout << "Hello " << id;
		std::cout << "world " << std::endl;
	}
}
\end{lstlisting}
\selectlanguage{greek}

\newpage
\subsection{Σκοπός – Στόχοι}
\subparagraph{}
Σκοπός της παρούσας εργασίας είναι η ανάλυση της διεπαφής του \emph{\en{OpenMP}}, δίνοντας μεγαλύτερη βαρύτητα στις εκδόσεις από 3.0 έως 4.5 και στα χαρακτηριστικά και τις δυνατότητες που εισήχθησαν σε αυτές. Αναλύονται σε θεωρητικό επίπεδο οι νέες οδηγίες και φράσεις των εκδόσεων αυτών, ενώ γίνεται μια προσπάθεια υλοποίησης και συγκριτικής μελέτης απλών και σύνθετων προβλημάτων, επιλυόμενων με διαφορετικές μεθόδους και παραλλαγές, που βασίζονται στα καινούργια χαρακτηριστικά της διεπαφής. Στόχος είναι η σαφής κατανόηση των εισαγόμενων χαρακτηριστικών της διεπαφής στις συγκεκριμένες εκδόσεις, η εξαγωγή συμπερασμάτων μέσα από τις υλοποιήσεις των προβλημάτων αλλά και η σύγκριση των διαφορετικών μεθόδων επίλυσης κάθε επιμέρους προβλήματος με βάση τις επιδόσεις τους.

\subsection{Διάρθρωση της μελέτης}
\subparagraph{}
Στα κεφάλαια που ακολουθούν γίνεται μια σύντομη περιγραφή του μοντέλου προγραμματισμού της διεπαφής \emph{\en{OpenMP}} και της αλληλεπίδρασης των παραγόμενων νημάτων με το περιβάλλον δεδομένων. Ακόμη γίνεται μια σύντομη αναδρομή σε έννοιες απαραίτητες για την μελέτη των νέων χαρακτηριστικών που εισήχθησαν στις επόμενες εκδόσεις μετά την 2.5.
Η αναδρομή περιορίζεται κυριώς στις οδηγίες και τις φράσεις που υπάρχουν στις παλαιότερες εκδόσεις της διεπαφής.
Στη συνέχεια ακολουθεί ανάλυση σε θεωρητικό υπόβαθρο, των σημαντικότερων εννοιών που εισήχθησαν στις εκδόσεις 3.0 - 4.5 όπως \emph{\en{tasking, offloading, vectorization}} κ.α.
Αμέσως μετά, ακολουθεί η υλοποίηση, ο σχολιασμός και η συγκριτική μελέτη απλών και σύνθετων προβλήμάτων. Τέλος, ακολουθεί η εξαγωγή συμπερασμάτων βασιζόμενων στις προηγούμενες υλοποιήσεις.